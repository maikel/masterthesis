\documentclass{beamer}
\usepackage[ngerman]{babel}
\usepackage[utf8]{inputenc}
\usepackage[T1]{fontenc}
\usepackage{xcolor}
\usepackage{amsmath}
\usepackage{amsthm}
\usepackage{amssymb}

\newcommand{\eps}{\varepsilon}
\renewcommand{\u}[1]{u^{(#1)}}
\newcommand{\tu}[1]{\tilde u^{(#1)}}
\newcommand{\R}{\mathbb R}
\newcommand{\Rp}{\mathbb R^+_0}
\newcommand{\N}{\mathbb N}
\newcommand{\Z}{\mathbb Z}
\newcommand{\Cinf}{C^{\infty}\left( \Rp \times \R \right)}
\DeclareMathOperator{\id}{Id}
\DeclareMathOperator{\Ima}{Im}
\DeclareMathOperator{\Ree}{Re}
\DeclareMathOperator{\err}{err}
\newcommand{\dd}{\,\mathrm d}
\newcommand{\pkt}{\, \cdot \,}
\newcommand{\abs}[1]{\left| #1 \right|}
\newcommand{\norm}[1]{\left\lVert #1 \right\rVert}

\usetheme{metropolis}           % Use metropolis theme
\title{Multiskalenasymptotik diskreter Verfahren}
\date{\today}
\author{Maikel Nadolski}
\institute{Freie Universität Berlin}
\begin{document}
  \maketitle
  \section{Die Problemstellung}

  \begin{frame}{Die PDE und das numerische Verfahren}
    Wir betrachten die eindimensionale Transportgleichung mit konstanter 
    Geschwindigkeit
    \begin{align}\label{eq:pde}
      \partial_t u(t,x) + a \cdot \partial_x u(t,x) &= 0\\
      u(0,x) &= u_0(x) \quad \forall x \in \R
    \end{align}
    mit dem Gitter
    $G^\lambda_h = \left\{ (ih, \lambda jh) \, \mid \, (i,j) \in \N^2 \right\}$
    und das Upwind-Verfahren
    \begin{align}
      v^{n+1}_i &= v^n_i - \lambda ( v^n_i - v^n_{i-1} )\\
      v^0_i &= u_0(x_i), \quad \forall (x_i, 0) \in G^\lambda_h
    \end{align}
  \end{frame}

  \begin{frame}{Asymptotik am Upwindverfahren}
    Wenn man in $h$ den asymptotischen Ansatz 
    \begin{align}
      v^n_i = u_0(t_n, x_i) + h u_1(t_n, x_i) + O(h^2)
    \end{align}
    macht und in \eqref{eq:pde} einsetzt, so erhält man für $u_0, u_1$ die Gleichungen
  \end{frame}
\end{document}