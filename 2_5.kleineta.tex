%!TEX root=thesis.tex

$h \ll \eta$ führt fundamental zu dem Problem, dass kleine Rundungsfehler zu jeder noch so kleinen Zeit $t > 0$ beliebig groß werden und die richtige Lösung komplett überdecken.
Wir rechnen in der Praxis jedoch mit einem positvem $h > 0$, für das eventuell auch $h \sim \eta$ gelten kann.
Konkrete numerische Realisierungen existieren und entwickeln Oszillationen für feste $h$ zu positiven Zeiten $t > 0$.
Gerade für Probleme, bei denen numerische Daten nur in einem ``kurzem'' Zeitinterval unter instabilen Bedingungen gerechnet werden, könnte es also interessante Einblicke geben, das asymptotische Verhalten eines Verfahren für ``kleine'' Störungen der CFL Zahl zu kennen.
Der bisherige Ansatz hat dies ignoriert und obwohl Abbildung \ref{fig:transport:osz:max_error} zeigt, dass \eqref{eq:transport:osz:loesung} bereits gute Approximationen liefert, werden solche Überlegungen spätestens für die Advektion mit variabler Geschwindigkeit oder auch nichtlineare Probleme von Nöten sein.
Daher betrachten wir nun die Koppelung $\eta = h$ und setzen $\lambda = 1 + h$ in die Gleichungen ein \eqref{eq:transport:diskret:o1} bis \eqref{eq:transport:diskret:oh3} ein:

\subsection*{Allgemeine Gleichungen}

\vspace{0.4cm}
\noindent \emph{in der Ordnung $O(1)$:}
\begin{align}\label{eq:transport:kleineta:o1}
u_0(n+1, i, t_n, x_i) - u_0(n, i-1, t_n, x_i) = 0
\end{align}\\

\noindent \emph{in der Ordnung $O(h)$:}
\begin{align}\label{eq:transport:kleineta:oh}
\partial_t u_0(n+1, i, t_n, x_i) + \partial_x u_0(n, i-1, t_n, x_i) =
\begin{split}
&- \bigl( u_0(n, i, t_n, x_i) - u_0(n, i-1, t_n, x_i) \bigr)\\
&- \bigl( u_1(n+1, i, t_n, x_i) - u_1(n, i-1, t_n, x_i) \bigr)
\end{split}
\end{align}\\

\noindent \emph{in der Ordnung $O(h^2)$:}
{\small
\begin{align}\label{eq:transport:kleineta:oh2}
\begin{split}
\partial_t u_1(n+1, i, t_n, x_i) + \partial_x u_1(n, i-1, t_n, x_i) +\\
\partial_t u_0(n+1, i, t_n, x_i) + \partial_x u_0(n, i-1, t_n, x_i)
\end{split}
&=
\begin{split}
&\frac{1}{2} \bigl( \partial^2_x u_0(n, i-1, t_n, x_i) - \partial^2_t u_0(n+1,i, t_n, x_i)\bigr)\\
&- \bigl( u_1(n, i, t_n, x_i) - u_1(n, i-1, t_n, x_i) \bigr)\\
&- \bigl( u_2(n+1, i, t_n, x_i) - u_2(n, i-1, t_n, x_i) \bigr)
\end{split}
\end{align}
}\\

\noindent \emph{in der Ordnung $O(h^3)$:}
{\small
\begin{align}\label{eq:transport:kleineta:oh3}
\begin{split}
\partial_t u_2(n+1, i, t_n, x_i) + \partial_x u_2(n, i-1, t_n, x_i) +\\
\partial_t u_1(n+1, i, t_n, x_i) + \partial_x u_1(n, i-1, t_n, x_i)
\end{split}
&=
\begin{split}
&\frac{1}{2} \partial^2_x u_0(n, i-1, t_n, x_i) - \partial^2_t u_0(n+1,i, t_n, x_i)\\
&- \frac{1}{6}\bigl(\partial^3_x u_0(n, i-1, t_n, x_i) + \partial^3_t u_0(n+1,i, t_n, x_i) \bigr)\\
&\frac{1}{2} \bigl( \partial^2_x u_1(n, i-1, t_n, x_i) - \partial^2_t u_1(n+1,i, t_n, x_i) \bigr)\\
&- \bigl( u_2(n, i, t_n, x_i) - u_2(n, i-1, t_n, x_i) \bigr)
\end{split}
\end{align}\\
}

\emph{mit den Anfangsbedingungen}
\begin{align}\label{eq:transport:kleineta:anfangsbedinungen}
\begin{split}
w_0(0, x_i) &= U(x_i),\\
w_1(0, x_i) &= 0,\\
w_2(0, x_i) &= 0,
\end{split}&
\begin{split}
z_0(0, x_i) &= V(x_i),\\
z_1(0, x_i) &= 0,\\
z_2(0, x_i) &= 0
\end{split}
\end{align}

\subsection*{Alternierende Gittervorzeichen}

Setzt man wieder den Ansatz mit alternierenden Gittervorzeichen
\[ u_k(n, i, t, x) = w_k(t, x) + (-1)^i \Omega(n) z_k(t, x), \qquad k = 0,1,2 \]
ein, so folgt aus \eqref{eq:transport:kleineta:o1}
\begin{align*}
0 &= u_0(n+1, i, t_n, x_i) - u_0(n, i-1, t_n, x_i)\\
&= w_0(t_n, x_i) + (-1)^i \Omega(n+1) z_0(t_n, x_i) - \bigl( w_k(t_n, x_i) + (-1)^{i-1} \Omega(n) z_k(t_n, x_i) \bigr)\\
&= (-1)^i \Omega(n+1) z_0(t_n, x_i) - (-1)^{i-1} \Omega(n) z_k(t_n, x_i)\\
&= (-1)^i \Omega(n+1) z_0(t_n, x_i) + (-1)^{i} \Omega(n) z_k(t_n, x_i)\\
&= (-1)^i z_0(t_n, x_i) \bigl( \Omega(n+1) + \Omega(n) \bigr),
\end{align*}
und das impliziert ganz analog wie in Satz \ref{satz:omega_n}, dass dieses mal $\Omega(n) = (-1)^n$ gilt.
Folglich ist unser Ansatz durch
\begin{align}
v^n_i = \begin{split}
&w_0(t_n, x_i) + h w_1(t_n, x_i) + h^2 w_2(t_n, x_i)\\
&\qquad + (-1)^{i+n} \bigl( z_0(t_n, x_i) + h z_1(t_n, x_i) + h^2 z_2(t_n, x_i) \bigr)
\end{split}
\end{align}
und diskreten Ansatzfunktionen sind dieses mal durch
\begin{align}\label{eq:ke:ansatz} u_k(n, i, t, x) = w_k(t,x) + (-1)^{i+n} z_k(t, x) \qquad k = 0,1,2 \end{align}
gegeben. Hieraus folgt zunächst für alle $k = 0,1,2$
\begin{align}\label{eq:appendix:kleineta:bedingung}
u_k(n+1, i, t, x) - u_k(n, i-1, t, x) = 0.
\end{align}
Die Gleichung \eqref{eq:transport:kleineta:oh} lautet mit der Bedingung \eqref{eq:appendix:kleineta:bedingung} nun
\begin{align*}
\partial_t u_0(n+1, i, t_n, x_i) + \partial_x u_0(n, i-1, t_n, x_i) = - \bigl( u_0(n, i, t_n, x_i) - u_0(n, i-1, t_n, x_i) \bigr)\\
\end{align*}
Setzt man \eqref{eq:ke:ansatz} für $u_0$ ein, so gilt für die linke Seite ferner
\begin{align}\label{eq:appendix:kleineta:oh:linkeseite}
&\partial_t u_0(n+1, i, t_n, x_i) + \partial_x u_0(n, i-1, t_n, x_i) \nonumber\\
&= \partial_t w_0(t_n,x_i) + (-1)^{i+n+1} \partial_t z_0(t_n,x_i)+ \partial_x w_0(t_n,x_i) + (-1)^{i+n-1} \partial_x z_0(t_n,x_i)\nonumber\\
&= \partial_t w_0(t_n,x_i) + \partial_x w_0(t_n,x_i) - (-1)^{i+n} \bigl( \partial_t z_0(t_n,x_i) + \partial_x z_0(t_n,x_i) \bigr).
\end{align}
und für die rechte Seite erhalten wir
\begin{align}\label{eq:appendix:kleineta:oh:rechteseite}
&- \bigl( u_0(n, i, t_n, x_i) - u_0(n, i-1, t_n, x_i) \bigr)\\
&= -\Bigl( w_0(t_n, x_i) + (-1)^{i+n} z_0(t_n, x_i) - \bigl( w_0(t_n, x_i) + (-1)^{i+n-1} z_0(t_n, x_i) \bigr) \Bigr) \nonumber\\
&= - (-1)^{i+n} 2 z_0(t_n,x_i) 
\end{align}

Genau wie im vorangegangen Kapitel \ref{sec:transport:osz} folgern wir hieraus je eine Bedingung für die Funktionen $w_0$ und $z_0$.
Jedoch können wir nicht das Gleiche wie in Satz \ref{satz:alt:beidenull} machen, da es dort essentiell war, dass wir eine in $n$ unbeschränkte Funktion betrachten.
Allerdings haben wir im Ansatz glatte Abbildungen $w_k, z_k \in \Cinf$ für $k = 0,1,2$ angenommen und wir können hier mit ihrer Regularität argumentieren.
\begin{satz}\label{satz:glgregulaer}
Sei $\omega \in (0, 2\pi)$ und $\Omega\colon \N \to \R$.
Angenommen es gäbe außerdem zwei positiven Zahlen $\kappa, K > 0$ mit $\kappa \leq \abs{\Omega(n)} \leq K$ für alle $n \in \N$.
Es gelte für alle $h > 0$ und alle Gitterpunkte $(t_n, x_k) \in G_h$
\begin{align}\label{satz:glgreg:vor}
\begin{split}
F(t_n, x_k) &= e^{i \omega k} \Omega(n) f(t_n, x_k) \qquad \text{bzw.}\\
F(t_n, x_k) &= \Ree\left[ e^{i \omega k} \Omega(n) f(t_n, x_k) \right]
\end{split}
\end{align}
für zwei glatte Funktionen $F,f \in \Cinf$. Dann gilt für alle $(t,x) \in \Rp \times \R$
\begin{align*}
F(t, x) &= 0 \quad \text{und}\\
f(t, x) &= 0.
\end{align*}
\end{satz}
\begin{proof}
Weil $f$ stetig differenzierbar ist, gilt mit der Voraussetzung \eqref{satz:glgreg:vor}
\begin{align*}
F(t_n, x_k) &= e^{i \omega k} \Omega(n) f(t_n, x_k)\\
&\text{und}\\
F(t_n, x_{k+1}) &= e^{i \omega (k + 1)} \Omega(n) f(t_n, x_{k+1})\\
&= e^{i \omega k} e^{i \omega} \Omega(n) f(t_n, x_{k+1})\\
&= e^{i \omega k} e^{i \omega} \Omega(n) \bigl( f(t_n, x_k) + h \partial_x f(t_n, x_k) + o(h) \bigr)
\end{align*}
also gilt für alle $h > 0$ auch
\begin{align}
\begin{split}\nonumber
\abs{F(t_n, x_k + h) - F(t_n, x_k)} &= \abs{F(t_n, x_{k+1}) - F(t_n, x_k)}\\
&= \abs{e^{i \omega k} (e^{i \omega} - 1) \Omega(n) f(t_n, x_k) + h e^{i \omega k} \Omega(n) \partial_x f(t_n, x_k)}\\
&\geq \abs{ e^{i \omega k} (e^{i \omega} - 1) \Omega(n) f(t_n, x_k) }  - h  \abs{ e^{i \omega k} \Omega(n) \partial_x f(t_n, x_k) }\\
&\geq \kappa \cdot \abs{ e^{i \omega k} (e^{i \omega} - 1) f(t_n, x_k) }  - h  K \cdot \abs{ e^{i \omega k} \partial_x f(t_n, x_k) }\\
&= \kappa \cdot \underbrace{\abs{ e^{i \omega k} }}_{= 1} \abs{ (e^{i \omega} - 1) f(t_n, x_k) } - hK \cdot  \underbrace{\abs{ e^{i \omega k} }}_{= 1} \abs{ \partial_x f(t_n, x_k) }
\end{split}\\ \label{satz:glgregulaer:bed}
&= \kappa \cdot \underbrace{\abs{e^{i \omega} - 1}}_{\neq 0} \abs{f(t_n, x_k)} - hK \cdot \abs{\partial_x f(t_n, x_k)}.
\end{align}
Unser Gitter $G_h$ ist gerade so gemacht, dass es für alle $(t,x) \in \Rp \times \R$ und zu jeder Folge von Gitterweiten $(h^p)_{p \in \N}$ mit $h_p \to 0$ eine Folge von Gitterpunkten $(t^p, x^p) \in G_{h_p}$ und $(t^p, x^p) \to (t,x)$ gibt.
Es gilt $\lim_{h_p \to 0} \abs{F(t^p, x^p + h_p) - F(t^p, x^p)} = 0$ für $h_p \to 0$, weil $F$ stetig ist.
Es gilt weiterhin
\[ \lim_{p \to \infty} \kappa \abs{e^{i \omega} - 1} \abs{f(t^p, x^p)} - h_pK \abs{\partial_x f(t^p, x^p)} = \kappa \abs{e^{i \omega} - 1} \abs{f(t, x)}. \]
Die Voraussetzung $0 < \omega < 2 \pi$ impliziert nun $\abs{e^{i \omega} - 1} \neq 0$ und somit folgt aus der Stetigkeit von $f$ und der Ungleichung \eqref{satz:glgregulaer:bed} $f(t,x) = 0$, was zu zeigen war.
\end{proof}
% In Satz \ref{satz:glgregulaer} ist nun hingegen wichtig, dass $\Omega(n)$ eine Art Sublinear-Growth-Bedingung erfüllt.
% D.\,h. wir brauchen in Gleichung \eqref{satz:glgregulaer:bed}, dass $h \cdot \Omega\left( h^{-1} \right) \to 0$ für $h \to 0$ gilt. 
Setzt man \eqref{eq:appendix:kleineta:oh:linkeseite} und \eqref{eq:appendix:kleineta:oh:rechteseite} nun gleich, so leitet man mithilfe des Satzes \ref{satz:glgregulaer} die beiden Anfangswertprobleme
\begin{align}\label{eq:appendix:kleineta:oh}
\partial_t w_0(t,x) + \partial_x w_0(t,x) &= 0 &w_0(0,x) &= U(x) \quad \text{und}\\
\partial_t z_0(t,x) + \partial_x z_0(t,x) &= 2 z_0(t,x)  &z_0(0,x) &= V(x) 
\end{align}
her. Die Lösungen dieser Anfangswertprobleme haben wir schon im Anhang \ref{appendix:loesungen} bestimmt und sie lauten
\begin{align}
w_0(t,x) = U(x - t) \qquad \text{und} \qquad z_0(t,x) = V(x - t) e^{2 t}.
\end{align}
Insbesondere gelten dann
\begin{align}\label{eq:appendix:kleineta:korollar:h2}
\begin{split}
\partial^2_t w_0(t,x) &= \partial^2_x w_0(t,x)\qquad\\&\text{und}\\
\partial^2_t z_0(t,x) &= \partial_t \bigl( \partial_t z_0(t,x) \bigr)\\
&= \partial_t \bigl( 2z_0(t,x)  - \partial_x z_0(t,x) \bigr)\\
&= 2 \partial_t z_0(t,x) - \partial_t \bigl( \partial_x z_0(t,x) \bigr)\\
&= 2 \partial_t z_0(t,x) - \partial_x \bigl( \partial_t z_0(t,x) \bigr)\\
&= 2 \bigl( 2z_0(t,x) - \partial_x z_0(t,x) \bigr) - \partial_x \bigl( 2 z_0(t,x) - \partial_x z_0(t,x) \bigr)\\
&= 4 z_0(t,x) - 4 \partial_x z_0(t,x) + \partial^2_x z_0(t,x).
\end{split}
\end{align}

Wir widmen uns nun der nächsten Ordnung zu, der Gleichung \eqref{eq:transport:kleineta:oh2} in der Ordnung $O(h^2)$.
Mit der Eigenschaft \eqref{eq:appendix:kleineta:bedingung} lautet sie
{\small
\begin{align}\label{eq:ke:oh2}
\begin{split}
\partial_t u_1(n+1, i, t_n, x_i) + \partial_x u_1(n, i-1, t_n, x_i) +\\
\partial_t u_0(n+1, i, t_n, x_i) + \partial_x u_0(n, i-1, t_n, x_i)
\end{split}
&=
\begin{split}
&\frac{1}{2} \bigl( \partial^2_x u_0(n, i-1, t_n, x_i) - \partial^2_t u_0(n+1,i, t_n, x_i)\bigr)\\
&- \bigl( u_1(n, i, t_n, x_i) - u_1(n, i-1, t_n, x_i) \bigr)\\
\end{split}
\end{align}
}
Und ganz analog wie schon im Fall von $u_0$ gelten die beiden Gleichungen
\begin{align} \label{eq:ke:glg1}
- \bigl( u_1(n, i, t_n, x_i) - u_1(n, i-1, t_n, x_i) \bigr) = - (-1)^{i+n} 2 z_1(t_n,x_i) 
\end{align}
und
\begin{align}
\begin{split}
&\partial_t u_1(n+1, i, t_n, x_i) + \partial_x u_1(n, i-1, t_n, x_i)\\
&=\partial_t w_1(t_n,x_i) + \partial_x w_1(t_n,x_i) - (-1)^{i+n} \bigl( \partial_t z_1(t_n,x_i) + \partial_x z_1(t_n,x_i) \bigr).
\end{split}
\end{align}
Außerdem gilt
\begin{align}
\begin{split}
\partial_t u_0(n+1, i, t_n, x_i) + \partial_x u_0(n, i-1, t_n, x_i)
&= - \bigl( u_0(n, i, t_n, x_i) - u_0(n, i-1, t_n, x_i) \bigr)\\
&= - (-1)^{i+n} 2 z_0(t_n,x_i).
\end{split}
\end{align}
Und aus der kleinen Nebenrechnung in \eqref{eq:appendix:kleineta:korollar:h2} folgern wir 
\begin{align}\label{eq:ke:glg4}
\begin{split}
&\frac{1}{2} \bigl( \partial^2_x u_0(n, i-1, t_n, x_i) - \partial^2_t u_0(n+1,i, t_n, x_i)\bigr)\\
&= \frac{1}{2} \Bigl[ \partial^2_x u_0(n, i-1, t_n, x_i) - \partial^2_t \bigl( u_0(n+1,i, t_n, x_i)\bigr) \Bigr]\\
&= \frac{1}{2} \Bigl[ \partial^2_x w_0(t_n, x_i) + (-i)^{i+n-1} \partial^2_x z_0(t_n, x_i) - \bigl( \underbrace{\partial^2_t w_0(t_n, x_i)}_{= \partial^2_x w_0(t_n, x_i)} + (-i)^{i+n+1} \partial^2_t z_0(t_n, x_i) \bigr) \Bigr]\\
&= \frac{1}{2} \Bigl[ (-i)^{i+n-1} \partial^2_x z_0(t_n, x_i) - (-i)^{i+n+1} \partial^2_t z_0(t_n, x_i) \Bigr]\\
&= (-1)^{i+n} \frac{1}{2} \Bigl[ \underbrace{\partial^2_t z_0(t_n, x_i)}_{= 4 z_0(t_n,x_i) - 4 \partial_x z_0(t_n,x_i) + \partial^2_x z_0(t_n,x_i)} - \partial^2_x z_0(t_n, x_i) \Bigr]\\
&= (-1)^{i+n} 2 \Bigl[ z_0(t_n,x_i) - \partial_x z_0(t_n,x_i) \Bigr]\\
\end{split}
\end{align}
Setzen wir die vier Gleichungen \eqref{eq:ke:glg1} bis \eqref{eq:ke:glg4} zu der Gleichung \eqref{eq:ke:oh2} zusammen, so erhalten wir
\begin{align*}
&\partial_t w_1(t_n,x_i) + \partial_x w_1(t_n,x_i) - (-1)^{i+n} \bigl( \partial_t z_1(t_n,x_i) + \partial_x z_1(t_n,x_i) \bigr)\\
&= (-1)^{i+n} \Bigl[ - 2 z_1(t_n,x_i) + 2 \left( 2 z_0(t_n,x_i) - \partial_x z_0(t_n,x_i) \right) \Bigr] 
\end{align*}
Und wir erhalten mit Satz \ref{satz:glgregulaer} die Anfangswertprobleme 
\begin{align}\label{eq:appendix:kleineta:oh2}
\partial_t w_1(t,x) + \partial_x w_1(t,x) &= 0 &w_1(0,x) &= 0 \quad \text{und}\\
\partial_t z_1(t,x) + \partial_x z_1(t,x) &= 2 z_1(t,x) + 2 \bigl( \partial_x z_0(t,x) - 2z_0(t,x) \bigr)  &z_1(0,x) &= 0
\end{align}
Folglich gilt schon mal $w_1 = 0$ und wir können für $z_0$ wieder eine langsame Zeitvariable, dieses mal $\tau = 2ht$, einführen.
Unsere Ansatzfunktionen $u_k$ lautet nun für $k = 0, 1, 2$
\[ u_k(n, i, t, x) = w_k(t, x) + (-1)^{i+n} \tilde z_k(t_n, 2 h t_n, x_i). \]
Sei $\tilde z(t, \tau, x)$ derart, dass $z(t_n, x_i) = \tilde z(t_n, 2 h t_n, x_i)$ gilt, dann folgt
\[ \partial_t z(t_n, x_i) = \partial_t \tilde z(t_n, \tau_n, x_i) + 2 h \partial_\tau \tilde z(t_n, \tau_n, x_i). \]
Mit diesem Ansatz stellen wir das Anfangswertproblem
\begin{align*}
\partial_t z_0(t,\tau,x) + \partial_x z_0(t,\tau,x) &= 2 z_0(t,\tau,x)\\
\partial_t z_1(t,\tau,x) + \partial_x z_1(t,\tau,x) &= 2 z_1(t,\tau,x) + 2 \bigl( \partial_x z_0(t,\tau,x) - \partial_\tau z_0(t,\tau,x) - 2z_0(t,\tau,x) \bigr)\\
z_0(0,0,x) &= V(x) \quad \forall x \in \R\\
z_1(0,0,x) &= 0  \quad \forall x \in \R
\end{align*}
auf und fordern zusätzlich
\begin{align}\label{eq:ke:lang}
\partial_\tau z_0(t,\tau,x) - \partial_x z_0(t, \tau, x) = -2z_0(t,\tau,x).
\end{align}
Aus der ersten Gleichung $\partial_t z_0(t,\tau,x) + \partial_x z_0(t,\tau,x) = 2 z_0(t,\tau,x)$, mit $z_0(0,0,x) = V(x)$ folgern wir
$z_0(t, \tau, x) = A(\tau, x - t) e^{2t}$ für eine Abbildung $A$ mit $A(0, x) = V(x)$ für alle $x \in \R$.
Mit \eqref{eq:ke:lang} lässt sich A dann für alle $(t, \tau, x) \in \left(\Rp\right)^2 \times \R$ durch
\[ A(\tau, x - t) = A(0, x - t + \tau) e^{- 2\tau} \]
bestimmen und es folgt für alle $(t, \tau, x) \in \Rp \times \R$
\[ z_0(t, \tau, x) = V(x - t + \tau) e^{2 (t - \tau)}, \qquad z_1(t, \tau, x) = 0. \]


Zumindest bis zur Ordnung $O(h^2)$ können wir unseren Ansatz nun schon konkretisieren und er ist bestimmt, durch
\begin{align}\label{eq:ke:loesung}
v^n_i = U(x_i - t_n) + V(x_i - t_n + 2 h t_i) e^{2 (t_i - 2 h t_i)} + O(h^2).
\end{align}

\begin{figure}
\centering
\begin{tikzpicture}
\begin{semilogyaxis}[
    title={$\err^n = \max_{i \in \Z} \abs{\sin\bigl(\pi (x_i - t_n )\bigr) - v^n_i}$},
    xlabel={Zeit $t$},
    ylabel={maximaler Fehler},
    legend entries={$\err^n$,$\eps_M e^{2t}$, $t h^2 \frac{\pi^2}{2}$, $\eps_M e^{2t} + t h^2 \frac{\pi^2}{2}$},
    legend style={at={(1.5,1)}}
]
\addplot[myblue, line width={2}] table {data/max_errors_small_eta_0.001.dat};
\addplot[black, line width={0.5}, dotted] table [y index={2}] {data/max_errors_small_eta_0.001.dat};
\addplot[mygray, line width={1}, dashdotted] table [y index={4}] {data/max_errors_small_eta_0.001.dat};
\addplot[black, line width={0.5}, mark=o, mark repeat={10}] table [y index={3}] {data/max_errors_small_eta_0.001.dat};
\end{semilogyaxis}
\end{tikzpicture}
\caption{Hier vergleichen wir die maximalen Fehler der numerischen Lösungen für die Startwerte $U(x) = sin(\pi x)$, $V(x) = \eps_M$ und $\eta = h = 10^{-3}$ zur echten Lösung $u(t,x) = sin(\pi(x - t))$ mit dem geschätztem Fehler $\eps_M e^{2t}$ aus der asymptotischen Entwicklung in .}
\label{fig:transport:kleineta:max_error}
\end{figure}

Auch dieses Ergebnis vergleichen wir nun mit unserer Referenzlösung in der Abbildung \ref{fig:transport:kleineta:max_error} und man erkennt, dass wir die Steigung des Fehlerwachstums gut eingefangen haben.

\subsection*{Fehler bestimmen: eine Ordnung weiter}

Setzen wir unsere bisherigen Erkenntnisse nun wie in Anhang \ref{appendix:kleineta:rechnung} ein, so erhalten wir in Ordnung $O(h^3)$ schlussendlich die Gleichungen
\begin{align}
\partial_t w_2(t,x) + \partial_x w_2(t,x) &= - \frac{1}{2} \partial^2_t w_0(t,x)\\
\begin{split}
\partial_t z_2(t,\tau,x) + \partial_x z_2(t,\tau,x) &= 2 z_2(t,\tau,x) - \Bigl( 2 \bigl( \partial_\tau z_0(t,\tau,x) + \partial^2_\tau z_0(t,\tau,x) \bigr)\\
&\quad + \partial^2_t z_0(t,\tau,x) - \frac{1}{2} \partial^2_x z_0(t,\tau,x) + \frac{1}{6} \bigl(\partial^3_t z_0(t,\tau,x) + \partial^3_x z_0(t,\tau,x)\bigr) \Bigr)
\end{split}
\end{align}

Damit halten wir schonmal die Lösung $w_2(t,x) = -\frac{t}{2} \partial^2_t w_0(t,x)$ fest und führen für $z_2$ ein paar Nebenrechnungen durch.
Es gilt
\begin{align}
\begin{split}
\partial^3_t z_0 &= \partial_t \bigl( \partial^2_t z_0 \bigr)\\
&= \partial_t \bigl( 4 z_0 - 4 \partial_x z_0 + \partial^2_x z_0 \bigr)\\
&= 4 ( 2z_0 - \partial_x z_0 ) - 4 \partial_x \bigl( 2z_0 - \partial_x z_0 \bigr) + \partial^2_x \bigl( 2 z_0 - \partial_x z_0 \bigr)\\
&= 8z_0 - 4 \partial_x z_0 - 8 \partial_x z_0 + 4 \partial^2_x z_0 + 2 \partial^2_x z_0 - \partial^3_x z_0\\
&= 8 z_0 - 12 \partial_x z_0 + 6 \partial^2_x z_0 - \partial^3_x z_0
\end{split}
\end{align}
also
\begin{align}
\frac{1}{6} \bigl(\partial^3_t z_0 + \partial^3_x z_0 \bigr)
= \frac{4}{3} z_0 - 2 \partial_x z_0 + \partial^2_x z_0.
\end{align}
Außerdem gilt
\[ \partial_\tau z_0 = - \partial_t z_0 \]
und das impliziert
\begin{align*}
2 \bigl( \partial_\tau z_0 + \partial^2_\tau z_0 \bigr) + \partial^2_t z_0
&= - 2 \partial_t z_0 + 2 \partial^2_t z_0 + \partial^2_t z_0\\
&= - 2 \partial_t z_0 + 3 \partial^2_t z_0 \\
&= - 2 (2 z_0 - \partial_x z_0) + 3 (4 z_0 - 4 \partial_x z_0 + \partial^2_x z_0)\\
&= - 4 z_0 - 2 \partial_x z_0 + 12 z_0 - 12 \partial_x z_0 + 3 \partial^2_x z_0\\
&= 8 z_0 - 14 \partial_x z_0 + 3 \partial^2_x z_0.
\end{align*}
In Summe ergeben diese Rechnungen
\begin{align}
\begin{split}
&2 \bigl( \partial_\tau z_0 + \partial^2_\tau z_0 \bigr) + \partial^2_t z_0 - \frac{1}{2} \partial^2_x z_0 + \frac{1}{6} \bigl(\partial^3_t z_0 + \partial^3_x z_0 \bigr)\\
&= 8 z_0 - 14 \partial_x z_0 + 3 \partial^2_x z_0 - \frac{1}{2} \partial^2_x z_0 + \frac{4}{3} z_0 - 2 \partial_x z_0 + \partial^2_x z_0\\
&= \frac{28}{3} z_0 - 16 \partial_x z_0 + \frac{7}{2} \partial^2_x z_0
\end{split}
\end{align}
Man kann das gleiche Spiel wie in der letzten Ordnung, $O(h^2)$, wiederholen und eine langsame weitere Zeitvariable $\tau_2 = h^2 t$ einführen.
Fordert man nun, dass 
\begin{align}
\nonumber
z_0(0,0,0,x) &= V(x)\\
\nonumber
\partial_t z_0(t,\tau, \tau_2, x) + \partial_x z_0(t,\tau,\tau_2,x) &= 2 z_0(t,\tau,\tau_2,x)\\
\nonumber
\partial_\tau z_0(t,\tau, \tau_2, x) - \partial_x z_0(t,\tau, \tau_2, x) &= -2z_0(t,\tau, \tau_2, x)\\
\label{eq:ke:addiff}
\partial_{\tau_2} z_0(t,\tau, \tau_2, x) + 16 \partial_x z_0(t,\tau, \tau_2, x) &= \frac{28}{3} z_0(t,\tau, \tau_2, x) + \frac{7}{2} \partial^2_x z_0(t,\tau, \tau_2, x)
\end{align}
gilt, so folgt auch $z_2 = 0$. Gleichung \eqref{eq:ke:addiff} ist eine Advektion-Diffusion Gleichung.
Zusammen kriegt man nun das Ergebnis

\begin{align*}
v^n_i = U(x_n-t_n) + z_0(t_n, 2 h t_n, h^2 t_n, x_i) -\frac{th^2}{2} \partial^2_t U(t,x) + O(h^3).
\end{align*}