%!TEX root=thesis.tex

\section{Die lineare Transportgleichung}
\subsection{Analytische Lösungen}
\label{appendix:loesungen}
Im Laufe der asymptotischen Entwicklung von \eqref{eq:adv:scheme} stellt sich heraus, dass die Ansatzfunktionen häufig selbst Lösungen der Transportgleichung \eqref{eq:adv:pde} oder Variationen von dieser sind.
Deshalb ist es nützlich und sinnvoll, die richtige Lösung solcher Gleichungen auch analytisch zu untersuchen.
Wir betrachten hier bekannte Gleichungen, deren Lösungen $u$ sogar exakt zu bestimmen sind.
Bei der Lösungskonstruktion nutzen wir die Methode der Charakteristiken.
Dabei bestimmen wir Kurven $\varphi\colon \Rp \to \Rp \times \R$ durch die Zeit-Raum Ebene, welche uns Gleichungen liefern, mit denen zunächst $u \circ \varphi$ bestimmt werden kann und schließen dann auf Lösungen von $u$.
Im Fall unserer linearen partiellen Differentialgleichungen werden diese Kurven $\varphi$ immer Geraden sein.
Zunächst betrachten wir

\subsection*{Die homogene Transportgleichung}
\begin{align}\label{eq:transport:analytisch:homogen}
\partial_t u(t,x) + \partial_x u(t,x) = 0, \quad u(0, x) = U(x).
\end{align}
Wenn $u$ eine Lösung von \eqref{eq:transport:analytisch:homogen} ist, dann ist $u$ konstant auf Geraden der Steigung 1.
Betrachte nämlich die Zeit-Raum Kurve 
\begin{align}\label{eq:transport:analytisch:phi}
\varphi_t(\tau) = (\tau, x+\tau -t).
\end{align}
Dann gilt für alle $t,\tau \in \Rp:$ $D \varphi_t(\tau) = (1, 1)$ und folglich auch
\begin{align}
D u\bigl(\varphi_t(\tau)\bigr) = \partial_t u(\tau, x + \tau - t) + \partial_x u(\tau, x + \tau - t) = 0.
\end{align}
Insbesondere impliziert das wegen des Hauptsatzes für die Integral und Differentialrechnung für alle $t \in \Rp$
\begin{align}
u(t,x) - u(0, x-t) = \int_0^t D u(\tau, x + \tau - t) \dd \tau = \int_0^t D u\bigl( \varphi_t(\tau) \bigr) \dd \tau = 0
\end{align}
und wegen der Anfangsbedingung $u(0,x) = U(x-t)$ gilt also auch für alle $(t, x) \in \Rp \times \R$
\begin{align}\label{eq:transport:analytisch:homogen:loesung}u(t,x) = U(x-t).\end{align}
Dies zeigt die Eindeutigkeit von Lösungen von \eqref{eq:transport:analytisch:homogen}.
Andererseits ist jedes so definierte $u(t,x) = U(x-t)$ auch eine Lösung von \eqref{eq:transport:analytisch:homogen}, solange das für $U$, im womöglich schwachem Sinne, möglich ist.
Denn es gilt
\begin{align}
\partial_t U(x-t) + \partial_x U(x-t) = DU(x-t) \cdot (-1) + DU(x-t) \cdot 1 = 0.
\end{align}

\subsection*{Die inhomogene Transportgleichung}
Sei $F \in L^1\left( \Rp \times \R \right)$. Wir betrachten nun die lineare Transportgleichung mit nichtverschwindender rechten Seite
\begin{align}\label{eq:transport:analytisch:inhomogen}
\partial_t u(t,x) + \partial_x u(t,x) = F(t,x), \quad u(0, x) = U(x).
\end{align}
Dann gilt für Lösungen $u$ von \eqref{eq:transport:analytisch:inhomogen} ganz analog wie im homogenen Fall mit $\varphi$ wie in \eqref{eq:transport:analytisch:phi}
\begin{align}
D u\bigl(\varphi_t(\tau)\bigr) = F(\tau,x + \tau - t).
\end{align}
Demnach folgt wieder mit dem Hauptsatz
\begin{align} \label{eq:transport:analytisch:inhomogen:loesung:allgemein}
\begin{split}
u( t, x ) - u( 0, x -t ) = \int_0^t F(\tau, x + \tau - t) \dd \tau.
\end{split}
\end{align}
Wir werden oft den Spezialfall sehen, dass $F$ selbst Lösung von \eqref{eq:transport:analytisch:homogen} ist.
Es gilt dann nämlich $F(\tau, x + \tau - t) = F(0, x-t)$ für alle $\tau,t \in \Rp$ und somit folgt für alle $t \in \Rp$
\begin{align} 
u( t, x ) - u( 0, x -t ) = \int_0^t F(0, x - t) \dd \tau = t F(0,x-t),
\end{align}
also auch für alle $(t, x) \in \Rp \times \R$
\begin{align}\label{eq:transport:analytisch:inhomogen:loesung:spezial}u(t,x) = U(x - t) + t F(0, x-t).\end{align}

\subsection*{Eine homogene Exponentialgleichung}

Sei $a \in \R$ eine reelle Zahl. Wir betrachten nun die Differentialgleichung
\begin{align}\label{eq:transport:analytisch:exp:homogen}
\partial_t u(t,x) + \partial_x u(t,x) = a u(t,x), \quad u(0, x) = U(x).
\end{align}
Sei wieder $\varphi$ aus \eqref{eq:transport:analytisch:phi} gegeben. Dann gilt dieses mal
\begin{align}\label{eq:transport:analytisch:exp:charode}
D u\bigl(\varphi_t(\tau)\bigr) = a u(\varphi_t(\tau)), \qquad \text{mit $u\bigl(\varphi_t(0) \bigr) = U(x-t)$.}
\end{align}
D.\,h. $u$ erfüllt entlang der Kurve $\varphi_t$ die Differentialgleichung \eqref{eq:transport:analytisch:exp:charode}.
Die Lösung dieser Differentialgleichung lautet bekanntlich $u\bigl(\varphi_t(\tau) \bigr) = U(x-t) e^{a \tau}$ und es folgt weiterhin
\begin{align}
\begin{split}
u(t,x) - u(0,x-t) &= \int_0^t D u(\tau, x + \tau - t) \dd \tau\\
&= \int_0^t D u\bigl(\varphi_t(\tau) \bigr) \dd \tau\\
&= \int_0^t a u(\varphi_t(\tau)) \dd \tau\\
&= \int_0^t a U(x-t) e^{a \tau} \dd \tau\\
&= U(x-t) e^{a t} - U(x-t).
\end{split}
\end{align}
Wegen $u(0,x-t) = U(x-t)$ impliziert das hiermit für alle $(t, x) \in \Rp \times \R$
\begin{align}\label{eq:transport:analytisch:exp:loesung}u(t,x) = U(x-t) e^{at}.\end{align}

Betrachte man f"ur $a$ hingegen eine reelle Abbildung $a \in L^1(\R)$, so folgt f"ur die partielle Differentialgleichung
\begin{align}\label{eq:transport:analytisch:exp:homogen:av}
\partial_t u(t,x) + \partial_x u(t,x) = a(x) u(t,x), \quad u(0, x) = U(x).
\end{align}
ein ganz "ahnliches Ergebnis. Denn die gew"ohnliche Differentialgleichung
\begin{align}\label{eq:transport:analytisch:exp:charode:av}
D u\bigl(\varphi_t(\tau)\bigr) = a(x-t+ \tau) u(\varphi_t(\tau)), \qquad \text{mit $u\bigl(\varphi_t(0) \bigr) = U(x-t)$.}
\end{align}
l"asst sich mit der Methode der Trennung der Variablen l"osen. Die L"osung lautet $u(\varphi_t(\tau)) = U(x-t) e^{\int^\tau_0 a(x-t+s) ds}$ und somit folgt ganz analog
\begin{align}
u(t,x) = U(x-t) e^{\int_0^t a(x-t+s) ds}
\end{align}

\subsection*{Eine inhomogene Exponentialgleichung}

Sei $a \in \R$ eine reelle Zahl und $F \in L^1\left( \Rp \times \R \right)$.
Betrachte nun das Anfangswertproblem
\begin{align}\label{eq:transport:analytisch:exp:inhomogen}
\partial_t u(t,x) + \partial_x u(t,x) = a u(t,x) + F(t,x), \quad \text{für $u(0,x) = U(x)$.}
\end{align}
Diesesmal kriegen wir für $u \circ \varphi_t$ die Differentialgleichung
\begin{align}\label{eq:transport:analztisch:exp:inhomogen:charode}
Du\bigl( \varphi_t(\tau) \bigr) = a u\bigl( \varphi_t(\tau) \bigr) + F\bigl( \varphi_t(\tau) \bigr).
\end{align}
Dies ist eine gewöhnliche lineare Differentialgleichung, deren homogene Lösung wir schon in \eqref{eq:transport:analytisch:exp:charode} bestimmt haben.
Nun wenden wir die Methode der Variation der Konstanten an, um eine Lösung von \eqref{eq:transport:analztisch:exp:inhomogen:charode} zu bestimmen.
Wir machen den Ansatz 
\begin{align}
u\bigl(\phi_t(\tau)\bigr) = A(\tau) e^{a \tau}
\end{align}
und gesucht ist eine Koeffizientenabbildung $A$, so dass für alle $\tau \in \Rp$
\begin{align}
D \left[ A(\tau) e^{a\tau} \right] = a u\bigl( \varphi_t(\tau) \bigr) + F\bigl( \varphi_t(\tau) \bigr) \quad \text{mit $A(0) = U(x-t)$}
\end{align}
gilt. Mit der Produktregel folgt
\begin{align}
\begin{split}
D \left[ A(\tau) e^{a\tau} \right] &= A(\tau) \cdot a e^{a \tau} + e^{a \tau} DA(\tau)\\
&= a u\bigl( \varphi_t(\tau) \bigr) + e^{a \tau} DA(\tau)
\end{split}
\end{align}
und somit muss für alle $\tau \in \Rp$
\begin{align}
e^{a \tau} DA(\tau) = F\bigl( \varphi_t(\tau) \bigr)
\end{align}
bzw. nach dem Anwenden des Hauptsatzes der Integral- und Differentialrechnung
\begin{align}
\begin{split}
A(t) - A(0) &= \int_0^t e^{-a \tau} F\bigl( \varphi_t(\tau) \bigr) \dd \tau\\
&= \int_0^t e^{-a \tau} F\bigl( \tau, x + \tau - t \bigr) \dd \tau
\end{split}
\end{align}
gelten. Für den speziellen Fall, dass $F$ selbst Lösung der homogenen partiellen Differentialgleichung \eqref{eq:transport:analytisch:exp:homogen} mit $F(0,x) = V(x)$ ist, folgt $F(t,x) = e^{at} V(x - t)$ und somit gilt
\begin{align}
A(\tau) &= A(0) +\int_0^\tau e^{-a s} F\bigl( \varphi_t(s) \bigr) \dd s\\
&= U(x-t) +\int_0^\tau e^{-a s} e^{as} V(x - t) \dd s\\
&= U(x-t) + \tau V(x-t) \qquad \text{für alle $\tau, t \in \Rp.$}
\end{align}
Insgesamt erhält man ganz analog wie bisher in diesem Spezialfall die folgende Lösung für alle $(t,x) \in \Rp \times \R$:
\begin{align}\label{appendix:expinhomloes}
u(t,x)= A(t) e^{a t} = \bigl( U(x-t) + t V(x-t) \bigr) e^{a t}.
\end{align}

\subsection{Rechnungen für den regulären Ansatz}
\label{appendix:regulaer:rechnungen}
\subsection*{Ohne langsame Zeitskala}
Sei der Ansatz aus \eqref{eq:transport:regulaer:ansatz}
\begin{align}
v^n_i = u_0(t_n, x_i) + h u_1(t_n, x_i) + o(h)
\end{align}
für glatte Abbildungen $u_0, u_1 \in C^{\infty}\left( \Rp \times \R \right)$ gegeben.
Wir rechnen nun die beiden Summanden aus \eqref{eq:adv:scheme} aus und benutzen dabei, dass $t_n = n h \lambda$ und $x_i = i h$ gilt.
Das impliziert nämlich
\begin{align}
\begin{split}
v^{n+1}_i - v^n_i &= \bigl( u_0(t_{n+1}, x_i) + h u_1(t_{n+1}, x_i)\bigr) - \bigl( u_0(t_n, x_i) + h u_1(t_n, x_i) \bigr)\\
&= \bigl( u_0(t_{n+1}, x_i) - u_0(t_n, x_i) \bigr) + h \cdot \bigl( u_1(t_{n+1}, x_i) - u_1(t_n, x_i)\bigr)\\
&= \bigl( u_0(t_n + h \lambda, x_i) - u_0(t_n, x_i) \bigr) + h \cdot \bigl( u_1(t_n + h \lambda, x_i) - u_1(t_n, x_i)\bigr)\\
&= \left( h \cdot \lambda \partial_t u_0(t_n, x_i) + \frac{(h \lambda)^2}{2} \partial^2_t u_0(t_n, x_i) + o(h^2) \right)\\
&\quad + h \cdot \Bigr( h \lambda \partial_t u_1(t_n, x_i) + o(h) \Bigr)\\
&= h \cdot \lambda \partial_t u_0(t_n, x_i) + h^2 \cdot \left( \frac{\lambda^2}{2} \partial^2_t u_0(t_n, x_i) + \lambda \partial_t u_1(t_n, x_i) \right) + o(h^2)
\end{split}
\end{align}
und
\begin{align} \label{appendix:regulaer:rechnungen:diff2}
\begin{split}
\lambda \bigl( v^n_i - v^n_{i-1} \bigr)
&= \lambda \Bigl[ \bigl( u_0(t_n, x_i) + h u_1(t_n, x_i)\bigr) - \bigl( u_0(t_n, x_{i-1}) + h u_1(t_n, x_{i-1}) \bigr) \Bigr]\\
&= \lambda \Bigl[ \bigl( u_0(t_n, x_i) - u_0(t_n, x_{i-1})\bigr) + h \cdot \bigl( u_1(t_n, x_i) - h u_1(t_n, x_{i-1}) \bigr) \Bigr]\\
&= \lambda \Bigl[ \bigl( u_0(t_n, x_i) - u_0(t_n, x_i - h)\bigr) + h \cdot \bigl( u_1(t_n, x_i) - h u_1(t_n, x_i - h) \bigr) \Bigr]\\
&= \lambda \Bigl[ h \cdot \partial_x u_0(t_n, x_i) - \frac{h^2}{2} \partial^2_x u_0(t_n,x_i) + o(h^2)\\
&\qquad + h \cdot \bigl( h \cdot \partial_x u_1(t_n, x_i) + o(h) \bigr) \Bigr]\\
&= h \cdot \lambda \partial_x u_0(t_n, x_i) + h^2 \cdot \left( - \frac{\lambda}{2} \partial^2_x u_0(t_n,x_i) + \lambda \partial_x u_1(t_n, x_i) \right) + o(h^2).
\end{split}
\end{align}
Sortiert man diese Ergebnisse nun nach den Ordnungen von $h$ so erhält man die beiden Gleichungen für alle Gitterpunkte $(t_n, x_i) \in G_h$
\begin{align*}
\lambda \cdot \bigl( \partial_t u_0(t_n, x_i) + \partial_x u_0(t_n, x_i) \bigr) &= 0\\
\lambda \cdot \bigl( \partial_t u_1(t_n, x_i) + \partial_x u_1(t_n, x_i) \bigr) &= \lambda \left( \frac{1}{2} \partial^2_x u_0(t_n, x_i) - \frac{\lambda}{2} \partial^2_t u_0(t_n, x_i) \right)
\end{align*}
Diese Gleichungen gelten nun für alle Gitterweiten $h > 0$.
Teilt man die Gleichungen durch $\lambda > 0$ und verwendet die stetige Differenzierbarkeit der Ansatzfunktionen, so erhält man hieraus für alle $(t,x) \in \Rp \times \R$
\begin{align}\label{eq:appendix:regulaer:oh}
\partial_t u_0(t, x) + \partial_x u_0(t, x) &= 0\\
\partial_t u_1(t, x) + \partial_x u_1(t, x) &= \frac{1}{2} \partial^2_x u_0(t, x) - \frac{\lambda}{2} \partial^2_t u_0(t, x) \nonumber\\
&= \frac{1-\lambda}{2} \partial^2_x u_0(t, x) \label{eq:appendix:regulaer:oh2}
\end{align}
wobei wir in \eqref{eq:appendix:regulaer:oh2} schon die Gleichung \eqref{eq:appendix:regulaer:oh} für $u_0$ benutzt haben, denn es gilt
\begin{align*}
\partial^2_t u_0(t,x) &= \partial_t \, \bigl( \partial_t u_0(t,x) \bigr)\\
&= \partial_t \, \bigl( - \partial_x u_0(t,x) \bigr)\\
&= - \partial_x \, \bigl( \partial_t u_0(t,x) \bigr)\\
&= - \partial_x \, \bigl( - \partial_x u_0(t,x) \bigr)\\
&= \partial^2_x u_0(t,x).
\end{align*}
\subsection*{Mit langsamer Zeitskala}
Wir betrachten nun den Ansatz \eqref{eq:transport:regulaer:tau:ansatz}
\begin{align}
v^n_i = u_0(t, \tau, x) + h u_1(t, \tau, x) + o(h)
\end{align}
mit $\tau = h t.$
Die beiden Summanden aus dem Upwind-Verfahren \eqref{eq:adv:scheme} ergeben nun
{\small \begin{align} \label{eq:appendix:regulaer:tau:diff1}
\begin{split}
v^{n+1}_i - v^n_i &= \bigl( u_0(t_{n+1}, \tau_{n+1}, x_i) + h u_1(t_{n+1}, \tau_{n+1}, x_i)\bigr) - \bigl( u_0(t_n, \tau_n, x_i) + h u_1(t_n, \tau_n, x_i) \bigr)\\
&= \bigl( u_0(t_{n+1}, \tau_{n+1}, x_i) - u_0(t_n, \tau_n, x_i) \bigr) + h \cdot \bigl( u_1(t_{n+1}, \tau_{n+1}, x_i) - u_1(t_n, \tau_n, x_i)\bigr)\\
&= \bigl( u_0(t_n + h \lambda, \tau_n + h^2 \lambda, x_i) - u_0(t_n, \tau_n, x_i) \bigr)\\
&\quad + h \cdot \bigl( u_1(t_n + h \lambda, \tau_n + h^2 \lambda, x_i) - u_1(t_n, \tau_n, x_i)\bigr)\\
&= \left( h \cdot \lambda \partial_t u_0(t_n, \tau_n, x_i) + \frac{(h \lambda)^2}{2} \partial^2_t u_0(t_n, \tau_n, x_i) + h^2 \lambda \partial_\tau u_0(t_n, \tau_n, x_i) + o(h^2) \right)\\
&\quad + h \cdot \Bigr( h \lambda \partial_t u_1(t_n, \tau_n, x_i) + o(h) \Bigr)\\
&= h \cdot \lambda \partial_t u_0(t_n, \tau_n, x_i)\\
&\quad + h^2 \cdot \left( \frac{\lambda^2}{2} \partial^2_t u_0(t_n, \tau_n, x_i) + \lambda \partial_\tau u_0(t_n, \tau_n, x_i) + \lambda \partial_t u_1(t_n, \tau_n, x_i) \right) + o(h^2)
\end{split}
\end{align}}
und, weil $\tau_n$ nicht von $i$ abhängt, folgt ganz analog wie in \eqref{appendix:regulaer:rechnungen:diff2} für die zweite Differenz
{\small \begin{align} \label{eq:appendix:regulaer:tau:diff2}
\begin{split}
\lambda \bigl( v^n_i - v^n_{i-1} \bigr)
&= \lambda \Bigl[ \bigl( u_0(t_n, \tau_n, x_i) + h u_1(t_n, \tau_n,  x_i)\bigr) - \bigl( u_0(t_n, \tau_n, x_{i-1}) + h u_1(t_n, \tau_n, x_{i-1}) \bigr) \Bigr]\\
&= h \cdot \lambda \partial_x u_0(t_n,\tau_n,  x_i) + h^2 \cdot \left( - \frac{\lambda}{2} \partial^2_x u_0(t_n,\tau_n, x_i) + \lambda \partial_x u_1(t_n, \tau_n, x_i) \right) + o(h^2).
\end{split}
\end{align} }
Addiert man beide Summanden \eqref{eq:appendix:regulaer:tau:diff1} und \eqref{eq:appendix:regulaer:tau:diff2}, teilt durch $\lambda$, sortiert nach den Ordnungen von $h$ und benutzt wieder die stetige Differenzierbarkeit der Ansatzfunktionen $u_0$ und $u_1$, so erhält man für alle $(t, \tau, x)$ das Gleichungssystem
\begin{align}\label{eq:appendix:regulaer:tau:oh}
\partial_t u_0(t, \tau, x) + \partial_x u_0(t, \tau, x) &= 0\\
\partial_t u_1(t, \tau, x) + \partial_x u_1(t, \tau, x) &= \frac{1-\lambda}{2} \partial^2_x u_0(t, \tau, x) - \partial_\tau u_0(t, \tau, x) \label{eq:appendix:regulaer:tau:oh2}
\end{align}

\subsection{Rechnungen für diskrete Ansätze}
\label{appendix:diskret:summanden}
In diesem Kapitel betrachten wir den allgemeinen Ansatz \eqref{eq:transport:diskret:ansatz} mit der diskreten Skala $(n,i)$
\begin{align*}
v^n_i = u_0(n, i, t_n, x_i) + h u_1(n, i, t_n, x_i) + h^2 u_2(n, i, t_n, x_i) + o(h^2),
\end{align*}
wobei $u_k(n, i, \pkt, \pkt) \in C^\infty\left( \Rp \times \R \right)$ für alle $n \in \N$, $i \in \Z$ und $k=0,1,2$ gilt.
Wir rechnen nun die einzelnen Summanden von \eqref{eq:adv:scheme_rechnung} aus und benutzen dabei ganz wesentlich folgende Gleichungen für alle $k = 0,1,2$:
\begin{align}\label{apx:diskret:lemma1}
\begin{split}
u_k(n+1, i, t_{n+1}, x_i)\\
- u_k(n, i, t_n, x_i)
&= u_k(n+1, i, t_n + \lambda h, x_i) - u_k(n, i, t_n, x_i)\\
&= u_k(n+1, i, t_n, x_i) - u_k(n, i, t_n, x_i) + \lambda h \partial_t u_k(n+1, i, t_n, x_i)\\
&\quad + \frac{(\lambda h)^2}{2} \partial^2_t u_k(n+1, i, t_n, x_i) + \frac{(\lambda h)^3}{6} \partial^3_t u_k(n+1, i, t_n, x_i)
\end{split}
\end{align}
und
\begin{align}
\begin{split}\label{apx:diskret:lemma2}
u_k(n, i, t_n, x_i)\qquad \qquad&\\
- u_k(n, i-1, t_n, x_{i-1})
&= u_k(n, i, t_n, x_i) - u_k(n, i-1, t_n, x_i - h)\\
&= u_k(n, i, t_n, x_i) - u_k(n, i-1, t_n, x_i) + h \partial_x u_k(n, i-1, t_n, x_i)\\
&\quad - \frac{h^2}{2} \partial^2_x u_k(n, i-1, t_n, x_i) + \frac{h^3}{6} \partial^3_x u_k(n, i-1, t_n, x_i) + o(h^3).
\end{split}
\end{align}
Somit folgt mit \eqref{apx:diskret:lemma1} zunächst für die Differenz in $n$
{\begin{align*}
v^{n+1}_i - v^{n}_i
&= \sum_{k=0}^2 h^k \bigl( u_k(n+1, i, t_{n+1}, x_i) - u_k(n, i, t_n, x_i) \bigr)\\
&= \sum_{k=0}^2 h^k \Biggl( u_k(n+1, i, t_n, x_i) - u_k(n, i, t_n, x_i) + \lambda h \partial_t u_k(n+1, i, t_n, x_i)\\
&\qquad \qquad + \frac{(\lambda h)^2}{2} \partial^2_t u_k(n+1, i, t_n, x_i) + \frac{(\lambda h)^3}{6} \partial^3_t u_k(n+1, i, t_n, x_i) + o(h^3) \Biggr).
\end{align*}}
und sortiert man das nach den Ordnungen von $h$, so erhält man weiterhin
{\footnotesize\begin{align}\label{eq:transport:diskret:diff1}
\begin{split}
v^{n+1}_i - v^{n}_i
&= u_0(n+1, i, t_n, x_i) - u_0(n, i, t_n, x_i)\\
&+ h \cdot \bigl( \lambda \partial_t u_0(n+1, i, t_n, x_i) + u_1(n+1, i, t_n, x_i) - u_1(n, i, t_n, x_i) \bigr)\\
&+ h^2 \cdot \Biggl( \frac{\lambda^2}{2} \partial^2_t u_0(n+1, i, t_n, x_i) + \lambda \partial_t u_1(n+1, i, t_n, x_i) + u_2(n+1, i, t_n, x_i) - u_2(n, i, t_n, x_i) \Biggr)\\
&+ h^3 \cdot \Biggl( \frac{\lambda^3}{6} \partial^3_t u_0(n+1, i, t_n, x_i) \frac{\lambda^2}{2} \partial^2_t u_1(n+1, i, t_n, x_i) + \lambda \partial_t u_2(n+1, i, t_n, x_i)\Biggr)\\
&+ o(h^3).
\end{split}
\end{align}}
Mit \eqref{apx:diskret:lemma2} folgt ganz analog für die Differenz in $i$
{\footnotesize\begin{align}\nonumber
\begin{split}
\lambda \bigl( v^n_i - v^n_{i-1} \bigr)
&= \lambda \cdot \Biggl[ \sum_{k=0}^2 h^k \bigl( u_k(n, i, t_n, x_i) - u_k(n, i-1, t_n, x_{i-1}) \bigr) \Biggr]\\
&= \lambda \cdot \Biggl[ \sum_{k=0}^2 h^k \Bigl( u_k(n, i, t_n, x_i) - u_k(n, i-1, t_n, x_i) + h \partial_x u_k(n, i-1, t_n, x_i)\\
&\qquad \qquad - \frac{h^2}{2} \partial^2_x u_k(n, i-1, t_n, x_i) + \frac{h^3}{6} \partial^3_x u_k(n, i-1, t_n, x_i) + o(h^3) \Bigr) \Biggr]
\end{split}\\
\label{eq:transport:diskret:diff2}
\begin{split}
&= \lambda \cdot \bigl( u_0(n, i, t_n, x_i) - u_0(n, i-1, t_n, x_i) \bigr)\\
&+ h \cdot \Bigl(\lambda \partial_x u_0(n, i-1, t_n, x_i) + \lambda \bigl( u_1(n, i, t_n, x_i) - u_1(n, i-1, t_n, x_i)\bigr)\Bigr)\\
&+ h^2 \cdot \left( - \frac{\lambda}{2} \partial^2_x u_0(n, i-1, t_n, x_i) + \lambda \partial_x u_1(n, i-1, t_n, x_i) + \lambda \bigl( u_2(n, i, t_n, x_i) - u_2(n, i-1, t_n, x_i)\bigr)\right)\\
&+ h^3 \cdot \left( \frac{h^3}{6} \partial^3_x u_k(n, i-1, t_n, x_i) - \frac{\lambda}{2} \partial^2_x u_1(n, i-1, t_n, x_i) + \lambda \partial_x u_2(n, i-1, t_n, x_i) \right)\\
&+ o(h^3).
\end{split}
\end{align}}
% \subsection{Rechnungen für alternierende Vorzeichen}
% \label{appendix:osz:sortiere_nach_frequenz}
% Wir arbeiten nun mit dem im Kapitel \ref{sec:transport:osz} hergeleitetem Ansatz \eqref{eq:transport:osz:ansatz}. 
% D.\,h. unsere diskreten Funktionen $u_i$ sind durch
% \[ u_k(n, i, t, x) = w_k(t, x) + (-1)^{i+n} (1 + 2\eta)^n z_k(t, x) \quad \text{für $k = 0,1$} \]
% gegeben, wobei $w_k, z_k \in \Cinf$ für $k=0, 1$ gilt.
% Dieser diskrete Ansatz ist gerade so bestimmt, dass für beide $k=0$ und $k=1$ die Gleichung
% \begin{align}\label{eq:apx:alt:prop1}
% u_k(n+1, i, t_n, x_i) - u_k(n, i, t_n, x_i) + (1 + 2\eta)\bigl(u_k(n, i, t_n, x_i) - u_k(n, i-1, t_n, x_i) \bigr) = 0
% \end{align}
% für alle $n \in \N$ und $i \in \Z$ erfüllt ist.
% Dies setzen wir in die Gleichungen \eqref{eq:transport:diskret:oh} und \eqref{eq:transport:diskret:oh2} ein und bestimmen daraus Bedingungen für die glatten Funktionen $w_k$ und $z_k$. 
% In der Ordnung $O(h)$ gilt nach den Rechnungen zum diskreten Ansatz die Gleichung \eqref{eq:transport:diskret:oh}, die da lautet:
% \begin{align*}
% (1 + 2\eta) \bigl( \partial_t u_0(n+1, i, t_n, x_i) + \partial_x u_0(n, i-1, t_n, x_i) \bigr) =
% \begin{split}
% &- \bigl( u_1(n+1, i, t_n, x_i) - u_1(n, i, t_n, x_i) \bigr)\\
% &- (1 + 2 \eta) \bigl(u_1(n, i, t_n, x_i) - u_1(n, i-1, t_n, x_i) \bigr).
% \end{split}
% \end{align*}
% Die Rechte Seite dieser Gleichung verschwindet wegen der Eigenschaft \eqref{eq:apx:alt:prop1} und es folgt demnach für alle $n \in \N$, $i \in \Z$ und $h > 0$
% \begin{align*}
% \partial_t u_0(n+1, i, t_n, x_i) + \partial_x u_0(n, i-1, t_n, x_i) = 0.
% \end{align*}
% Weil sowohl $w_0$ als auch $z_0$ stetig differenzierbar sind, folgt für $n \in \N$ und $i \in \Z$
% \begin{align}
% \begin{split}
% \partial_t u_0(n+1, i, t_n, x_i) &= \partial_t \bigl( w_0(t_n, x_i) + (-1)^{i+n+1} (1 + 2\eta)^{n+1} z_0(t_n, x_i) \bigr)\\
% &= \partial_t w_0(t_n, x_i) + (-1)^{i+n} (1 + 2\eta)^n \bigl( - (1 + 2 \eta) \partial_t z_0(t_n, x_i) \bigr)
% \end{split}
% \end{align}
% und
% \begin{align}
% \begin{split}
% \partial_x u_0(n, i-1, t_n, x_i) &= \partial_x \bigl( w_0(t_n, x_i) + (-1)^{i+n-1} (1 + 2\eta)^n z_0(t_n, x_i) \bigr)\\
% &= \partial_x w_0(t_n, x_i) + (-1)^{i+n} (1 + 2\eta)^n \bigl( - \partial_x z_0(t_n, x_i) \bigr).
% \end{split}
% \end{align}
% In der Summe ergibt das nun
% \begin{align*}
% 0 &= \partial_t u_0(n+1, i, t_n, x_i) + \partial_x u_0(n, i-1, t_n, x_i)\\
% &= \partial_t w_0(t_n, x_i) + \partial_x w_0(t_n, x_i)\\
% &\quad - (-1)^{i+n} (1 + 2\eta)^n \bigl( (1 + 2 \eta) \partial_t z_0(t_n, x_i) + \partial_x z_0(t_n, x_i) \bigr),
% \end{align*}
% also
% \begin{align}\label{eq:apx:beidenull}
% \partial_t w_0(t_n, x_i) + \partial_x w_0(t_n, x_i) = (-1)^{i+n} (1 + 2\eta)^n \bigl( (1 + 2 \eta) \partial_t z_0(t_n, x_i) + \partial_x z_0(t_n, x_i) \bigr).
% \end{align}
% Wir verwenden nun Lemma \ref{lemma:transport:diskret:konvergenz_gitter}, um zu zeigen, dass beide Seiten umanhängig von einander Null sein müssen.
% \begin{satz}
% Für alle $(t, x) \in \Rp \times \R$ gilt
% \begin{align}\label{eq:apx:wgl}
% \partial_t w_0(t, x) + \partial_x w_0(t, x) &= 0 \qquad \text{und}\\ 
% (1 + 2 \eta) \partial_t z_0(t, x) + \partial_x z_0(t, x) &= 0. \label{eq:apx:zgl}
% \end{align} 
% \end{satz}
% \begin{proof}
% In das Lemma setzen wir
% \[ F(t,x) = \partial_t w_0(t, x) + \partial_x w_0(t, x)\] 
% und
% \begin{align*}
% f(n,i,t,x) &= (-1)^{i+n} (1 + 2\eta)^n \bigl( \underbrace{(1 + 2 \eta) \partial_t z_0(t, x) + \partial_x z_0(t, x)}_{=: g(t,x)} \bigr) = (-1)^{i+n} (1 + 2\eta)^n g(t,x)
% \end{align*}
% ein, wobei wegen $w_0, z_0 \in \Cinf$ die beiden Abbildungen $F$ und $g$ beide stetig differenzierbar sind.
% Es existiert also für alle $(t,x) \in \Rp \times \R$ und $n \in \N$ ein $i(n) \in \Z$, so dass die Ungleichung \eqref{eq:lemma:diag:aussage}
% \begin{align}
% \abs{F(t, x) - f(n, i(n), t, x)} \leq \frac {t}{\lambda n} \Bigl(\bigl\lvert f_x(n, i(n), t, x) \bigr\rvert + \bigl\lvert F_x(t, x) \bigr\rvert \Bigr)
% \end{align}
% erfüllt ist.
% Wegen $\abs{f(n, i(n), t, x)} - \abs{F(t,x)} \leq \abs{F(t, x) - f(n, i(n), t, x)}$ und $\frac {t}{\lambda n} \bigl\lvert F_x(t, x) \bigr\rvert \to 0$ für $n \to \infty$, impliziert das
% \[ \abs{f(n, i(n), t, x)} \leq \frac {t}{\lambda n} \abs{f_x(n, i(n), t, x)} +  \frac {t}{\lambda n} \bigl\lvert F_x(t, x) \bigr\rvert + \abs{F(t,x)} \]
% also
% \begin{align}
% (1 + 2\eta)^n \abs{g(t,x)} &\leq \frac{t}{\lambda n} \cdot (1 + 2\eta)^n \abs{\partial_x g(t,x)} + \frac {t}{\lambda n} \bigl\lvert \partial_x F(t, x) \bigr\rvert + \abs{F(t,x)}\\
% \Leftrightarrow \qquad \abs{g(t,x)} &\leq \underbrace{\frac{t}{\lambda n} \abs{\partial_x g(t,x)}}_{\to 0} + \underbrace{(1 + 2 \eta)^{-n} \left( \frac {t}{\lambda n} \bigl\lvert \partial_x F(t, x) \bigr\rvert + \abs{F(t,x)} \right)}_{\to 0} \to 0 \qquad \text{für $n \to \infty$},
% \end{align}
% woraus $g(t,x) = 0$ und somit auch $F(t,x) = 0$ für alle $(t,x) \in \Rp \times \R$ folgt.
% \end{proof}
% Aus den Gleichungen \eqref{eq:apx:wgl} und \eqref{eq:apx:zgl} folgt insbesondere auch
% \[ \partial^2_t w_0 = \partial^2_x w_0 \quad \text{und} \quad \partial^2_t z_0 = \frac {1}{(2\lambda - 1)^2} \partial^2_x z_0. \]

% \vspace{0.4cm}
% \noindent \textbf{In der Ordnung $O(h^2)$:}
% \begin{align*}
% \partial_t u_1(n+1, i, t_n, x_i) + \partial_x u_1(n, i-1, t_n, x_i) &=
% &\frac {1}{2} \partial^2_x u_0(n, i-1, t_n, x_i) - \frac{\lambda}{2} \partial^2_t u_0(n+1, i, t_n, x_i)
% \end{align*}
% Es gilt 
% \begin{align*}
% \frac {1}{2} \partial^2_x u_0(n, i-1, t_n, x_i) \qquad\\
% - \frac{\lambda}{2} \partial^2_t u_0(n+1, i, t_n, x_i)
% &= \frac{1}{2} \bigl( \partial^2_x w_0(t_n, x_i) + (-1)^{i-1} (1 - 2\lambda)^n \partial^2_x z_0(t_n, x_i) \bigr) \\
% &\qquad - \frac{\lambda}{2} \bigl( \partial^2_t w_0(t_n, x_i) + (-1)^i (1 - 2\lambda)^{n+1} \partial^2_t z_0(t_n, x_i) \bigr)\\
% &= \frac{1 - \lambda}{2}  \partial^2_x w_0(t_n, x_i)\\
% &\qquad + (-1)^i (1 - 2\lambda)^{n} \left( - \frac{1}{2} \partial^2_x z_0(t_n, x_i) - \frac{\lambda (1 - 2\lambda)}{2} \partial^2_t z_0(t_n, x_i) \right)\\
% &= \frac{1 - \lambda}{2}  \partial^2_x w_0(t_n, x_i)\\
% &\qquad + (-1)^i (1 - 2\lambda)^{n} \left( - \frac{1}{2} \partial^2_x z_0(t_n, x_i) - \frac{\lambda (1 - 2\lambda)}{2 (2 \lambda - 1)^2} \partial^2_x z_0(t_n, x_i) \right)\\
% &= \frac{1 - \lambda}{2}  \partial^2_x w_0(t_n, x_i)\\
% &\qquad + (-1)^i (1 - 2\lambda)^{n} \left( - \frac{1}{2} \partial^2_x z_0(t_n, x_i) + \frac{\lambda}{2 (2 \lambda - 1)} \partial^2_x z_0(t_n, x_i) \right)\\
% &= \frac{1 - \lambda}{2}  \partial^2_x w_0(t_n, x_i)\\
% &\qquad + (-1)^i (1 - 2\lambda)^{n} \left( \frac{1 - \lambda}{2 (2 \lambda - 1)} \partial^2_x z_0(t_n, x_i) \right)
% \end{align*}
% und ganz analog wie bei $O(h)$ gilt ferner
% \begin{align*}
% \partial_t u_1(n+1, i, t_n, x_i) \qquad\\
% + \: \partial_x u_1(n, i-1, t_n, x_i) &= \partial_t w_1(t_n, x_i) + \partial_x w_1(t_n, x_i)\\
% &\qquad + (-1)^i (1 - 2\lambda)^n \bigl( (1 - 2\lambda) \partial_t z_1(t_n, x_i) - \partial_x z_1(t_n, x_i) \bigr) \\
% \end{align*}
% und hieraus folgt für alle $(t_n, x_i)$
% \begin{align*}
% \partial_t w_1(t_n, x_i) + \partial_x w_1(t_n, x_i) &= \frac{1 - \lambda}{2}  \partial^2_x w_0(t_n, x_i)\\
% \partial_t z_1(t_n, x_i) + \frac{1}{2 \lambda - 1} \partial_x z_1(t_n, x_i) &= \frac{\lambda - 1}{2 (2 \lambda - 1)^2} \partial^2_x z_0(t_n, x_i)
% \end{align*}
\clearpage
\subsection{Rechnung des Ansatzes für kleine Störungen der CFL Zahl}
\label{appendix:kleineta:rechnung}
% Unsere Ansatzfunktionen sind dieses mal durch
% \begin{align} u_k(n, i, t, x) = w_k(t,x) + (-1)^{i+n} z_k(t, x) \qquad k = 0,1,2 \end{align}
% gegeben. Hieraus folgt zunächst für alle $k = 0,1,2$
% \begin{align}\label{eq:appendix:kleineta:bedingung}
% u_k(n+1, i, t, x) - u_k(n, i-1, t, x) = 0.
% \end{align}
% und somit gilt in Ordnung $O(h)$ auf der linken Seite von \eqref{eq:transport:kleineta:oh}
% \begin{align}\label{eq:appendix:kleineta:oh:linkeseite}
% \partial_t u_0(n+1, i, t_n, x_i) \qquad&\nonumber \\
% + \partial_x u_0(n, i-1, t_n, x_i) &= \partial_t w_0(t_n,x_i) + (-1)^{i+n+1} \partial_t z_0(t_n,x_i)\nonumber\\
% &\qquad + \partial_x w_0(t_n,x_i) + (-1)^{i+n-1} \partial_x z_0(t_n,x_i)\nonumber\\
% \begin{split}
% &= \partial_t w_0(t_n,x_i) + \partial_x w_0(t_n,x_i) \\
% &\qquad - (-1)^{i+n} \bigl( \partial_t z_0(t_n,x_i) + \partial_x z_0(t_n,x_i) \bigr).
% \end{split}
% \end{align}
% Auf der rechten Seite von \eqref{eq:transport:kleineta:oh} finden wir wegen \eqref{eq:appendix:kleineta:bedingung}
% \begin{align}\label{eq:appendix:kleineta:oh:rechteseite}
% - \bigl( u_0(n, i, t_n, x_i) - u_0(n, i-1, t_n, x_i) \bigr) &= -\Bigl( w_0(t_n, x_i) + (-1)^{i+n} z_0(t_n, x_i)\nonumber\\
% &\qquad - \bigl( w_0(t_n, x_i) + (-1)^{i+n-1} z_0(t_n, x_i) \bigr) \Bigr) \nonumber\\
% &= - (-1)^{i+n} 2 z_0(t_n,x_i) 
% \end{align}
% Setzt man \eqref{eq:appendix:kleineta:oh:linkeseite} und \eqref{eq:appendix:kleineta:oh:rechteseite} nun gleich, so leitet man hieraus in $O(h)$ die beiden Gleichungen
% \begin{align}\label{eq:appendix:kleineta:oh}
% \partial_t w_0(t,x) + \partial_x w_0(t,x) &= 0 \qquad \text{und}\\
% \partial_t z_0(t,x) + \partial_x z_0(t,x) &= 2 z_0(t,x)
% \end{align}
% ab. Insbesondere gelten dann
% \begin{align}\label{eq:appendix:kleineta:korollar:h2}
% \begin{split}
% \partial^2_t w_0(t,x) &= \partial^2_x w_0(t,x)\qquad\\&\text{und}\\
% \partial^2_t z_0(t,x) &= \partial_t \bigl( 2z_0(t,x)  - \partial_x z_0(t,x) \bigr)\\
% &= 2 \partial_t z_0(t,x) - \partial_t \bigl( \partial_x z_0(t,x) \bigr)\\
% &= 2 \partial_t z_0(t,x) - \partial_x \bigl( \partial_t z_0(t,x) \bigr)\\
% &= 2 \bigl( 2z_0(t,x) - \partial_x z_0(t,x) \bigr) - \partial_x \bigl( 2 z_0(t,x) - \partial_x z_0(t,x) \bigr)\\
% &= 4 z_0(t,x) - 4 \partial_x z_0(t,x) + \partial^2_x z_0(t,x).
% \end{split}
% \end{align}

Es sei der Ansatz
\begin{align}
\begin{split}
v^n_i &= w_0(t_n, x_i) + h w_1(t_n, x_i) + h^2 w_2(t_n, x_i)\\
&\quad + (-1)^{i+n} \bigl( z_0(t_n, 2 h t_n, x_i) + + h z_1(t_n, 2 h t_n, x_i) + h^2 z_2(t_n, 2 h t_n, x_i) \bigr) + O(h^3)
\end{split}\\
&= w_0 + h w_1 + h^2 w_2 + (-1)^{i+n} (z_0 + h z_1 + h^2 z_2) + O(h^3)
\end{align}
für glatte Abbildungen $w_k(t,x)$ und $z_k(t, \tau, x)$ gegeben.
Für alle $k = 0, 1, 2$ gilt
{\small
\begin{align}
\begin{split}
w_k(t_{n+1}, x_i) &= w_k(t_n + h(1+h), x_i)\\
&= \left(1 + h(1+h) \partial_t + \frac{h^2(1+h)^2}{2} \partial^2_t + \frac{h^3(1+h)^3}{6} \partial^3_t \right) w_k + O(h^4)\\
&= \left(1 + h(1+h) \partial_t + \frac{h^2(1+2h)}{2} \partial^2_t + \frac{h^3}{6} \partial^3_t \right) w_k + O(h^4)\\
&= \left(1 + h \partial_t + h^2 \left( \partial_t + \frac{1}{2} \partial^2_t \right) + h^3 \left( \partial^2_t + \frac{1}{6} \partial^3_t \right) \right) w_k + O(h^4)
\end{split}
\end{align}
}
{\small
\begin{align}
\begin{split}
z_k(t_{n+1}, 2h t_{n+1}, x_i) &= z_k(t_n + h(1+h), 2 h t_n + 2 h^2(1+h), x_i)\\
&= \left(1 + h(1+h) \partial_t + 2 h^2(1+h) \partial_\tau + \frac{h^2(1+h)^2}{2} \partial^2_t + \frac{h^3(1+h)^3}{6} \partial^3_t \right) z_k + O(h^4)\\
&= \left(1 + h(1+h) \partial_t + 2 h^2(1+h) \partial_\tau + \frac{h^2(1+2h)}{2} \partial^2_t + \frac{h^3}{6} \partial^3_t \right) z_k + O(h^4)\\
&= \left(1 + h \partial_t + h^2 \left( \partial_t + 2 \partial_\tau + \frac{1}{2} \partial^2_t \right) + h^3 \left( 2 \partial_\tau + \partial^2_t + \frac{1}{6} \partial^3_t \right) \right) z_k + O(h^4)
\end{split}
\end{align}
}
{\small
\begin{align}
\begin{split}
(1 + h) w_k(t_n, x_{i-1}) &= (1+h) w_k(t_n, x_i - h)\\
&= (1 + h) \left( 1 - h \partial_x + \frac{h^2}{2} \partial^2_x - \frac {h^3}{6} \partial^3_x \right) w_k + O(h^4)\\
&= \left( 1 + h( 1 - \partial_x ) + h^2  \left( -\partial_x + \frac{1}{2} \partial^2_x \right) + h^3 \left(\frac{1}{2} \partial^2_x - \frac{1}{6} \partial^3_x \right) \right) w_k + O(h^4)
\end{split}
\end{align}
}
{\small
\begin{align}
\begin{split}
(1 + h) z_k(t_n, 2 h t_n, x_{i-1}) &= (1+h) z_k(t_n, 2 h t_n, x_i - h)\\
&= (1 + h) \left( 1 - h \partial_x + \frac{h^2}{2} \partial^2_x - \frac {h^3}{6} \partial^3_x \right) z_k + O(h^4)\\
&= \left( 1 + h( 1 - \partial_x ) + h^2  \left( -\partial_x + \frac{1}{2} \partial^2_x \right) + h^3 \left(\frac{1}{2} \partial^2_x - \frac{1}{6} \partial^3_x \right) \right) z_k + O(h^4).
\end{split}
\end{align}
}
Es folgt einerseits die Rechnung
\begin{align}
\begin{split}
v^{n+1}_i - v^n_i &= w_0(t_{n+1}, 2 h t_{n+1}, x_i) - w_0(t_n, 2 h t_n, x_i)\\
&\quad + h \left(  w_1(t_{n+1}, 2 h t_{n+1}, x_i) - w_1(t_n, 2 h t_n, x_i) \right)\\
&\quad + h^2 \left(  w_2(t_{n+1}, 2 h t_{n+1}, x_i) - w_2(t_n, 2 h t_n, x_i) \right)\\
&\quad + (-1)^{i+n+1} z_0(t_{n+1}, 2 h t_{n+1}, x_i) - (-1)^{n+i} z_0(t_n, 2 h t_n, x_i)\\
&\quad + h \left( (-1)^{i+n+1} z_1(t_{n+1}, 2 h t_{n+1}, x_i) - (-1)^{n+i} z_1(t_n, 2 h t_n, x_i) \right)\\
&\quad + h^2 \left( (-1)^{i+n+1} z_2(t_{n+1}, 2 h t_{n+1}, x_i) - (-1)^{n+i} z_2(t_n, 2 h t_n, x_i) \right)
\end{split}\\
\begin{split}
&= \left( h \partial_t + h^2 \left( \partial_t + \frac{1}{2} \partial^2_t \right) + h^3 \left( \partial^2_t + \frac{1}{6} \partial^3_t \right) \right) w_0\\
&\quad + h \left(h \partial_t + h^2 \left( \partial_t + \frac{1}{2} \partial^2_t \right) \right) w_1\\
&\quad + h^3 \partial_t w_2 + O(h^4)\\
&\quad - (-1)^{i+n}     \left( z_0(t_{n+1}, 2 h t_{n+1}, x_i) + z_0(t_n, 2 h t_n, x_i) \right)\\
&\quad - (-1)^{i+n} h   \left( z_1(t_{n+1}, 2 h t_{n+1}, x_i) + z_1(t_n, 2 h t_n, x_i) \right)\\
&\quad - (-1)^{i+n} h^2 \left( z_2(t_{n+1}, 2 h t_{n+1}, x_i) + z_2(t_n, 2 h t_n, x_i) \right)
\end{split}\\
\begin{split}
&= \left( h \partial_t + h^2 \left( \partial_t + \frac{1}{2} \partial^2_t \right) + h^3 \left( \partial^2_t + \frac{1}{6} \partial^3_t \right) \right) w_0\\
&\quad + h \left(h \partial_t + h^2 \left( \partial_t + \frac{1}{2} \partial^2_t \right) \right) w_1\\
&\quad + h^3 \partial_t w_2\\
&\quad - (-1)^{i+n}     \left( 2 + h \partial_t + h^2 \left( \partial_t + 2 \partial_\tau + \frac{1}{2} \partial^2_t \right) + h^3 \left( 2 \partial_\tau + \partial^2_t + \frac{1}{6} \partial^3_t \right) \right) z_0\\
&\quad - (-1)^{i+n} h   \left( 2 + h \partial_t + h^2 \left( \partial_t + 2 \partial_\tau + \frac{1}{2} \partial^2_t \right) \right) z_1\\
&\quad - (-1)^{i+n} h^2 \left( 2 + h \partial_t \right) z_2 + O(h^4)
\end{split}\\
\label{eq:apx:mega_lang1}
\begin{split}
&=  h \partial_t w_0 \\
&\quad + h^2 \left( \partial_t w_1 + \partial_t w_0 + \frac{1}{2} \partial^2_t w_0 \right)\\
&\quad + h^3 \left( \partial_t w_2 + \partial_t w_1 + \frac{1}{2} \partial^2_t w_1 + \partial^2_t w_0 + \frac{1}{6} \partial^3_t w_0 \right)\\
&\quad - (-1)^{i+n} \left( 2 z_0 + h 2 z_1 + h^2 2 z_2 \right)\\
&\quad - (-1)^{i+n} h \partial_t z_0\\
&\quad - (-1)^{i+n} h^2 \left( \partial_t z_1 + \partial_t z_0 + 2 \partial_\tau z_0 + \frac{1}{2} \partial^2_t z_0 \right)\\
&\quad - (-1)^{i+n} h^3 \Bigl( \partial_t z_2 + \partial_t z_1 + 2 \partial_\tau z_1 + \frac{1}{2} \partial^2_t z_1\\
&\qquad + 2 \partial_\tau z_0 + \partial^2_t z_0+ \frac{1}{6} \partial^3_t z_0 \Bigr) + O(h^4)\\
\end{split}
\end{align}
und andererseits auch die Rechnung
{\small
\begin{align}
\begin{split}
(1 + h) \left(v^n_i - v^n_{i-1}\right) &= (1 + h) \left( w_0(t_n, 2 h t_n, x_i) - w_0(t_n, 2 h t_n, x_{i-1}) \right)\\
&\quad + h (1+h) \left( w_1(t_n, 2 h t_n, x_i) - w_1(t_n, 2 h t_n, x_{i-1}) \right)\\
&\quad + h^2 (1+h) \left( w_2(t_n, 2 h t_n, x_i) - w_2(t_n, 2 h t_n, x_{i-1}) \right)\\
&\quad + (1+h) \left( (-1)^{i+n} z_0(t_n, 2 h t_n, x_i) - (-1)^{i+n-1} z_0(t_n, 2 h t_n, x_{i-1}) \right)\\
&\quad + h (1 + h) \left( (-1)^{i+n} z_1(t_n, 2 h t_n, x_i) - (-1)^{i+n-1} z_1(t_n, 2 h t_n, x_{i-1}) \right)\\
&\quad + h^2 (1 + h) \left( (-1)^{i+n} z_2(t_n, 2 h t_n, x_i) - (-1)^{i+n-1} z_2(t_n, 2 h t_n, x_{i-1}) \right)
\end{split}\\
\begin{split}
&= (1 + h) \left( w_0(t_n, 2 h t_n, x_i) - w_0(t_n, 2 h t_n, x_{i-1}) \right)\\
&\quad + h (1+h) \left( w_1(t_n, 2 h t_n, x_i) - w_1(t_n, 2 h t_n, x_{i-1}) \right)\\
&\quad + h^2 (1+h) \left( w_2(t_n, 2 h t_n, x_i) - w_2(t_n, 2 h t_n, x_{i-1}) \right)\\
&\quad + (-1)^{i+n} (1+h) \left(  z_0(t_n, 2 h t_n, x_i) + z_0(t_n, 2 h t_n, x_{i-1}) \right)\\
&\quad + (-1)^{i+n} h (1 + h) \left( z_1(t_n, 2 h t_n, x_i) + z_1(t_n, 2 h t_n, x_{i-1}) \right)\\
&\quad + (-1)^{i+n} h^2 (1 + h) \left( z_2(t_n, 2 h t_n, x_i) + z_2(t_n, 2 h t_n, x_{i-1}) \right)
\end{split}\\
\begin{split}
&= \left( h \partial_x + h^2  \left( \partial_x - \frac{1}{2} \partial^2_x \right) + h^3 \left(- \frac{1}{2} \partial^2_x + \frac{1}{6} \partial^3_x \right) \right) w_0\\
&\quad + h \left( h \partial_x + h^2  \left( \partial_x - \frac{1}{2} \partial^2_x \right) \right) w_1\\
&\quad + h^3 \partial_x  w_2\\
&\quad + (-1)^{i+n} \left( 2 + h( 2 - \partial_x ) + h^2  \left( -\partial_x + \frac{1}{2} \partial^2_x \right) + h^3 \left(\frac{1}{2} \partial^2_x - \frac{1}{6} \partial^3_x \right) \right) z_0\\
&\quad + (-1)^{i+n} h \left( 2 + h( 2 - \partial_x ) + h^2  \left( -\partial_x + \frac{1}{2} \partial^2_x \right) \right) z_1\\
&\quad + (-1)^{i+n} h^2 \left( 2 + h( 2 - \partial_x ) \right) z_2 + O(h^4)
\end{split}\\
\label{eq:apx:mega_lang2}
\begin{split} 
&= h \partial_x w_0\\
&\quad + h^2 \left( \partial_x w_1 + \partial_x w_0 - \frac{1}{2} \partial^2_x w_0 \right)\\
&\quad + h^3 \left( \partial_x  w_2 + \partial_x w_1 - \frac{1}{2} \partial^2_x w_1 - \frac{1}{2} \partial^2_x w_0 + \frac{1}{6} \partial^3_x w_0 \right)\\
&\quad + (-1)^{i+n} \left( 2 z_0 + h 2 z_1 + h^2 2 z_2 \right)\\
&\quad + (-1)^{i+n} h \left( 2 z_0 - \partial_x z_0 \right)\\
&\quad + (-1)^{i+n} h^2 \left( 2 z_1 - \partial_x z_1 -\partial_x z_0 + \frac{1}{2} \partial^2_x z_0 \right)\\
&\quad + (-1)^{i+n} h^3 \left( 2 z_2 - \partial_x z_2 -\partial_x z_1 + \frac{1}{2} \partial^2_x z_1 + \frac{1}{2} \partial^2_x z_0 - \frac{1}{6} \partial^3_x z_0 \right) + O(h^4).
\end{split}
\end{align}
}
Wir summieren \eqref{eq:apx:mega_lang1} und \eqref{eq:apx:mega_lang2} und erhalten
\begin{align}\begin{split}
& v^{n+1}_i - v^n_i + (1 + h) \left(v^n_i - v^n_{i-1}\right)\\
&= h \left( \partial_t w_0 + \partial_x w_0 \right)\\
&\quad + h^2\left( \partial_t w_1 + \partial_t w_0 + \frac{1}{2} \partial^2_t w_0 + \partial_x w_1 + \partial_x w_0 - \frac{1}{2} \partial^2_x w_0 \right)\\
&\quad + h^3 \Bigl( \partial_t w_2 + \partial_t w_1 + \frac{1}{2} \partial^2_t w_1 + \partial^2_t w_0 + \frac{1}{6} \partial^3_t w_0\\
&\qquad +  \partial_x  w_2 + \partial_x w_1 - \frac{1}{2} \partial^2_x w_1 - \frac{1}{2} \partial^2_x w_0 + \frac{1}{6} \partial^3_x w_0 \Bigr)\\
&\quad + (-1)^{i+n} h \left( - \partial_t z_0 - \partial_x z_0 + 2 z_0 \right)\\
&\quad + (-1)^{i+n} h^2 \left( - \partial_t z_1 - \partial_x z_1 + 2 z_1 - \partial_x z_0 + \frac{1}{2} \partial^2_x z_0 - \partial_t z_0 - 2 \partial_\tau z_0 - \frac{1}{2} \partial^2_t z_0  \right)\\
&\quad + (-1)^{i+n} h^3 \Bigl( - \partial_t z_2 - \partial_t z_1 - 2 \partial_\tau z_1 - \frac{1}{2} \partial^2_t z_1 - 2 \partial_\tau z_0 - \partial^2_t z_0 - \frac{1}{6} \partial^3_t z_0\\
&\qquad + 2 z_2 - \partial_x z_2 -\partial_x z_1 + \frac{1}{2} \partial^2_x z_1 + \frac{1}{2} \partial^2_x z_0 - \frac{1}{6} \partial^3_x z_0 \Bigr) + O(h^4).
\end{split}
\end{align}
Für jede Ordnung in $h$ erhalten wir mit Hilfe des Satzes \ref{satz:glgregulaer} je für den glatten Anteil $w$ und auch oszillierenden Anteil $z$ eine Gleichung.
Für $w$ gelten dann die Gleichungen
\begin{align}
\begin{split}
\partial_t w_0 + \partial_x w_0 &= 0\\ 
\partial_t w_1 + \partial_x w_1 + \underbrace{\partial_t w_0 + \partial_x w_0}_{= 0} &= \frac{1}{2} \bigl( \underbrace{ \partial^2_x w_0 - \partial^2_t w_0}_{= 0} \bigr)\\
\partial_t w_2 + \partial_x  w_2 + \underbrace{\partial_t w_1 + \partial_x w_1}_{= 0} &= \frac{1}{2} \bigl( \underbrace{\partial^2_x w_0 - 2 \partial^2_t w_0}_{= - \partial^2_x w_0} \bigr)\\
&\quad + \frac{1}{2} \bigl( \underbrace{\partial^2_x w_1 - \partial^2_t w_1}_{= 0} \bigr) - \frac{1}{6} \bigl( \underbrace{\partial^3_t w_0 +  \partial^3_x w_0}_{= 0} \bigr)
\end{split}
\end{align}
also
\begin{align}\label{eq:apx:fuerw}
\partial_t w_0 + \partial_x w_0 &= 0\\ 
\partial_t w_1 + \partial_x w_1 &= 0\\
\partial_t w_2 + \partial_x w_2 &= -\frac{1}{2} \partial^2_x w_0\\
\begin{split}
w_0(0, 0, x) &= U(x),\\
w_1(0, 0, x) &= 0, \qquad \qquad \forall x \in \R\\
w_2(0, 0, x) &= 0.
\end{split}
\end{align}
Für die Abbildungen $z_k$, $k = 0,1,2$ folgt hingegen
\begin{align}
\begin{split}
\partial_t z_0 + \partial_x z_0 &= 2 z_0\\ 
\partial_t z_1 + \partial_x z_1 &= 2 z_1 \underbrace{- \bigl( \partial_t z_0 + \partial_x z_0 \bigr) + \frac{1}{2} \left(\partial^2_x z_0 - \partial^2_t z_0\right) - 2 \partial_\tau z_0}_{= 0}\\
\partial_t z_2 + \partial_x z_2 &= 2 z_2 - \frac{1}{6} \bigl( \partial^3_x z_0 + \partial^3_t z_0 \bigr) + \underbrace{\frac{1}{2} \bigl(\partial^2_x z_0 - \partial^2_t z_0 \bigr) - 2 \partial_\tau z_0}_{= \partial_t z_0 + \partial_x z_0 = 2z_0} - \frac{1}{2} \partial^2_t z_0\\
&\quad \underbrace{ - \bigl(\partial_t z_1 +\partial_x z_1 \bigr) + \frac{1}{2}\bigl( \partial^2_x z_1 - \partial^2_t z_1 \bigr) - 2 \partial_\tau z_1}_{= 0}\\
z_0(0,0,x) &= U(x),\\
z_1(0,0,x) &= 0,\\
z_2(0,0,x) &= 0.
\end{split}
\end{align}

% \subsection{WKB Ansatz}

% Es seien die Ansätze 
% \begin{align}\label{wkb:ansatz}
% u_l(n,k,t,x) = w_l(t,x) + e^{i (\omega k - \theta n)} z_l(t,x) \quad \text{für $l=0,1,2$}
% \end{align}
% gegeben. Dann gilt
% {\small\begin{align}\nonumber
% u_l(n+1,k,t_n,x_k) - u_l(n,k,t_n,x_k)
% &= w_l(t,x) + e^{i (\omega k - \theta (n+1))} z_l(t,x) - \bigl( w_l(t,x) + e^{i (\omega k - \theta n)} z_l(t,x) \bigr)\\
% &= e^{i (\omega k - \theta n)} (e^{-i\theta} - 1) z_l(t,x).
% \end{align}}
% und
% {\small\begin{align}\nonumber
% u_l(n,k,t_n,x_k) - u_l(n,k-1,t_n,x_k)
% &= w_l(t,x) + e^{i (\omega k - \theta n)} z_l(t,x) - \bigl( w_l(t,x) + e^{i (\omega (k-1) - \theta n)} z_l(t,x) \bigr)\\
% &= e^{i (\omega k - \theta n)} (1 - e^{-i \omega}) z_l(t,x).
% \end{align}
% also folgt aus
% \begin{align}\label{eq:wkb:bed1}
% u_0(n+1,k,t_n,x_k) - u_0(n,k,t_n,x_k) + \lambda (u_0(n,k,t_n,x_k) - u_0(n,k-1,t_n,x_k)) = 0
% \end{align}
% auch
% \begin{align}
% 0 &= e^{i (\omega k - \theta n)} (e^{-i\theta} - 1) z_0(t,x) + \lambda e^{i (\omega k - \theta n)} (1 - e^{-i \omega}) z_0(t,x)\\
% &= e^{i (\omega k - \theta n)} z_0(t,x) ( \lambda - 1 + e^{-i\theta} - \lambda e^{-i \omega} )
% \end{align}
% Somit muss
% \begin{align}
% e^{-i \theta} - \lambda e^{-i \omega} = 1 - \lambda
% \end{align}
% gelten und das liefert uns das Gleichungssystem
% \begin{align}\label{eq:wkb:gl1}
% \Ima\left( e^{-i \theta} - \lambda e^{-i \omega} \right) &= 0 \qquad \text{und}\\
% \Ree\left( e^{-i \theta} - \lambda e^{-i \omega} \right) &= 1 - \lambda. \label{eq:wkb:gl2}
% \end{align}
% Die Abbildungen $S_1\colon \theta \mapsto e^{-i \theta}$ bzw. $S_\lambda\colon \omega \mapsto \lambda e^{- \omega}$ beschreiben jeweils einen Kreis in der komplexen Zahleneben mit den Radien $1$ und $\lambda > 1$.
% Wegen der Dreiecksungleichung haben zwei Komplexe Zahlen $z_\theta \in S_1$ und $z_\omega \in S_\lambda$ nur dann Abstand $\abs{z_\theta - z_\omega} = \lambda-1$, wenn $z_\theta = \lambda z_\omega =: \lambda z$ gilt.
% Das folgt beispielsweise aus
% \begin{align}
% \left(\abs{z_\theta - z_\omega}\right)^2 = (z_\theta - z_\omega)^2 &= (1 - \lambda)^2\\
% z_\theta^2 - 2 z_\theta z_\omega + z_\omega^2 &= 1 - 2 \lambda + \lambda^2 & \Leftrightarrow\\
% z_\theta z_\omega &= \lambda = \abs{z_\theta} \cdot \abs{z_\omega}  &\Leftrightarrow
% \end{align}
% Aus Gleichung \eqref{eq:wkb:gl1} folgt, dass $z$ reell sein muss und aus \eqref{eq:wkb:gl2} folgt $z = 1$.
% Im Wesentlichen kommt nun raus, dass der Ansatz \eqref{wkb:ansatz} nur funktionieren kann, wenn wir keine Oszillationen annehmen.
% Der Unterschied zum diskreten Ansatz mit alternierenden Gitterzeichen ist, dass wir hier keine in $n$ steigende Amplitudenfunktion
% angenommen haben, die uns einen zusätzlichen Freiheitsgrad bei der Bestimmung der gibt.
% Das korrigieren wir, indem wir $\Omega(n)$ als Faktor hinzufügen und so erhalten wir nun den diskreten Ansatz
% \begin{align}\label{wkb:diskret:ansatz}
% u_l(n,k,t,x) = w_l(t,x) + \Omega(n) e^{i (\omega k - \theta n)} z_l(t,x) \quad \text{für $l=0,1,2$}
% \end{align}
% mit den Gleichungen
% {\small\begin{align}\nonumber
% u_0(n+1, k, t_n, x_k) \quad &\\
% - u_0(n, k, t_n, x_k) &= w_0 + \Omega(n+1) e^{i (\omega k - \theta (n+1))} z_0(t,x) - \bigl( w_0(t,x) + \Omega(n) e^{i (\omega k - \theta n)} z_0(t,x) \bigr)\\
% &= e^{i (\omega k - \theta n)} (\Omega(n+1) e^{-i\theta} - \Omega(n)) z_0(t,x).
% \end{align}}
% und
% {\small\begin{align}\nonumber
% u_0(n,k,t_n,x_k) \qquad&\\
% - u_0(n,k-1,t_n,x_k)
% &= w_0(t,x) + \Omega(n) e^{i (\omega k - \theta n)} z_0(t,x) - \bigl( w_0(t,x) + \Omega(n) e^{i (\omega (k-1) - \theta n)} z_0(t,x) \bigr)\\
% &= e^{i (\omega k - \theta n)} \Omega(n) (1 - e^{-i \omega}) z_0(t,x).
% \end{align}}
% Es gilt weiterhin immernoch \eqref{eq:wkb:bed1}, also folgt aus
% \begin{align}
% e^{i (\omega k - \theta n)} (\Omega(n+1) e^{-i\theta} - \Omega(n)) z_0(t,x) + \lambda e^{i (\omega k - \theta n)} \Omega(n) (1 - e^{-i \omega}) z_0(t,x) = 0
% \end{align}
% die Gleichung
% \begin{align}
% 0 &= \Omega(n+1) e^{-i\theta} - \Omega(n) + \lambda \Omega(n) (1 - e^{-i \omega})\\
% &= \Omega(n+1) e^{-i\theta} - \lambda \Omega(n) e^{-i\omega} + (\lambda - 1) \Omega(n)\\
% &= \Omega(n+1) e^{-i\theta} + \Omega(n) \bigl( \lambda (1 - e^{-i\omega}) - 1 \bigr)
% \end{align}

