%!TEX root=thesis.tex

\subsection*{Für eine allgemeine ortsabhängige Geschwindigkeit $a$}

Wir betrachten nun den Ansatz

\[ v^n_k = e^{i \omega k} \Omega(n) \bigl( u_0(t_n, x_k) + h u_1(t_n, x_k) \bigr) + O(h^2). \]

Es gilt
\begin{align}
\begin{split}
v^{n+1}_k - v^n_k &= e^{i \omega k} \Omega(n+1) \bigl( u_0(t_{n+1}, x_k) + h u_1(t_{n+1}, x_k) \bigr)
- e^{i \omega k} \Omega(n) \bigl( u_0(t_n, x_k) + h u_1(t_n, x_k) \bigr)\\
&= e^{i \omega k} \Omega(n+1) \bigl( u_0(t_n + h, x_k) + h u_1(t_n + h, x_k) \bigr)
- e^{i \omega k} \Omega(n) \bigl( u_0(t_n, x_k) + h u_1(t_n, x_k) \bigr)\\
&= e^{i \omega k} \Bigl( \Omega(n+1) \bigl( u_0(t_n, x_k) + h \partial_t u_0(t_n, x_k) + h^2 \frac{1}{2} \partial^2_t u_0(t_n, x_k)\\
&\quad + h u_1(t_n, x_k) + h^2 \partial_t u_1(t_n, x_k) \bigr) - \Omega(n) \bigl( u_0(t_n, x_k) + h u_1(t_n, x_k) \bigr) \Bigr)\\
&= e^{i \omega k} \Bigl( \bigl(\Omega(n+1) - \Omega(n) \bigr) u_0(t_n, x_k) + h \bigl(\Omega(n+1) - \Omega(n) \bigr) u_1(t_n, x_k)\\
&\quad + h \Omega(n+1) \partial_t u_0(t_n, x_k) + h^2 \Omega(n+1) \bigl( \frac{1}{2} \partial^2_t u_0(t_n, x_k) + \partial_t u_1(t_n, x_k) \bigr) \Bigr)
\end{split}
\end{align}
und
\begin{align}
\begin{split}
a(x_k) \bigl( v^n_k - v^n_{k-1} \bigr)
&= a(x_k) \Bigl( e^{i \omega k} \Omega(n) \bigl( u_0(t_n, x_k) + h u_1(t_n, x_k) \bigr)\\
&\quad - e^{i \omega (k-1)} \Omega(n) \bigl( u_0(t_n, x_{k-1}) + h u_1(t_n, x_{k-1}) \bigr) \Bigr)\\
&= a(x_k) e^{i \omega k} \Omega(n) \Bigl( \bigl( u_0(t_n, x_k) + h u_1(t_n, x_k) \bigr)\\
&\quad - e^{- i \omega} \bigl( u_0(t_n, x_k-h) + h u_1(t_n, x_k-h) \bigr) \Bigr)\\
&= a(x_k) e^{i \omega k} \Omega(n) \Bigl( \bigl( u_0(t_n, x_k) + h u_1(t_n, x_k) \bigr)\\
&\quad - e^{- i \omega} \bigl( u_0(t_n, x_k) - h \partial_x u_0(t_n, x_k) + h^2 \frac{1}{2} \partial^2_x u_0(t_n, x_k)\\
&\quad + h u_1(t_n, x_k) - h^2 \partial_x u_1(t_n, x_k) \bigr) \Bigr) + O(h^3)\\
&= a(x_k) e^{i \omega k} \Omega(n) \Bigl( \left(1-e^{-i \omega}\right) \bigl( u_0(t_n, x_k) + h u_1(t_n, x_k) \bigr)\\
&\quad + h e^{-i \omega} \partial_x u_0(t_n, x_k) + h^2 e^{-i \omega} \bigl( \partial_x u_1(t_n, x_k) - \frac{1}{2} \partial^2_x u_0(t_n, x_k) \bigr) \Bigr).
\end{split}
\end{align}
In Summe folgt
\begin{align}
\begin{split}
&v^{n+1}_k - v^n_k + a(x_k) \bigl( v^n_k - v^n_{k-1} \bigr)\\
&= e^{i \omega k} \Bigl( \bigl(\Omega(n+1) - \Omega(n) \bigr) u_0(t_n, x_k) + h \bigl(\Omega(n+1) - \Omega(n) \bigr) u_1(t_n, x_k)\\
&\quad + h \Omega(n+1) \partial_t u_0(t_n, x_k) + h^2 \Omega(n+1) \bigl( \frac{1}{2} \partial^2_t u_0(t_n, x_k) + \partial_t u_1(t_n, x_k) \bigr) \Bigr)\\
&\quad + a(x_k) e^{i \omega k} \Omega(n) \Bigl( \left(1-e^{-i \omega}\right) \bigl( u_0(t_n, x_k) + h u_1(t_n, x_k) \bigr)\\
&\quad + h e^{-i \omega} \partial_x u_0(t_n, x_k) + h^2 e^{-i \omega} \bigl( \partial_x u_1(t_n, x_k) - \frac{1}{2} \partial^2_x u_0(t_n, x_k) \bigr) \Bigr)\\
\end{split}\\
\label{eq:wkb:difflsg}
\begin{split}
&= e^{i \omega k} \Bigl(
\bigl(\Omega(n+1) - \Omega(n) \bigr) u_0(t_n, x_k) + a(x_k) \Omega(n) \left(1-e^{-i \omega}\right) u_0(t_n, x_k)\\
&\quad + h \bigl(\Omega(n+1) - \Omega(n) \bigr) u_1(t_n, x_k) + a(x_k) \Omega(n) \left(1-e^{-i \omega}\right) u_1(t_n, x_k) \bigr)\\
&\quad + h \bigl( \Omega(n+1) \partial_t u_0(t_n, x_k) + a(x_k) \Omega(n) e^{-i \omega} \partial_x u_0(t_n, x_k) \bigr)\\
&\quad + h^2 \bigl( \Omega(n+1) \partial_t u_1(t_n, x_k) + a(x_k) \Omega(n) e^{-i \omega} \partial_x u_1(t_n, x_k)\\
&\qquad + \frac{1}{2} \left( \Omega(n+1) \partial^2_t u_0(t_n, x_k) - a(x_k) \Omega(n) e^{-i \omega} \partial^2_x u_0(t_n, x_k) \right) \bigr) 
\Bigr)
\end{split}
\end{align}
und das heißt wir erhalten in führender Ordnung die Gleichung für alle $h > 0$, $n \in \N$ und $k \in \Z$
\begin{align}
\left( \Omega(n+1) - \left( 1 - a(x_k) \left(1-e^{-i \omega}\right) \right) \Omega(n) \right) u_0(t_n, x_k) = 0.
\end{align}
Hieraus folgt, dass 
$\Omega(n+1) = \left( 1 - a(x_k) \left(1-e^{-i \omega}\right) \right) \Omega(n)$
gelten muss, allerdings ist $\Omega$ nicht von $k$ abhängig und sobald es zwei Punkte $x,y \in \R$ mit $a(x) \neq a(y)$ gibt, folgt als einize sinnvolle Lösung $\Omega(n) = 0$.
Das Problem lässt sich angehen, indem man statt allgemeiner $a$ wieder wie in den Betrachtungen vorher kleine Störungen um 1 betrachtet.
D.\,h. wir setzen in die Gleichung \eqref{eq:wkb:difflsg} die Geschwindigkeit $a(x) = 1 + h b(x)$ ein.

\subsection*{Für den Fall $a(x) = 1 + h b(x)$}
{\small \begin{align}
\begin{split}
&v^{n+1}_k - v^n_k + (1 + h b(x_k)) \bigl( v^n_k - v^n_{k-1} \bigr)\\
&= e^{i \omega k} \Bigl(
\bigl(\Omega(n+1) - \Omega(n) \bigr) u_0(t_n, x_k) + (1 + h b(x_k)) \Omega(n) \left(1-e^{-i \omega}\right) u_0(t_n, x_k)\\
&\quad + h \bigl(\Omega(n+1) - \Omega(n) \bigr) u_1(t_n, x_k) + (1 + h b(x_k)) \Omega(n) \left(1-e^{-i \omega}\right) u_1(t_n, x_k) \bigr)\\
&\quad + h \bigl( \Omega(n+1) \partial_t u_0(t_n, x_k) + (1 + h b(x_k)) \Omega(n) e^{-i \omega} \partial_x u_0(t_n, x_k) \bigr)\\
&\quad + h^2 \bigl( \Omega(n+1) \partial_t u_1(t_n, x_k) + (1 + h b(x_k)) \Omega(n) e^{-i \omega} \partial_x u_1(t_n, x_k)\\
&\qquad + \frac{1}{2} \left( \Omega(n+1) \partial^2_t u_0(t_n, x_k) - (1 + h b(x_k)) \Omega(n) e^{-i \omega} \partial^2_x u_0(t_n, x_k) \right) \bigr)
\Bigr) + O(h^3)
\end{split}\\
\begin{split}
&= e^{i \omega k} \Bigl(
\bigl(\Omega(n+1) - \Omega(n) \bigr) u_0(t_n, x_k) + \Omega(n) \left(1-e^{-i \omega}\right) u_0(t_n, x_k)\\
&\quad + h \bigl(\Omega(n+1) - \Omega(n) \bigr) u_1(t_n, x_k) + \Omega(n) \left(1-e^{-i \omega}\right) u_1(t_n, x_k) \bigr)\\
&\quad + h \bigl( \Omega(n+1) \partial_t u_0(t_n, x_k) + \Omega(n) e^{-i \omega} \partial_x u_0(t_n, x_k) + b(x_k) \Omega(n) \left(1-e^{-i \omega}\right) u_0(t_n, x_k) \bigr)\\
&\quad + h^2 \bigl( \Omega(n+1) \partial_t u_1(t_n, x_k) + \Omega(n) e^{-i \omega} \partial_x u_1(t_n, x_k) + b(x_k) \Omega(n) e^{-i \omega} \partial_x u_0(t_n, x_k)\\
&\qquad + \frac{1}{2} \left( \Omega(n+1) \partial^2_t u_0(t_n, x_k) - \Omega(n) e^{-i \omega} \partial^2_x u_0(t_n, x_k) \right)\\
&\qquad + b(x_k) \Omega(n) \left(1-e^{-i \omega}\right) u_1(t_n, x_k) \bigr) 
\Bigr) + O(h^3)
\end{split}
\end{align}}
In erster Ordnung gilt also
\begin{align}
\left( \Omega(n+1) - \left( 1 - \left(1-e^{-i \omega}\right) \right) \Omega(n)  \right) u_0(t_n, x_k) = 0
\end{align}
und daher folgen wir $\Omega$