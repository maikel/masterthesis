%!TEX root=thesis.tex

Wir betrachten nun den Ansatz

\[ v^n_k = e^{i \omega k} \Omega(n) \bigl( u_0(t_n, x_k) + h u_1(t_n, x_k) \bigr) + O(h^2). \]

Es gilt
\begin{align} \label{wkb:zeitdiff}
\begin{split}
v^{n+1}_k - v^n_k
% &= e^{i \omega k} \Omega(n+1) \bigl( u_0(t_{n+1}, x_k) + h u_1(t_{n+1}, x_k) \bigr)
% - e^{i \omega k} \Omega(n) \bigl( u_0(t_n, x_k) + h u_1(t_n, x_k) \bigr)\\
% &= e^{i \omega k} \Omega(n+1) \bigl( u_0(t_n + h, x_k) + h u_1(t_n + h, x_k) \bigr)
% - e^{i \omega k} \Omega(n) \bigl( u_0(t_n, x_k) + h u_1(t_n, x_k) \bigr)\\
% &= e^{i \omega k} \Bigl( \Omega(n+1) \bigl( u_0(t_n, x_k) + h \partial_t u_0(t_n, x_k) + h^2 \frac{1}{2} \partial^2_t u_0(t_n, x_k)\\
% &\quad + h u_1(t_n, x_k) + h^2 \partial_t u_1(t_n, x_k) \bigr) - \Omega(n) \bigl( u_0(t_n, x_k) + h u_1(t_n, x_k) \bigr) \Bigr)\\
&= e^{i \omega k} \Bigl( \bigl(\Omega(n+1) - \Omega(n) \bigr) u_0(t_n, x_k) + h \bigl(\Omega(n+1) - \Omega(n) \bigr) u_1(t_n, x_k)\\
&\quad + h \Omega(n+1) \partial_t u_0(t_n, x_k) + h^2 \Omega(n+1) \bigl( \frac{1}{2} \partial^2_t u_0(t_n, x_k) + \partial_t u_1(t_n, x_k) \bigr) \Bigr) + O(h^3)
\end{split}
\end{align}
und
\begin{align}
\begin{split}\label{wkb:ortdiff}
a(x_k) \bigl( v^n_k - v^n_{k-1} \bigr)
% &= a(x_k) \Bigl( e^{i \omega k} \Omega(n) \bigl( u_0(t_n, x_k) + h u_1(t_n, x_k) \bigr)\\
% &\quad - e^{i \omega (k-1)} \Omega(n) \bigl( u_0(t_n, x_{k-1}) + h u_1(t_n, x_{k-1}) \bigr) \Bigr)\\
% &= a(x_k) e^{i \omega k} \Omega(n) \Bigl( \bigl( u_0(t_n, x_k) + h u_1(t_n, x_k) \bigr)\\
% &\quad - e^{- i \omega} \bigl( u_0(t_n, x_k-h) + h u_1(t_n, x_k-h) \bigr) \Bigr)\\
% &= a(x_k) e^{i \omega k} \Omega(n) \Bigl( \bigl( u_0(t_n, x_k) + h u_1(t_n, x_k) \bigr)\\
% &\quad - e^{- i \omega} \bigl( u_0(t_n, x_k) - h \partial_x u_0(t_n, x_k) + h^2 \frac{1}{2} \partial^2_x u_0(t_n, x_k)\\
% &\quad + h u_1(t_n, x_k) - h^2 \partial_x u_1(t_n, x_k) \bigr) \Bigr) + O(h^3)\\
&= a(x_k) e^{i \omega k} \Omega(n) \Bigl( \left(1-e^{-i \omega}\right) \bigl( u_0(t_n, x_k) + h u_1(t_n, x_k) \bigr)\\
&\quad + h e^{-i \omega} \partial_x u_0(t_n, x_k) + h^2 e^{-i \omega} \bigl( \partial_x u_1(t_n, x_k) - \frac{1}{2} \partial^2_x u_0(t_n, x_k) \bigr) \Bigr) + O(h^3).
\end{split}
\end{align}
In Summe folgt
\begin{align}\label{eq:wkb:difflsg}
\begin{split}
&v^{n+1}_k - v^n_k + a(x_k) \bigl( v^n_k - v^n_{k-1} \bigr)\\
% &= e^{i \omega k} \Bigl( \bigl(\Omega(n+1) - \Omega(n) \bigr) u_0(t_n, x_k) + h \bigl(\Omega(n+1) - \Omega(n) \bigr) u_1(t_n, x_k)\\
% &\quad + h \Omega(n+1) \partial_t u_0(t_n, x_k) + h^2 \Omega(n+1) \bigl( \frac{1}{2} \partial^2_t u_0(t_n, x_k) + \partial_t u_1(t_n, x_k) \bigr) \Bigr)\\
% &\quad + a(x_k) e^{i \omega k} \Omega(n) \Bigl( \left(1-e^{-i \omega}\right) \bigl( u_0(t_n, x_k) + h u_1(t_n, x_k) \bigr)\\
% &\quad + h e^{-i \omega} \partial_x u_0(t_n, x_k) + h^2 e^{-i \omega} \bigl( \partial_x u_1(t_n, x_k) - \frac{1}{2} \partial^2_x u_0(t_n, x_k) \bigr) \Bigr)\\
% \end{split}\\
% \begin{split}
&= e^{i \omega k} \Bigl(
\bigl(\Omega(n+1) - \Omega(n) \bigr) u_0(t_n, x_k) + a(x_k) \Omega(n) \left(1-e^{-i \omega}\right) u_0(t_n, x_k)\\
&\quad + h \bigl(\Omega(n+1) - \Omega(n) \bigr) u_1(t_n, x_k) + a(x_k) \Omega(n) \left(1-e^{-i \omega}\right) u_1(t_n, x_k) \bigr)\\
&\quad + h \bigl( \Omega(n+1) \partial_t u_0(t_n, x_k) + a(x_k) \Omega(n) e^{-i \omega} \partial_x u_0(t_n, x_k) \bigr)\\
&\quad + h^2 \bigl( \Omega(n+1) \partial_t u_1(t_n, x_k) + a(x_k) \Omega(n) e^{-i \omega} \partial_x u_1(t_n, x_k)\\
&\qquad + \frac{1}{2} \left( \Omega(n+1) \partial^2_t u_0(t_n, x_k) - a(x_k) \Omega(n) e^{-i \omega} \partial^2_x u_0(t_n, x_k) \right) \bigr) 
\Bigr)
\end{split}
\end{align}
und das heißt wir erhalten in führender Ordnung die Gleichung für alle $h > 0$, $n \in \N$ und $k \in \Z$
\begin{align}
\left( \Omega(n+1) - \left( 1 - a(x_k) \left(1-e^{-i \omega}\right) \right) \Omega(n) \right) u_0(t_n, x_k) = 0.
\end{align}
Hieraus folgt, dass 
$\Omega(n+1) = \left( 1 - a(x_k) \left(1-e^{-i \omega}\right) \right) \Omega(n)$
gelten muss.
Allerdings ist $\Omega$ nicht von $k$ abhängig und sobald es zwei Punkte $x,y \in \R$ mit $a(x) \neq a(y)$ gibt, folgt als einize sinnvolle Lösung $\Omega(n) = 0$.
Das Problem bei diesem Ansatz ist, dass wir nun zwar eine ortsabhängige Geschwindigkeit $a$ betrachten, $\Omega$ jedoch nicht entsrepchend angepasst haben.

\subsection*{Ortsabhängige Korrektur für $\Omega$}

Wir fordern daher nun zusätzlich, dass $\Omega(n,x)$ ortsabhängig und glatt in $x$ ist.
Der Ansatz lautet
\begin{align}
v^n_k = e^{i \omega k} \Omega(n, x_k) \bigl( u_0(t_n, x_k) + h u_1(t_n, x_k) \bigr) + O(h^2).
\end{align}

Da die Gleichung \eqref{wkb:zeitdiff} nur eine Differenz in der Zeit beschreibt, folgt für diesen Ansatz ganz analog
\begin{align}
\begin{split}
v^{n+1}_k - v^n_k &= e^{i \omega k} \Bigl( \bigl(\Omega(n+1, x_k) - \Omega(n,x_k) \bigr) u_0(t_n, x_k)\\
&+ h \bigl(\Omega(n+1,x_k) - \Omega(n,x_k) \bigr) u_1(t_n, x_k) + h \Omega(n+1, x_k) \partial_t u_0(t_n, x_k)\\
&+ h^2 \Omega(n+1, x_k) \bigl( \frac{1}{2} \partial^2_t u_0(t_n, x_k) + \partial_t u_1(t_n, x_k) \bigr) \Bigr) + O(h^3)
\end{split}
\end{align}
Wegen der neuen Regularität von $\Omega$ in der Ortskoordinate $x$ müssen wir Gleichung \eqref{wkb:ortdiff} neu berechnen.
Es gilt

\begin{align}\label{eq:wkb:adiff}
\begin{split}
&a(x_k) \bigl( v^n_k - v^n_{k-1} \bigr)\\
% &=
% a(x_k) \Bigl( e^{i \omega k} \Omega(n,x_k) \bigl( u_0(t_n, x_k) + h u_1(t_n, x_k) \bigr)
% - e^{i \omega (k-1)} \Omega(n,x_{k-1}) \bigl( u_0(t_n, x_{k-1}) + h u_1(t_n, x_{k-1}) \bigr) \Bigr)\\
% &=
% a(x_k) \Bigl( e^{i \omega k} \Omega(n,x_k) \bigl( u_0(t_n, x_k) + h u_1(t_n, x_k) \bigr)
% - e^{i \omega (k-1)} \Omega(n,x_{k-1}) \bigl( u_0(t_n, x_{k-1}) + h u_1(t_n, x_{k-1}) \bigr) \Bigr)\\
% &=
% a(x_k) \Bigl( e^{i \omega k} \Omega(n,x_k) \bigl( u_0(t_n, x_k) + h u_1(t_n, x_k) \bigr)\\
% &\quad - e^{i \omega (k-1)} \left( \Omega(n,x_k) - h \partial_x \Omega(n,x_k) + \frac {h^2}2 \partial^2_x \Omega(n,x_k) \right)
% \cdot \bigl( u_0(t_n, x_{k-1}) + h u_1(t_n, x_{k-1}) \bigr) \Bigr) + O(h^3)\\
% &=
% a(x_k) e^{i \omega k} \Bigl( \Omega(n,x_k) \bigl( u_0(t_n, x_k) + h u_1(t_n, x_k) \bigr) - e^{- i \omega} \left( \Omega(n,x_k) - h \partial_x \Omega(n,x_k) + \frac {h^2}2 \partial^2_x \Omega(n,x_k) \right)\\
% &\qquad \cdot \bigl( u_0(t_n, x_k) - h \partial_x u_0(t_n, x_k) + h^2 \frac{1}{2} \partial^2_x u_0(t_n, x_k) + h u_1(t_n, x_k) - h^2 \partial_x u_1(t_n, x_k) \bigr) \Bigr) + O(h^3)\\
&=
a(x_k) e^{i \omega k} \Bigl( \Omega(n,x_k) (1-e^{-i\omega}) u_0(t_n, x_k)\\
&\quad + h \Omega(n,x_k) (1 - e^{-i \omega}) u_1(t_n, x_k) + h e^{-i \omega} \bigl( \Omega(n,x_k) \partial_x u_0(t_n, x_k) + \partial_x \Omega(n,x_k)  u_0(t_n, x_k) \bigr)\\
&\quad + h^2 e^{-i\omega}  \bigl( \left( \Omega(n,x_k) \partial_x u_1(t_n, x_k) + \partial_x \Omega(n,x_k) u_1(t_n,x_k) \right) - \partial_x \Omega(n,x_k) \partial_x u_0(t_n, x_k) \bigr)\\
&\quad - h^2 \frac{1}{2} e^{-i\omega} \left( \Omega(n,x_k) \partial^2_x u_0(t_n, x_k) + \partial^2_x \Omega(n, x_k) u_0(t_n, x_k) \right)   \Bigr) + O(h^3)\\
% &\quad - e^{- i \omega} \left( \Omega(n,x_k) - h \partial_x \Omega(n,x_k) + \frac {h^2}2 \partial^2_x \Omega(n,x_k) \right)\\
% &\qquad \cdot \bigl( u_0(t_n, x_k) - h \partial_x u_0(t_n, x_k) + h^2 \frac{1}{2} \partial^2_x u_0(t_n, x_k)\\
% &\quad + h u_1(t_n, x_k) - h^2 \partial_x u_1(t_n, x_k) \bigr) \Bigr)\\
\end{split}
\end{align}

Betrachten wir beide Ergebnisse in der Summe \eqref{eq:varadv:verfahren}, so erhalten wir in $O(1)$ zunächst die von uns gewünschte Gleichung
\begin{align}
\Bigl( \Omega(n+1, x_k)  - \Omega(n,x_k) \bigl(1 - a(x_k) (1-e^{-i\omega}) \bigr) \Bigr) u_0(t_n, x_k) &= 0
\end{align}
aus welcher wir
\begin{align}
\Omega(n,x_k) &= \bigl(1 - a(x_k) (1-e^{-i\omega}) \bigr)^n, \qquad \text{und}\\
\begin{split}\label{eq:omegadiff}
\partial_x \Omega(n,x_k) &= n \bigl(1 - a(x_k) (1-e^{-i\omega}) \bigr)^{n-1} a'(x_k) (1-e^{-i\omega})\\
&= n a'(x_k) (1-e^{-i\omega}) \Omega(n-1, x_k)
\end{split}
\end{align}
ableiten.
Das führt in der Ordnung $O(h)$ zu dem Gleichungssystem
\begin{align}\label{eq:wkb:ohglgpre}
\partial_t u_0(t_n, x_k) + e^{-i \omega} \frac{\Omega(n,x_k)}{\Omega(n+1, x_k)} \partial_x u_0(t_n, x_k) &= - e^{-i \omega} \frac{\partial_x \Omega(n,x_k)}{\Omega(n+1, x_k)} u_0(t_n, x_k).
\end{align}
Auf der rechten Seite dieser Gleichung taucht nun der Term $- e^{-i \omega} \partial_x \Omega(n,x_k) u_0(t_n, x_k)$ auf.
In Gleichung \eqref{eq:omegadiff} haben wir $\partial_x \Omega(n,x_k) = n a'(x_k) (1-e^{-i\omega}) \Omega(n-1, x_k)$ gezeigt.
Darum wird aus Gleichung \eqref{eq:wkb:ohglgpre} 
\begin{align}\label{eq:wkb:ohpost}
\begin{split}
&\partial_t u_0(t_n, x_k) + \frac{e^{-i \omega}}{1 - a(x_k) (1-e^{-i\omega})} \partial_x u_0(t_n, x_k) =\\
&\qquad \qquad - \frac{e^{-i \omega} (1 - e^{-i\omega})}{(1 - a(x_k) (1-e^{-i\omega}))^2} \cdot n a'(x_k) \cdot u_0(t_n, x_k).
\end{split}
\end{align}

Weil aus $h \to 0$ auch $n \to \infty$ folgt, impliziert das Konvergenzprobleme für die linke Seite dieser Gleichung.
Bisher haben wir an dieser Stelle das Lemma \ref{lemma:transport:diskret:konvergenz_gitter} angewandt um von diskreten Gleichungen auf kontinuierliche Differentialgleichungen zu schlie"sen.
Lemma \ref{lemma:transport:diskret:konvergenz_gitter} macht in Gleichung \eqref{eq:lemma:diag:aussage} hier keine Aussage "uber die Konvergenz.
Ferner kann man dies hier durch eine Art Sublinear-Growth Bedingung ``korrigieren''.
Betrachte man $a(x) = 1 + O\left(\frac{1}{n} \right)$, zum Beispiel $a(x) = 1 + \frac{1}{n} b(x)$, so folgt $a'(x) = \frac{1}{n} b'(x)$ und somit
\begin{align}\label{eq:wkb:ohpost2}
\begin{split}
&\partial_t u_0(t_n, x_k) + \frac{e^{-i \omega}}{1 - a(x_k) (1-e^{-i\omega})} \partial_x u_0(t_n, x_k) =\\
&\qquad \qquad - \frac{e^{-i \omega} (1 - e^{-i\omega})}{(1 - a(x_k) (1-e^{-i\omega}))^2} \cdot b(x_k) \cdot u_0(t_n, x_k).
\end{split}
\end{align}

Das motiviert uns nun direkt in den Fall $a(x) = 1 + hb(x)$ überzugehen.

\section{Exponentialansatz f"ur kleine St"orungen}

Wir verwenden hier das Ergebnis aus Gleichung \eqref{eq:wkb:difflsg} aus dem Vorkapitel wieder und setzen $a(x) = 1 + hb(x)$ ein.
Es gilt
\begin{align}
\begin{split}
&v^{n+1}_k - v^n_k + \bigl(1 + hb(x_k)\bigr) \bigl( v^n_k - v^n_{k-1} \bigr)\\
&= e^{i \omega k} \Bigl(
\bigl(\Omega(n+1) - \Omega(n) \bigr) u_0(t_n, x_k) + \bigl(1 + hb(x_k)\bigr) \Omega(n) \left(1-e^{-i \omega}\right) u_0(t_n, x_k)\\
&\quad + h \bigl( \bigl(\Omega(n+1) - \Omega(n) \bigr) u_1(t_n, x_k) + \bigl(1 + hb(x_k)\bigr) \Omega(n) \left(1-e^{-i \omega}\right) u_1(t_n, x_k) \bigr)\\
&\quad + h \bigl( \Omega(n+1) \partial_t u_0(t_n, x_k) + \bigl(1 + hb(x_k)\bigr) \Omega(n) e^{-i \omega} \partial_x u_0(t_n, x_k) \bigr)\\
&\quad + h^2 \bigl( \Omega(n+1) \partial_t u_1(t_n, x_k) + \bigl(1 + hb(x_k)\bigr) \Omega(n) e^{-i \omega} \partial_x u_1(t_n, x_k)\\
&\qquad + \frac{1}{2} \left( \Omega(n+1) \partial^2_t u_0(t_n, x_k) - \bigl(1 + hb(x_k)\bigr) \Omega(n) e^{-i \omega} \partial^2_x u_0(t_n, x_k) \right) \bigr) 
\Bigr)\\&\quad + O(h^3)
\end{split}\\
\begin{split}\label{eq:wkb:letzteglg}
&= e^{i \omega k} \Bigl(
\bigl(\Omega(n+1) - \Omega(n) \bigr) u_0(t_n, x_k) + \Omega(n) \left(1-e^{-i \omega}\right) u_0(t_n, x_k)\\
&\quad + h \bigl( \bigl(\Omega(n+1) - \Omega(n) \bigr) u_1(t_n, x_k) + \Omega(n) \left(1-e^{-i \omega}\right) u_1(t_n, x_k) \bigr)\\
&\quad + h \bigl( \Omega(n+1) \partial_t u_0(t_n, x_k) + \Omega(n) e^{-i \omega} \partial_x u_0(t_n, x_k) \bigr)\\
&\quad + h b(x_k) \Omega(n) \left(1-e^{-i \omega}\right) u_0(t_n, x_k)\\
&\quad + h^2 \bigl( \Omega(n+1) \partial_t u_1(t_n, x_k) + \Omega(n) e^{-i \omega} \partial_x u_1(t_n, x_k)\\
&\qquad + \frac{1}{2} \left( \Omega(n+1) \partial^2_t u_0(t_n, x_k) - \Omega(n) e^{-i \omega} \partial^2_x u_0(t_n, x_k) \right) \bigr)\\
&\qquad + b(x_k) \Omega(n) e^{-i \omega} \partial_x u_0(t_n, x_k) + b(x_k) \Omega(n) \left(1-e^{-i \omega}\right) u_1(t_n, x_k)
\Bigr)\\&\quad + O(h^3)
\end{split}
\end{align}
In erster Ordnung gilt also
\begin{align}
\left( \Omega(n+1) - e^{-i \omega} \Omega(n)  \right) u_0(t_n, x_k) = 0
\end{align}
und daraus folgern wir $\Omega(n) = e^{- i \omega n}$ für alle $n \in \N$.
Das setzen wir in die Gleichung \eqref{eq:wkb:letzteglg} ein und klammern den
Faktor $e^{- \omega n}$ aus:
\begin{align}
\begin{split}
0 &= v^{n+1}_k - v^n_k + \bigl(1 + hb(x_k)\bigr) \bigl( v^n_k - v^n_{k-1} \bigr)\\
&= e^{i \omega (k - n)} \Bigl(
h \bigl( e^{-i \omega} \partial_t u_0(t_n, x_k) + e^{-i \omega} \partial_x u_0(t_n, x_k) + b(x_k) \left(1-e^{-i \omega}\right) u_0(t_n, x_k) \bigr)\\
&\quad + h^2 \bigl( e^{-i \omega} \partial_t u_1(t_n, x_k) + e^{-i \omega} \partial_x u_1(t_n, x_k) + b(x_k) e^{-i \omega} \partial_x u_0(t_n, x_k)\\
&\qquad + \frac{1}{2} \left( e^{-i \omega} \partial^2_t u_0(t_n, x_k) - e^{-i \omega} \partial^2_x u_0(t_n, x_k) \right)  + b(x_k) \left(1-e^{-i \omega}\right) u_1(t_n, x_k) \bigr) 
\Bigr) + O(h^3)
\end{split}
\end{align}
Das liefert uns nun mit $C := e^{i \omega} \left(1-e^{-i \omega}\right)$ das Gleichungssystem
\begin{align}\label{eq:wkb:oh11}
\partial_t u_0(t, x) + \partial_x u_0(t, x) &= - C b(x) u_0(t, x)\\
\begin{split}\label{eq:wkb:oh22}
\partial_t u_1(t, x) + \partial_x u_1(t, x) &= - b(x) \partial_x u_0(t, x) - \frac{1}{2} \left( \partial^2_t u_0(t, x) - \partial^2_x u_0(t, x) \right)\\
&\quad - C b(x) u_1(t, x)
\end{split}\\
u_0(0,x) &= V(x)\\
u_1(0,x) &= 0.
\end{align}
Aus der Gleichung \eqref{eq:wkb:oh11} folgern wir
\begin{align*}
\partial^2_t u_0(t,x) &= \partial_t \Bigl( \partial_t u_0(t,x) \Bigr)\\
&= \partial_t \Bigl( -C b(x) u_0(t,x) - \partial_x u_0(t,x) \Bigr)\\
&= - C b(x) \partial_t u_0(t,x) - \partial_x \Bigl( \partial_t u_0(t,x)  \Bigr)\\
&= - C b(x) \Bigl( -C b(x) u_0(t,x) - \partial_x u_0(t,x) \Bigr) - \partial_x \Bigl( -C b(x) u_0(t,x) - \partial_x u_0(t,x) \Bigr)\\
&= C^2 b^2(x) u_0(t,x) + C b(x) \partial_x u_0(t,x) + C \partial_x \Bigl( b(x) u_0(t,x) \Bigr) + \partial^2_x u_0(t,x)\\
&= C^2 b^2(x) u_0(t,x) + C \partial_x b(x) u_0(t,x) + \partial^2_x u_0(t,x).
\end{align*}
Dies setzen wir in \eqref{eq:wkb:oh22} ein und erhalten
\begin{align}
\begin{split}\label{eq:wkb:oh222}
\partial_t u_1(t, x) + \partial_x u_1(t, x) &= - C b(x) u_1(t, x) \\
&\quad - b(x) \partial_x u_0(t, x) - \left( \frac{C^2}{2} b^2(x) - \frac{C}{2} \partial_x b(x) \right) u_0(t,x)
\end{split}
\end{align}