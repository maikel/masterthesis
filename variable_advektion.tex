\section{Variable Advektion}
\label{sec:varadv}

Wir betrachen die Differentialgleichung der eindimensionalen, variablen Advektion
\begin{align}\label{eq:varadv:pde}
u_t(t,x) + a(x) u_x(t,x) = 0, \quad u(0,x) = u_0(x)
\end{align}
mit glattem und beschränktem $a\colon \R \to \R$.
Wir wenden für diese Differentialgleichung das Upwindverfahren mit dem Gitter $G(n,i) = (t_n, x_i) = h \cdot (\lambda n,i)$ an.
Dieses Gitter ist rechteckig, falls $\lambda \neq 1$ gilt und wir erhalten die Gleichung
\begin{align}\label{eq:varadv:verfahren}
v^{n+1}_i = v^n_i - a_i \lambda \left( v^n_i - v^n_{i-1} \right).
\end{align}

\subsection{Analytische Lösung}

F"ur die kommende Asymptotik könnte es sinnvoll sein, die echte Lösung von \eqref{eq:varadv:pde} zumindest qualitativ zu kennen.
Die Lösung lässt sich analytisch mithilfe der Methode der Charakteristiken bestimmen.
Aus \eqref{eq:varadv:pde} folgt nämlich, dass jede Lösung $u$ dieser Gleichung konstant auf der Kurve $(\tau, x(\tau))$ ist, für die
\begin{align} \label{eq:varadv:ode}
\dot x(\tau) = a(x(\tau))
\end{align} gilt.
Sei nun $\phi\colon \R^2 \to \R$ der Fluss der Differentialgleichung \eqref{eq:varadv:ode}.
Dann gilt für jedes $x_0 \in \R$ per defintionem, dass $\phi(\, \cdot \,,x_0)$ das Anfangswertproblem \begin{align}\label{eq:varadv:ivp} \dot x(\tau) = a(x(\tau)), \quad x(0) = x_0 \end{align} löst.
Insbesondere gilt $\phi_{t+s} = \phi_t \circ \phi_s$ und $\phi_0 = \id$.
F"ur jedes $x \in \R$ gibt es ein $x_0 \in \R$, so dass $u(t,x) = u_0(x_0)$ gilt.
Wegen $x_0 = \phi(0,x_0)$ und $x = \phi(t, x_0)$ folgt damit \[ x_0 = \phi(0, x_0) = \phi(t - t, x_0) = \phi(-t, \phi(t, x_0)) = \phi(-t, x) \] und somit auch \begin{align}\label{eq:varadv:analytic_solution} u(t,x) = u_0(x_0) = u_0(\phi(-t, x)).\end{align}

\begin{example}
Für $a(x) = x$ folgt wegen des Anfangswertproblems $\dot x(\tau) = x, x(0) = x_0$, dass $u(t,x)$ konstant auf den Kurven der Form $x(\tau) = x_0 e^{\tau}$ ist.
Der Fluss $\phi$ ist dann durch $\phi(t,x) = x e^t$ gegeben. Wegen \eqref{eq:varadv:analytic_solution} folgt dann \[ u(x,t) = u_0(\phi(-t,x)) = u_0(x e^{-t}) \] als Lösung für \eqref{eq:varadv:pde}.
\end{example}
\begin{example} \label{ex:varadv:abs}
Setze
\[
a_\eps(x) = \begin{cases} 
    1 + \eps (1 - \abs{x}) & \abs{x} \leq 1\\
    1 & \text{ sonst.}
\end{cases}
\]
Dann ist $a$ absolut stetig und es gilt $a' \in L^\infty(\R)$.
Folglich ist $a$ Lipschitz-stetig und $\eqref{eq:varadv:ode}$ besitzt einen eindeutigen Fluss, welcher sich stückweise zusammensetzt. (TODO)
\end{example}

\begin{remark}Finde glatte Beispiele, für die man einen Fluss hinschreiben und Numerik machen kann!\end{remark}

\begin{remark}Die Flüsse für $a(x) = e^{-x^2}$ oder einer Glättung $a(x) = (\varphi * f)(x)$ für einen Glättungskern $\varphi$ und einer beliebigen Funktion $f$ scheinen mir nicht trivial.\end{remark}

\begin{frage}Geht das überhaupt?\end{frage}

\begin{frage}Ist $\phi$ genau so regulär wie $a$? ODE basics, technisch, aber vielleicht könnte man dann das Beispiel \ref{ex:varadv:abs} überall durchrechnen und konkrete Modelle testen.\end{frage}

Wenn wir eine Lösung $v$ von \eqref{eq:varadv:pde} haben, so können wir die folgende partielle Differentialgleichung
\begin{align}\label{eq:varadv:inhompde}
u_t(t,x) + a(x) u_x(t,x) = v(t,x), \quad u(0,x) = 0
\end{align}
lösen, indem wir einfach $u(t,x) = t v(t,x)$ setzen. Denn dann gilt
\[ u_t(t,x) = v(t,x) + t v_t(t,x) = v(t,x) - t a(x) v_x(t,x) \]
und wegen $t a(x) v_x(t,x) = a(x) (t v(t,x))_x = a(x) u_x(t,x)$ folgt damit \eqref{eq:varadv:inhompde}.

\subsection{Oszillatorischer Ansatz}

Wir verwenden hier Methoden, die in ~\cite{Junk2004} vorgestellt wurden.
Ziel ist es, oszillatorische Fehler im Kurzzeitverhalten einzufangen, die von $a$ erzeugt werden.
Daher wählen wir den Ansatz
\begin{align}\label{eq:varadv:ansatz}
v^n_i = \u0(t_n, x_i) + h \u1(t_n, x_i) + h^2 \tu2(i) + O(h^2)
\end{align}
für unsere Gitterfunktion $v$. 
Der Term $\u2(t_n, x_i)$ taucht bei dem Ansatz nicht auf, weil er hierbei zu Null verschwindet.
Für die Differenz $v^{n+1}_i - v^n_i$ folgt dann
\begin{align*}
v^{n+1}_i - v^n_i
&= \u0(t_{n+1}, x_i) + h \u1(t_{n+1}, x_i) + h^2 \left( \u2(t_{n+1}, x_i) + \tu2(i) \right) + O(h^2)\\
   &\qquad - \left( \u0(t_n, x_i) + h \u1(t_n, x_i) + h^2 \left( \u2(t_n, x_i) + \tu2(i)\right)  + O(h^2)\right)\\
&= \u0(t_{n+1}, x_i) - \u0(t_n, x_i) + h \left( \u1(t_{n+1}, x_i) - \u1(t_n, x_i)\right) + \\
   &\qquad + h^2 \left( \u2(t_{n+1}, x_i) - \u2(t_n, x_i) \right)\\
&= \u0(t_n + \lambda h, x_i) - \u0(t_n, x_i) + h \left( \u1(t_n + \lambda h, x_i) - \u1(t_n, x_i)\right)\\
   &\qquad + h^2 \left( \u2(t_n + \lambda h, x_i) - \u2(t_n, x_i) \right)\\
&= \lambda h \, \u0_t (t_n, x_i) + \frac 12 \lambda^2 h^2 \u0_{tt}(t_n, x_i) + \lambda h^2 \u1_t(t_n, x_i) + O(h^3)\\
&= \lambda h \left( \u0_t (t_n, x_i) + h \left( \frac \lambda 2 \u0_{tt}(t_n, x_i) + \u1_t(t_n, x_i) \right) \right) + O(h^3)
\end{align*}
und für $\lambda (v^n_{i} - v^n_{i-1})$ folgt ganz analog
{\small\begin{align*}
\lambda (v^n_{i} - v^n_{i-1})
&= \lambda \left( \u0(t_n, x_i) + h \u1(t_n, x_i) + h^2 \left(\u2(t_n, x_i) + \tu2(i) \right) + O(h^2) \right.\\
   &\qquad -  \left. \left( \u0(t_n, x_{i-1}) + h \u1(t_n, x_{i-1}) + h^2 \left( \u2(t_n, x_{i-1}) + \tu 2(i-1)\right)  + O(h^2) \right) \right)\\
&= \lambda \left( \u0(t_n, x_i) - \u0(t_n, x_{i-1}) + h \left( \u1(t_n, x_i) - \u1(t_n, x_{i-1}) \right) \right.\\
   &\qquad + \left. h^2 \left( \u2(t_n, x_i) - \u2(t_n, x_{i-1}) + \bigl( \tu2(i) - \tu2(i-1) \bigr) \right) \right)\\
&= \lambda \left( \u0(t_n, x_i) - \u0(t_n, x_i-h) + h \left( \u1(t_n, x_i) - \u1(t_n, x_i-h) \right) \right. \\
   &\qquad \left. + h^2 \left( \u2(t_n, x_i) - \u2(t_n, x_i-h) + \tu2(i) - \tu2(i-1) \right) \right)\\
&= \lambda \left( h \u0_x(t_n, x_i) - \frac 12 h^2 \u0_{xx}(t_n, x_i) + h^2 \u1_x(t_n, x_i) + h^2\Bigl( \tu1(i) - \tu1(i-1) \Bigr) \right) + O(h^3) \\
&= \lambda h \left( \u0_x(t_n, x_i) + h \left( \u1_x(t_n, x_i) - \frac 12 \u0_{xx}(t_n, x_i) + \tu2(i) - \tu2(i-1) \right) \right) + O(h^3).
\end{align*}}
Wenn wir dies nun in das Verfahren \eqref{eq:varadv:verfahren} einsetzen, erhalten wir die Gleichung
\begin{align*}
&\u0_t (t_n, x_i) + h \left( \frac \lambda 2 \u0_{tt}(t_n, x_i) + \u1_t(t_n, x_i) \right)\\ 
&\qquad +  a_i \u0_x(t_n, x_i) + h a_i \left( \u1_x(t_n, x_i) - \frac 12 \u0_{xx}(t_n, x_i) + \tu2(i) - \tu2(i-1) \right) = O(h^2).
\end{align*}
Nachdem wir nach den Ordnungen von $h$ sortiert haben, führt uns das zu dem Gleichungssystem
\begin{align*}
\u0_t(t_n, x_i) + a(x_i) \u0_x(t_n, x_i) &= 0,\\
\u1_t(t_n, x_i) + a(x_i) \u1_x(t_n, x_i) &= \frac {1 - a(x_i)\lambda}{2} \u0_{xx}(t_n, x_i) - \frac {a_x(x_i) \lambda} 2 \u0_x(t_n, x_i)\\
                                         & \quad + \tu2(i) - \tu2(i-1)
\end{align*}
mit den Anfangsbedingungen $\u0(0, x_i) = u_0(x_i)$