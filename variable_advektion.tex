%!TEX root=main.tex

Wir betrachten weiterhin das Upwind-Verfahren. Diesmal lassen wir jedoch eine
variable Geschwindigkeit zu und untersuchen insbesonders wie sich diese auf
hochfrequente Schwingungen auswirkt. Wie schon im Vorkapitel machen wir uns die
Linearität der Gleichungen und die Sätze \ref{satz:alt:beidenull} und
\ref{satz:glgregulaer} zu nutze und rechnen den hochfrequenten und den glatten
Anteil getrennt von einander aus.

Es sei also für alle das folgende Dirichlet Problem gegeben:
\begin{align}\label{eq:varadv:pde}
\begin{split}
\partial_t u(t, x) + a(x) \partial_x u(t, x) &= 0 \qquad \forall (t, x) \in \Rp \times \R\\
u(0, x) &= U(x),
\end{split}
\end{align}
wobei $a \in C^\infty(\R)$ und $U \in L^1(\R)$ gelten.

Wir betrachten weiterhin das Gitter $G_h = (h \lambda n, h i)$ und dazugehörige
Upwindverfahren
\begin{align}\label{eq:varadv:verfahren}
\begin{split}
v^{n+1}_i - v^n_i + a(x_i) \lambda \bigl( v^n_i - v^n_{i-1} \bigr) &= 0.\\
v^0_i &= U(x_i).
\end{split}
\end{align}

\section{Regulärer Ansatz}

\subsection*{Das zugrundeliegende Problem}

Angenommen es gibt glatte Abbildungen $u_0, u_1 \in \Cinf$ mit
\begin{align}\label{eq:varadv:reg:ansatz}
\begin{split}
v^n_i = u_0(t_n, x_i) + h u_1(t_n, x_i) + O(h^2).
\end{split}
\end{align}
Wir setzen das in \eqref{eq:varadv:verfahren} ein und erhalten die Gleichung
\begin{align}
\begin{split}
v^{n+1}_i - v^n_i &= u_0(t_{n+1}, x_i) - u_0(t_n, x_i) + h \bigl( u_1(t_{n+1}, x_i) - u_1(t_n, x_i) \bigr)\\
&= \lambda h \partial_t u_0(t_n, x_i) + h^2 \frac{\lambda^2}{2} \partial^2_t u_0(t_n, x_i)\\
&\quad + h^2 \lambda \partial_t u_1(t_n, x_i) + O(h^2).
\end{split}
\end{align}
und
\begin{align}
\begin{split}
a(x_i) \lambda \bigl( v^n_i - v^n_{i-1} \bigr) &= a(x_i) \lambda \Bigl( u_0(t_n, x_i) - u(t_n, x_{i-1}) + h \bigl( u_1(t_n, x_i) - u_1(t_n, x_{i-1}) \bigr) \Bigr)\\
&= a(x_i) \lambda \Bigl( h \partial_x u(t_n, x_i) + h^2 \frac{1}{2} \partial^2_x u(t_n, x_i) + h^2 \partial_x u_1(t_n, x_i) \Bigr) + O(h^2).
\end{split}
\end{align}
In der Summe ergibt das
\begin{align}
\begin{split}
0 &= v^{n+1}_i - v^n_i + a(x_i) \lambda \bigl( v^n_i - v^n_{i-1} \bigr)\\
&= \lambda h \partial_t u_0(t_n, x_i) + h^2 \frac{\lambda^2}{2} \partial^2_t u_0(t_n, x_i) + h^2 \lambda \partial_t u_1(t_n, x_i)\\
&\quad + h a(x_i) \lambda \partial_x u(t_n, x_i) - h^2 \frac{a(x_i) \lambda}{2} \partial^2_x u(t_n, x_i) + h^2 a(x_i) \lambda \partial_x u_1(t_n, x_i) + O(h^2).\\
\Leftrightarrow \quad 0 &= h \partial_t u_0(t_n, x_i) + h^2 \frac{\lambda}{2} \partial^2_t u_0(t_n, x_i) + h^2 \partial_t u_1(t_n, x_i)\\
&\quad + h a(x_i) \partial_x u(t_n, x_i) - h^2 \frac{a(x_i)}{2} \partial^2_x u(t_n, x_i) + h^2 a(x_i) \partial_x u_1(t_n, x_i) + O(h^2).
\end{split}
\end{align}
Sortiert nach den Ordnungen von $h$ und aufgrund der Stetigkeit der Abbildungen
$u_0, u_1$ und $a$ erhalten wir das Gleichungsystem für alle $(t,x) \in \Rp
\times \R$
\begin{align}
\label{eq:varadv:reg:u0}
\partial_t u_0(t,x) + a(x) \partial_x u_0(t, x) &= 0,\\
\label{eq:varadv:reg:u1}
\partial_t u_1(t,x) + a(x) \partial_x u_1(t, x) &= \frac{a(x)}{2} \partial^2_x u_0(t,x) - \frac{\lambda}{2} \partial^2_t u_0(t,x),\\
u_0(0,x) &= U(x) \quad \text{und}\\
u_1(0,x) &= 0.
\end{align}
Die Abbildung $u_0$ ist dann also eine Lösung der Differentialgleichung
\eqref{eq:varadv:pde} und eine Lösung zu $u_1$ gibt uns den Fehler des Verfahren
zur ersten Ordnung.
Die Lösung des Anfangswertproblems für $u_0$ lautet 
\begin{align}
\label{eq:varadv:reg:u0_loesung}
u_0(t,x) = U( \phi_a(-t, x) ).
\end{align}
Wobei $\phi_a \in C^\infty \left( \R \times \R \right)$ der Fluss der Differentialgleichung $\dot y = a(y)$ ist.
Aus Gleichung \eqref{eq:varadv:reg:u0} entnehmen wir
\begin{align}
\begin{split}
\partial^2_t u_0(t,x) &= \partial_t \bigl( \partial_t u_0(t,x) \bigr)\\
&= \partial_t \bigl( - a(x) \partial_x u_0(t, x) \bigr)\\
&= - a(x) \partial_x \bigl( \partial_t u_0(t, x) \bigr)\\
&= a(x) \partial_x \bigl( a(x) \partial_x u_0(t, x) \bigr)\\
&= a(x) a'(x) \partial_x u_0(t,x) + a^2(x) \partial^2_x u_0(t,x)
\end{split}
\end{align}
und setzen dies in Gleichung \eqref{eq:varadv:reg:u1} ein:
\begin{align}\label{eq:varadv:reg:u1_neu}
\begin{split}
\partial_t u_1(t,x) + a(x) \partial_x u_1(t, x) &= \frac{a(x)}{2} \partial^2_x u_0(t,x) - \frac{\lambda}{2} \partial^2_t u_0(t,x)\\
&= \frac{a(x)(1 - a(x) \lambda)}{2} \partial^2_x u_0(t,x) - a(x) a'(x) \lambda \partial_x u_0(t,x).
\end{split}
\end{align}
Und ganz analog wie im regulären Fall des Unterkapitels \ref{sec:regulaer} kann
man für den Fall $a(x)\lambda < 1$ ein langsame Zeitvariable $\tau = ht$
einführen und löst eine Diffusionsgleichung um die Quellterme in der Gleichung
\eqref{eq:varadv:reg:u1_neu} zu eliminieren.
In dem Fall, dass in einer offenen Umgebung $U$ die Ungleichung $a(x)\lambda > 1$
gilt, schreiben wir für $u_1$ die Lösung
\begin{align}
\begin{split}
u_1(t,x) &= t \left( \frac{a(x)(1 - a(x) \lambda)}{2} \partial^2_x U( \phi_a(-t, x) ) - a(x) a'(x) \lambda \partial_x U( \phi_a(-t, x) )\right)\\
&= t a(x) a'(x) \left( \frac{1 - a(x) \lambda}{2} U''( \phi_a(-t, x) ) - a(x) \lambda U'( \phi_a(-t, x) )\right).
\end{split}
\end{align}

\subsection*{Kleine Störungen}

Wir schreiben für $a$ nun $a(x) = \frac{1 + h A(x)}{\lambda}$ und $A(x) > 0$ in
einer Umgebung $U$. D.\,.h. wir untersuchen, wie sich das Verfahren für den Fall
$a \lambda \sim 1$ für $h \to 0$ verhält.