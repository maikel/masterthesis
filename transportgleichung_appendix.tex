\section{Die Transportgleichung}
\subsection{Beispielskript}
\label{appendix:transport:beispiel}
\lstinputlisting{octave/transport_beispiel.m}

\subsection{Diskreter Ansatz: Summanden vom Upwind-Schema entwickeln}
\label{appendix:diskret:summanden}

Wir rechnen hier nun die einzelnen Summanden von \eqref{eq:adv:scheme_rechnung} aus:
{
\begin{align} \label{eq:transport:diskret:diff1}
v^{n+1}_i - v^n_i =
\begin{split}
& u_0(n+1, i, t_n, x_i) - u_0(n, i, t_n, x_i)\\
&\lambda h \partial_t u_0(n+1, i, t_n, x_i) + \frac {(\lambda h)^2}{2} \partial^2_t u_0(n+1, i, t_n, x_i)\\
&+ \frac {(\lambda h)^3}{6} \partial^3_t u_0(n+1, i, t_n, x_i)\\
&+ h \bigl( u_1(n+1, i, t_n, x_i) - u_1(n, i, t_n, x_i) \bigr)\\
&+ \lambda h^2 \partial_t u_1(n+1, i, t_n, x_i) + \frac {\lambda^2 h^3}{2} \partial^2_t u_1(n+1, i, t_n, x_i)\\
&+ h^2 \bigl( u_2(n+1, i, t_n, x_i) - u_2(n, i, t_n, x_i) \bigr)\\
&+ \lambda h^3 \partial_t u_2(n+1, i, t_n, x_i) + o(h^3).
\end{split}
\end{align}
} und
{ 
\begin{align} \label{eq:transport:diskret:diff2}
\lambda (v^n_i - v^n_{i-1}) =
\begin{split}
& \lambda( u_0(n, i, t_n, x_i) - u_0(n, i-1, t_n, x_i) )\\
&\lambda h \partial_x u_0(n, i-1, t_n, x_i) - \frac {\lambda h^2}{2} \partial^2_x u_0(n, i-1, t_n, x_i)\\
& + \frac {\lambda h^3}{6} \partial^3_x u_0(n, i-1, t_n, x_i)\\
&+ \lambda h \bigl( u_1(n, i, t_n, x_i) - u_1(n, i-1, t_n, x_i) \bigr)\\
&+ \lambda h^2 \partial_x u_1(n, i-1, t_n, x_i) - \frac {\lambda h^3}{2} \partial^2_x u_1(n, i-1, t_n, x_i)\\
&+ \lambda h^2 \bigl( u_2(n, i, t_n, x_i) - u_2(n, i-1, t_n, x_i) \bigr)\\
&+ \lambda h^3 \partial_t u_2(n, i-1, t_n, x_i) + o(h^2).
\end{split}
\end{align}
}

\subsection{Oszillatorischer Produktansatz: Sortiere nach Frequenz}
\label{appendix:osz:sortiere_nach_frequenz}

Es gilt
\[ u_k(n, i, t_n, x_i) = w_k(t_n, x_i) + (-1)^i (1 - 2\lambda)^n z_k(t_n, x_i) \quad \text{für $k = 0,1,2$} \]
und aus \eqref{eq:transport:diskret:oh} folgt 

\vspace{0.4cm}
\noindent \textbf{In der Ordnung $O(h)$:}
\begin{align*}
0 &= \partial_t u_0(n+1, i, t_n, x_i) + \partial_x u_0(n, i-1, t_n, x_i)\\
&= \partial_t w_0(t_n, x_i) + (-1)^i (1 - 2\lambda)^{n+1} \partial_t z_0(t_n, x_i)\\
&\qquad + \partial_x w_0(t_n, x_i) + (-1)^{i-1} (1 - 2\lambda)^n \partial_x z_0(t_n, x_i)\\
&= \partial_t w_0(t_n, x_i) + \partial_x w_0(t_n, x_i)\\
&\qquad + (-1)^i (1 - 2\lambda)^n \bigl( (1 - 2\lambda) \partial_t z_0(t_n, x_i) - \partial_x z_0(t_n, x_i) \bigr) \\
\end{align*}

also
\begin{align*}
\partial_t w_0(t_n, x_i) + \partial_x w_0(t_n, x_i) &= 0 \qquad \text{und}\\
\partial_t z_0(t_n, x_i) + \frac {1}{2\lambda - 1} \partial_x z_0(t_n, x_i) &= 0.
\end{align*}

Hieraus folgt insbesondere auch $\partial^2_t w_0 = \partial^2_x w_0$ und $\partial^2_t z_0 = \frac {1}{(2\lambda - 1)^2} \partial^2_x z_0$.

\vspace{0.4cm}
\noindent \textbf{In der Ordnung $O(h^2)$:}
\begin{align*}
\partial_t u_1(n+1, i, t_n, x_i) + \partial_x u_1(n, i-1, t_n, x_i) &=
&\frac {1}{2} \partial^2_x u_0(n, i-1, t_n, x_i) - \frac{\lambda}{2} \partial^2_t u_0(n+1, i, t_n, x_i)
\end{align*}
Es gilt 
\begin{align*}
\frac {1}{2} \partial^2_x u_0(n, i-1, t_n, x_i) \qquad\\
- \frac{\lambda}{2} \partial^2_t u_0(n+1, i, t_n, x_i)
&= \frac{1}{2} \bigl( \partial^2_x w_0(t_n, x_i) + (-1)^{i-1} (1 - 2\lambda)^n \partial^2_x z_0(t_n, x_i) \bigr) \\
&\qquad - \frac{\lambda}{2} \bigl( \partial^2_t w_0(t_n, x_i) + (-1)^i (1 - 2\lambda)^{n+1} \partial^2_t z_0(t_n, x_i) \bigr)\\
&= \frac{1 - \lambda}{2}  \partial^2_x w_0(t_n, x_i)\\
&\qquad + (-1)^i (1 - 2\lambda)^{n} \left( - \frac{1}{2} \partial^2_x z_0(t_n, x_i) - \frac{\lambda (1 - 2\lambda)}{2} \partial^2_t z_0(t_n, x_i) \right)\\
&= \frac{1 - \lambda}{2}  \partial^2_x w_0(t_n, x_i)\\
&\qquad + (-1)^i (1 - 2\lambda)^{n} \left( - \frac{1}{2} \partial^2_x z_0(t_n, x_i) - \frac{\lambda (1 - 2\lambda)}{2 (2 \lambda - 1)^2} \partial^2_x z_0(t_n, x_i) \right)\\
&= \frac{1 - \lambda}{2}  \partial^2_x w_0(t_n, x_i)\\
&\qquad + (-1)^i (1 - 2\lambda)^{n} \left( - \frac{1}{2} \partial^2_x z_0(t_n, x_i) + \frac{\lambda}{2 (2 \lambda - 1)} \partial^2_x z_0(t_n, x_i) \right)\\
&= \frac{1 - \lambda}{2}  \partial^2_x w_0(t_n, x_i)\\
&\qquad + (-1)^i (1 - 2\lambda)^{n} \left( \frac{1 - \lambda}{2 (2 \lambda - 1)} \partial^2_x z_0(t_n, x_i) \right)
\end{align*}
und ganz analog wie bei $O(h)$ gilt ferner
\begin{align*}
\partial_t u_1(n+1, i, t_n, x_i) \qquad\\
+ \: \partial_x u_1(n, i-1, t_n, x_i) &= \partial_t w_1(t_n, x_i) + \partial_x w_1(t_n, x_i)\\
&\qquad + (-1)^i (1 - 2\lambda)^n \bigl( (1 - 2\lambda) \partial_t z_1(t_n, x_i) - \partial_x z_1(t_n, x_i) \bigr) \\
\end{align*}
und hieraus folgt für alle $(t_n, x_i)$
\begin{align*}
\partial_t w_1(t_n, x_i) + \partial_x w_1(t_n, x_i) &= \frac{1 - \lambda}{2}  \partial^2_x w_0(t_n, x_i)\\
\partial_t z_1(t_n, x_i) + \frac{1}{2 \lambda - 1} \partial_x z_1(t_n, x_i) &= \frac{\lambda - 1}{2 (2 \lambda - 1)^2} \partial^2_x z_0(t_n, x_i)
\end{align*}