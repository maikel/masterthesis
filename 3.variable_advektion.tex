%!TEX root=thesis.tex

Wir betrachten weiterhin das Upwind-Verfahren.
In diesem Kapitel lassen wir jedoch eine variable Geschwindigkeit zu und untersuchen insbesondere wie sich diese auf hochfrequente Schwingungen auswirkt.
Wie schon im Vorkapitel machen wir uns die Linearität der Gleichungen und die Sätze \ref{satz:alt:beidenull} und \ref{satz:glgregulaer} zu nutze und rechnen den hochfrequenten und den glatten Anteil getrennt von einander aus.

Es sei also das folgende Dirichlet Problem gegeben:
\begin{align}\label{eq:varadv:pde}
\begin{split}
\partial_t u(t, x) + a(x) \partial_x u(t, x) &= 0 \qquad \forall (t, x) \in \Rp \times \R\\
u(0, x) &= U(x),
\end{split}
\end{align}
wobei $a \in C^\infty(\R)$ und $U \in L^1(\R)$ gelten.

Im Gegensatz zum vorherigen Kapitel wählen wir hier ein quadratisches Gitter $G_h = h (n, i)$ und das dazugehörige Upwind-Verfahren
\begin{align}\label{eq:varadv:verfahren}
\begin{split}
v^{n+1}_i - v^n_i + a(x_i) \bigl( v^n_i - v^n_{i-1} \bigr) &= 0.\\
v^0_i &= U(x_i).
\end{split}
\end{align}
Diese Annahme vereinfacht viele der kommenden Rechnungen.

\section{Regulärer Ansatz}

%!TEX root=thesis.tex

\subsection*{Das zugrundeliegende Problem}

Angenommen es gibt glatte Abbildungen $u_0, u_1 \in \Cinf$ mit
\begin{align}\label{eq:varadv:reg:ansatz}
\begin{split}
v^n_i = u_0(t_n, x_i) + h u_1(t_n, x_i) + O(h^2).
\end{split}
\end{align}
Wir setzen das in Gleichung \eqref{eq:varadv:verfahren} ein und erhalten die Gleichung
\begin{align}
\begin{split}
v^{n+1}_i - v^n_i &= u_0(t_{n+1}, x_i) - u_0(t_n, x_i) + h \bigl( u_1(t_{n+1}, x_i) - u_1(t_n, x_i) \bigr)\\
&= h \partial_t u_0(t_n, x_i) + h^2 \frac{1}{2} \partial^2_t u_0(t_n, x_i) + h^2  \partial_t u_1(t_n, x_i) + O(h^3).
\end{split}
\end{align}
und
\begin{align}
\begin{split}
a(x_i) \bigl( v^n_i - v^n_{i-1} \bigr) &= a(x_i) \Bigl( u_0(t_n, x_i) - u(t_n, x_{i-1}) + h \bigl( u_1(t_n, x_i) - u_1(t_n, x_{i-1}) \bigr) \Bigr)\\
&= a(x_i)  \Bigl( h \partial_x u(t_n, x_i) - h^2 \frac{1}{2} \partial^2_x u(t_n, x_i) + h^2 \partial_x u_1(t_n, x_i) \Bigr) + O(h^3).
\end{split}
\end{align}
In der Summe ergibt das
\begin{align}
\begin{split}
0 &= v^{n+1}_i - v^n_i + a(x_i)  \bigl( v^n_i - v^n_{i-1} \bigr)\\
&=  h \partial_t u_0(t_n, x_i) + h^2 \frac{1}{2} \partial^2_t u_0(t_n, x_i) + h^2  \partial_t u_1(t_n, x_i)\\
&\quad + h a(x_i)  \partial_x u(t_n, x_i) - h^2 \frac{a(x_i) }{2} \partial^2_x u(t_n, x_i) + h^2 a(x_i)  \partial_x u_1(t_n, x_i) + O(h^3).\\
\Leftrightarrow \quad 0 &= h \partial_t u_0(t_n, x_i) + h^2 \frac{1}{2} \partial^2_t u_0(t_n, x_i) + h^2 \partial_t u_1(t_n, x_i)\\
&\quad + h a(x_i) \partial_x u(t_n, x_i) - h^2 \frac{a(x_i)}{2} \partial^2_x u(t_n, x_i) + h^2 a(x_i) \partial_x u_1(t_n, x_i) + O(h^3).
\end{split}
\end{align}
Sortiert nach den Ordnungen von $h$ und aufgrund der Stetigkeit der Abbildungen
$u_0, u_1$ und $a$ erhalten wir das Gleichungsystem für alle $(t,x) \in \Rp
\times \R$
\begin{align}
\label{eq:varadv:reg:u0}
\partial_t u_0(t,x) + a(x) \partial_x u_0(t, x) &= 0,\\
\label{eq:varadv:reg:u1}
\partial_t u_1(t,x) + a(x) \partial_x u_1(t, x) &= \frac{a(x)}{2} \partial^2_x u_0(t,x) - \frac{1}{2} \partial^2_t u_0(t,x),\\
u_0(0,x) &= U(x) \quad \text{und}\\
u_1(0,x) &= 0.
\end{align}
Die Abbildung $u_0$ ist dann also eine Lösung der Differentialgleichung
\eqref{eq:varadv:pde} und eine Lösung zu $u_1$ gibt uns den Fehler des Verfahren
zur ersten Ordnung.
Im Anhang \ref{sec:analyt} haben wir die analytische L"osung
des Anfangswertproblems für $u_0$ qualitativ untersucht und sie lautet
\begin{align}
\label{eq:varadv:reg:u0_loesung}
u_0(t,x) = U( \phi(-t, x) ).
\end{align}
Wobei $\phi \in C^\infty \left( \R \times \R \right)$ der Fluss der Differentialgleichung $\dot y = a(y)$ ist.
Aus Gleichung \eqref{eq:varadv:reg:u0} entnehmen wir
\begin{align}\label{eq:varadv:kleina:2teabl}
\begin{split}
\partial^2_t u_0(t,x) &= \partial_t \bigl( \partial_t u_0(t,x) \bigr)\\
&= \partial_t \bigl( - a(x) \partial_x u_0(t, x) \bigr)\\
&= - a(x) \partial_x \bigl( \partial_t u_0(t, x) \bigr)\\
&= a(x) \partial_x \bigl( a(x) \partial_x u_0(t, x) \bigr)\\
&= a(x) a'(x) \partial_x u_0(t,x) + a^2(x) \partial^2_x u_0(t,x)
\end{split}
\end{align}
und setzen dies in Gleichung \eqref{eq:varadv:reg:u1} ein:
\begin{align}\label{eq:varadv:reg:u1_neu}
\begin{split}
\partial_t u_1(t,x) + a(x) \partial_x u_1(t, x) &= \frac{a(x)}{2} \partial^2_x u_0(t,x) - \frac{1}{2} \partial^2_t u_0(t,x)\\
&= \frac{a(x)(1 - a(x) )}{2} \partial^2_x u_0(t,x) - a(x) a'(x)  \partial_x u_0(t,x).
\end{split}
\end{align}
Und ganz analog wie im regulären Fall des Unterkapitels \ref{sec:regulaer} kann man für den Fall, dass $0 < a(x) < 1$ für alle $x \in \R$ gilt, ein langsame Zeitvariable $\tau = ht$ einführen und löst eine Advektions-Diffusionsgleichung um die Quellterme in der Gleichung \eqref{eq:varadv:reg:u1_neu} zu eliminieren.
Für den Fall, dass $a(x) > 1$ in einer offenen Umgebung $U$ gilt, ist diese Gleichung jedoch nicht gut gestellt.
In diesem Fall schreiben wir für $u_1$ die Lösung
\begin{align}
\begin{split}
u_1(t,x) &= t \left( \frac{a(x)(1 - a(x) )}{2} \partial^2_x U( \phi(-t, x) ) - a(x) a'(x)  \partial_x U( \phi(-t, x) )\right)\\
&= t a(x) a'(x) \left( \frac{1 - a(x) }{2} U''( \phi(-t, x) ) - a(x)  U'( \phi(-t, x) )\right),
\end{split}
\end{align}
welche nur für kurze Zeiten $t$ gelten kann.

\subsection*{Kleine Störungen}

Wir schreiben für $a$ nun $a(x) = 1 + h b(x)$ für eine beschränkte Abbildung $b(x) > 0$ in einer Umgebung $U$.
D.\,h. wir untersuchen, wie sich das Verfahren für den Fall $a  \sim 1$ für $h \to 0$ verhält.

Dabei ``schieben'' wir wie schon im Fall kleiner Störungen der CFL Zahl in Kapitel \ref{sec:transport:kleineta} die ``störenden'' Anteile eine Ordnung weiter.
Daher wählen wir unseren Ansatz ebenfalls so, dass wir eine Ordnung in $h$ mehr betrachten.
Es seien also glatte Abbildungen $u_i \in \Cinf$, $i = 0,1,2$ mit
\begin{align}
\label{eq:varadv:kleina:ansatz}
v^n_i = u_0(t_n, x_i) + h u_1(t_n, x_i) + h^2 u_2(t_n, x_i) + O(h^3).
\end{align}
gegeben.
Dann gilt
\begin{align}
\label{eq:varadv:kleina:diff1}
\begin{split}
v^{n+1}_i - v^n_i &= u_0(t_{n+1}, x_i) - u_0(t_n, x_i) + h \bigl( u_1(t_{n+1}, x_i) - u_1(t_n, x_i) \bigr)\\
&\quad + h^2 \bigl( u_2(t_{n+1}, x_i) - u_2(t_{n+1}, x_i) \bigr)\\
&=  h \partial_t u_0(t_n, x_i) + h^2 \frac{1}{2} \partial^2_t u_0(t_n, x_i) + h^3 \frac{1}{6} \partial^3_t u_0(t_n, x_i)\\
&\quad + h^2  \partial_t u_1(t_n, x_i) + h^3 \frac{1}{2} \partial^2_t u_1(t_n, x_i)\\
&\quad + h^3  \partial_t u_2(t_n, x_i) + O(h^4)
\end{split}
\end{align}
und
\begin{align}
\label{eq:varadv:kleina:diff2}
\begin{split}
\bigl( 1 + h b(x_i) \bigr)  \bigl( v^n_i - v^n_{i-1} \bigr) &= \bigl(1 + h b(x_i) \bigr) \Bigl( u_0(t_n, x_i) - u_0(t_n, x_{i-1})\\
&\quad + h \bigl( u_1(t_n, x_i) - u_1(t_n, x_{i-1}) \bigr)\\
&\quad + h^2 \bigl( u_2(t_n, x_i) - u_2(t_n, x_{i-1}) \bigr) \Bigr)\\
&= \bigl(1 + h b(x_i) \bigr) \cdot \\
&\Bigl( h \partial_x u_0(t_n, x_i) - h^2 \frac{1}{2} \partial^2_x u_0(t_n, x_i) + h^3 \frac{1}{6} \partial^3_x u_0(t_n, x_i)\\
&\quad + h^2 \partial_x u_1(t_n, x_i) - h^3 \frac{1}{2} \partial_x u_1(t_n, x_i)\\
&\quad + h^3 \partial_x u_2(t_n, x_i) \Bigr) + O(h^4).
\end{split}
\end{align}
% Man beachte, dass der Faktor $\bigl(1 + h b(x_i)\bigr)$ jeden Term in der Differenz $v^n_i - v^n_{i-1}$ auch eine Ordnung höher auftauchen lässt.
Das liefert uns also das Gleichungssystem
\begin{align}\label{eq:varadv:kleina:oh}
\partial_t u_0(t, x) + \partial_x u_0(t, x) &= 0\\
\label{eq:varadv:kleina:oh2}
\partial_t u_1(t, x) + \partial_x u_1(t, x) &= \frac{1}{2} \bigl( \underbrace{\partial^2_x u_0(t,x) - \partial^2_t u_0(t,x)}_{= 0} \bigr) - b(x) \partial_x u_0(t, x)\\
\label{eq:varadv:kleina:oh3}
\begin{split}
\partial_t u_2(t, x) + \partial_x u_2(t, x) &= -\frac{b(x)}{2} \partial^2_x u_0(t,x) - b(x) \partial_x u_1(t,x)\\
&\quad - \frac{1}{6} \bigl( \underbrace{\partial^3_t u_0(t,x) + \partial^3_x u_0(t,x)}_{= 0} \bigr) + \frac{1}{2} \bigl( \partial^2_x u_1(t,x) - \partial^2_t u_1(t,x) \bigr).
\end{split}\\
u_0(0,x) &= U(x)\\
u_1(0,x) &= u_2(0, x) = 0 
\end{align}
Aus \eqref{eq:varadv:kleina:oh2} folgern wir
\begin{align}
\begin{split}
\partial^2_t u_1(t,x) &= \partial_t \bigl( - b(x) \partial_x u_0(t,x) - \partial_x u_1(t,x) \bigr)\\
&= b(x) \partial^2_x u_0(t,x) + \partial_x \bigl( - b(x) \partial_x u_0(t,x) + \partial_x u_1(t,x) \bigr)\\
&= b(x) \partial^2_x u_0(t,x) - b(x) \partial^2_x u_0(t,x) - b'(x) \partial_x u_0(t,x) + \partial^2_x u_1(t,x)\\
&= - b'(x) \partial_x u_0(t,x) + \partial^2_x u_1(t,x)
\end{split}
\end{align}
und so wird aus der Gleichung \eqref{eq:varadv:kleina:oh3} die Gleichung
\begin{align}
\partial_t u_2(t, x) + \partial_x u_2(t, x) &= \frac{1}{2} \left( b(x) \partial^2_x u_0(t,x) + b'(x) \partial_x u_0(t,x) \right) - b(x) \partial_x u_1(t,x).
\end{align}
Um in Gleichung \eqref{eq:varadv:kleina:oh} den Quellterm zu eliminieren, führen wir wie bisher die langsame Zeitvariable $\tau = h t$ ein
und fordern
\begin{align}
\partial_\tau u_k(t, \tau, x) + b(x) \partial_x u_k(t, \tau, x) &= 0 \qquad \text{für $k=0,1$.}
\end{align}
Damit folgt dann wegen $u_1(0, 0, x) = 0$ auch $u_1(t, \tau, x) = 0$ für alle $(t, \tau, x) \in \Rp \times \R$.
Dies liefert uns insgesamt das Gleichungssystem
\begin{align}
\label{eq:3_1:u01}
\partial_t u_0(t, \tau, x) + \partial_x u_0(t, \tau, x) &= 0\\
\label{eq:3_1:u02}
\partial_\tau u_0(t, \tau, x) + b(x) \partial_x u_0(t, \tau, x) &= 0\\
\partial_t u_1(t, \tau, x) + \partial_x u_1(t, \tau, x) &= 0\\
\partial_\tau u_1(t, \tau, x) + b(x) \partial_x u_1(t, \tau, x) &= 0\\
\label{eq:3_1.regulaer:quellterm_2}
\partial_t u_2(t, \tau, x) + \partial_x u_2(t, \tau, x) &= \frac{b(x)}{2} \partial^2_x u_0(t, \tau,x)\\
u_0(0,0,x) &= U(x)\\
u_1(0,0,x) &= u_2(0,0,x) = 0.
\end{align}
Aus der Gleichung \eqref{eq:3_1:u01} folgt, dass es eine Abbildung $A(\tau,y)$ mit
\begin{align}
u_0(t,\tau,x) &= A(\tau,x-t) \qquad \text{und}\\
A(0,y) &= U(y)
\end{align}
gibt.
Wegen der Gleichung \eqref{eq:3_1:u02} gilt f"ur $A$ nun zus"atzlich die Gleichung
\begin{align}
\partial_\tau A(\tau, y) + b(x) \partial_x A(\tau, y) &= 0
\end{align}
welche durch
\begin{align}
A(\tau, y) = U(\Phi(-\tau, y))
\end{align}
gel"ost wird, wobei $\Phi \in C^\infty\left(\Rp \times \R \right)$ der Fluss der Differentialgleichung $\dot x = b(x)$ ist.
Insgesamt folgt damit
\begin{align}
u_0(t,\tau,x) = U(\Phi(-\tau, x- t)).
\end{align}
Die Gleichung \eqref{eq:3_1.regulaer:quellterm_2} ist der Gleichung in Kapitel \ref{sec:regulaer} sehr "ahnlich!
Eine besondere Beobachtung ist nun, dass die Stabilit"at offenbar von $b(x)$ abh"angt.
F"ur Umgebungen $U$, in denen die Gleichung
\begin{align}
\partial_s v(s,x) = \frac{b(x)}{2} \partial^2_x v_0(s,x)
\end{align}
l"osbar ist, k"onnen wir hier ganz analog zum Vorgehen in \cite{Junk2004} Stabilit"at zeigen.
Unabh"angig von Stabilit"at l"asst sich \eqref{eq:3_1.regulaer:quellterm_2} auch als inhomogene Transportgleichung lesen.
Die L"osung hierf"ur ist durch
\begin{align}
u_2(t, \tau, x) = \frac{t b(x)}{2} \partial^2_x U(\Phi(-\tau, x- t))
\end{align}
gegeben. F"ur kleine Zeiten $t > 0$ l"asst sich der Ansatz also zu
\begin{align}
v^n_i = U(\Phi(-h t_n, x_i- t_n)) + h^2 t_n \frac{b(x_i)}{2} \partial^2_x U(\Phi(-h t_n, x_i - t_n)) + O(h^3).
\end{align}
schreiben
% Um den Quellterm in Gleichung \eqref{eq:3_1.regulaer:quellterm_2} zu eliminieren, führen wir eine zweite langsame Zeitvariable $\sigma = h^2 t$ ein, fordern
% \begin{align}
% \partial_\sigma u_0(t, \tau, \sigma, x) &= \frac{b(x)}{2} \partial^2_x u_0(t, \tau, \sigma, x) \qquad \text{und}\\
% u_0(0, 0, 0, x) &= 0.
% \end{align}

\section{Hochfrequenter Exponentialansatz}

%!TEX root=thesis.tex

Wir betrachten nun den Ansatz

\[ v^n_k = e^{i \omega k} \Omega(n) \bigl( u_0(t_n, x_k) + h u_1(t_n, x_k) \bigr) + O(h^2). \]

Es gilt
\begin{align} \label{wkb:zeitdiff}
\begin{split}
v^{n+1}_k - v^n_k
% &= e^{i \omega k} \Omega(n+1) \bigl( u_0(t_{n+1}, x_k) + h u_1(t_{n+1}, x_k) \bigr)
% - e^{i \omega k} \Omega(n) \bigl( u_0(t_n, x_k) + h u_1(t_n, x_k) \bigr)\\
% &= e^{i \omega k} \Omega(n+1) \bigl( u_0(t_n + h, x_k) + h u_1(t_n + h, x_k) \bigr)
% - e^{i \omega k} \Omega(n) \bigl( u_0(t_n, x_k) + h u_1(t_n, x_k) \bigr)\\
% &= e^{i \omega k} \Bigl( \Omega(n+1) \bigl( u_0(t_n, x_k) + h \partial_t u_0(t_n, x_k) + h^2 \frac{1}{2} \partial^2_t u_0(t_n, x_k)\\
% &\quad + h u_1(t_n, x_k) + h^2 \partial_t u_1(t_n, x_k) \bigr) - \Omega(n) \bigl( u_0(t_n, x_k) + h u_1(t_n, x_k) \bigr) \Bigr)\\
&= e^{i \omega k} \Bigl( \bigl(\Omega(n+1) - \Omega(n) \bigr) u_0(t_n, x_k) + h \bigl(\Omega(n+1) - \Omega(n) \bigr) u_1(t_n, x_k)\\
&\quad + h \Omega(n+1) \partial_t u_0(t_n, x_k) + h^2 \Omega(n+1) \bigl( \frac{1}{2} \partial^2_t u_0(t_n, x_k) + \partial_t u_1(t_n, x_k) \bigr) \Bigr) + O(h^3)
\end{split}
\end{align}
und
\begin{align}
\begin{split}\label{wkb:ortdiff}
a(x_k) \bigl( v^n_k - v^n_{k-1} \bigr)
% &= a(x_k) \Bigl( e^{i \omega k} \Omega(n) \bigl( u_0(t_n, x_k) + h u_1(t_n, x_k) \bigr)\\
% &\quad - e^{i \omega (k-1)} \Omega(n) \bigl( u_0(t_n, x_{k-1}) + h u_1(t_n, x_{k-1}) \bigr) \Bigr)\\
% &= a(x_k) e^{i \omega k} \Omega(n) \Bigl( \bigl( u_0(t_n, x_k) + h u_1(t_n, x_k) \bigr)\\
% &\quad - e^{- i \omega} \bigl( u_0(t_n, x_k-h) + h u_1(t_n, x_k-h) \bigr) \Bigr)\\
% &= a(x_k) e^{i \omega k} \Omega(n) \Bigl( \bigl( u_0(t_n, x_k) + h u_1(t_n, x_k) \bigr)\\
% &\quad - e^{- i \omega} \bigl( u_0(t_n, x_k) - h \partial_x u_0(t_n, x_k) + h^2 \frac{1}{2} \partial^2_x u_0(t_n, x_k)\\
% &\quad + h u_1(t_n, x_k) - h^2 \partial_x u_1(t_n, x_k) \bigr) \Bigr) + O(h^3)\\
&= a(x_k) e^{i \omega k} \Omega(n) \Bigl( \left(1-e^{-i \omega}\right) \bigl( u_0(t_n, x_k) + h u_1(t_n, x_k) \bigr)\\
&\quad + h e^{-i \omega} \partial_x u_0(t_n, x_k) + h^2 e^{-i \omega} \bigl( \partial_x u_1(t_n, x_k) - \frac{1}{2} \partial^2_x u_0(t_n, x_k) \bigr) \Bigr) + O(h^3).
\end{split}
\end{align}
In Summe folgt
\begin{align}\label{eq:wkb:difflsg}
\begin{split}
&v^{n+1}_k - v^n_k + a(x_k) \bigl( v^n_k - v^n_{k-1} \bigr)\\
% &= e^{i \omega k} \Bigl( \bigl(\Omega(n+1) - \Omega(n) \bigr) u_0(t_n, x_k) + h \bigl(\Omega(n+1) - \Omega(n) \bigr) u_1(t_n, x_k)\\
% &\quad + h \Omega(n+1) \partial_t u_0(t_n, x_k) + h^2 \Omega(n+1) \bigl( \frac{1}{2} \partial^2_t u_0(t_n, x_k) + \partial_t u_1(t_n, x_k) \bigr) \Bigr)\\
% &\quad + a(x_k) e^{i \omega k} \Omega(n) \Bigl( \left(1-e^{-i \omega}\right) \bigl( u_0(t_n, x_k) + h u_1(t_n, x_k) \bigr)\\
% &\quad + h e^{-i \omega} \partial_x u_0(t_n, x_k) + h^2 e^{-i \omega} \bigl( \partial_x u_1(t_n, x_k) - \frac{1}{2} \partial^2_x u_0(t_n, x_k) \bigr) \Bigr)\\
% \end{split}\\
% \begin{split}
&= e^{i \omega k} \Bigl(
\bigl(\Omega(n+1) - \Omega(n) \bigr) u_0(t_n, x_k) + a(x_k) \Omega(n) \left(1-e^{-i \omega}\right) u_0(t_n, x_k)\\
&\quad + h \bigl(\Omega(n+1) - \Omega(n) \bigr) u_1(t_n, x_k) + a(x_k) \Omega(n) \left(1-e^{-i \omega}\right) u_1(t_n, x_k) \bigr)\\
&\quad + h \bigl( \Omega(n+1) \partial_t u_0(t_n, x_k) + a(x_k) \Omega(n) e^{-i \omega} \partial_x u_0(t_n, x_k) \bigr)\\
&\quad + h^2 \bigl( \Omega(n+1) \partial_t u_1(t_n, x_k) + a(x_k) \Omega(n) e^{-i \omega} \partial_x u_1(t_n, x_k)\\
&\qquad + \frac{1}{2} \left( \Omega(n+1) \partial^2_t u_0(t_n, x_k) - a(x_k) \Omega(n) e^{-i \omega} \partial^2_x u_0(t_n, x_k) \right) \bigr) 
\Bigr)
\end{split}
\end{align}
und das heißt wir erhalten in führender Ordnung die Gleichung für alle $h > 0$, $n \in \N$ und $k \in \Z$
\begin{align}
\left( \Omega(n+1) - \left( 1 - a(x_k) \left(1-e^{-i \omega}\right) \right) \Omega(n) \right) u_0(t_n, x_k) = 0.
\end{align}
Hieraus folgt, dass 
$\Omega(n+1) = \left( 1 - a(x_k) \left(1-e^{-i \omega}\right) \right) \Omega(n)$
gelten muss.
Allerdings ist $\Omega$ nicht von $k$ abhängig und sobald es zwei Punkte $x,y \in \R$ mit $a(x) \neq a(y)$ gibt, folgt als einzige sinnvolle Lösung $\Omega(n) = 0$.
Das Problem bei diesem Ansatz ist, dass wir nun zwar eine ortsabhängige Geschwindigkeit $a$ betrachten, $\Omega$ jedoch nicht entsprechend angepasst haben.

\subsection*{Ortsabhängige Korrektur für $\Omega$}

Wir fordern daher nun zusätzlich, dass $\Omega(n,x)$ ortsabhängig und glatt in $x$ ist.
Der Ansatz lautet
\begin{align}
v^n_k = e^{i \omega k} \Omega(n, x_k) \bigl( u_0(t_n, x_k) + h u_1(t_n, x_k) \bigr) + O(h^2).
\end{align}

Da die Gleichung \eqref{wkb:zeitdiff} nur eine Differenz in der Zeit beschreibt, folgt für diesen Ansatz ganz analog
\begin{align}
\begin{split}
v^{n+1}_k - v^n_k &= e^{i \omega k} \Bigl( \bigl(\Omega(n+1, x_k) - \Omega(n,x_k) \bigr) u_0(t_n, x_k)\\
&+ h \bigl(\Omega(n+1,x_k) - \Omega(n,x_k) \bigr) u_1(t_n, x_k) + h \Omega(n+1, x_k) \partial_t u_0(t_n, x_k)\\
&+ h^2 \Omega(n+1, x_k) \bigl( \frac{1}{2} \partial^2_t u_0(t_n, x_k) + \partial_t u_1(t_n, x_k) \bigr) \Bigr) + O(h^3).
\end{split}
\end{align}
Wegen der neuen Regularität von $\Omega$ in der Ortskoordinate $x$ müssen wir Gleichung \eqref{wkb:ortdiff} neu berechnen.
Es gilt

\begin{align}\label{eq:wkb:adiff}
\begin{split}
&a(x_k) \bigl( v^n_k - v^n_{k-1} \bigr)\\
% &=
% a(x_k) \Bigl( e^{i \omega k} \Omega(n,x_k) \bigl( u_0(t_n, x_k) + h u_1(t_n, x_k) \bigr)
% - e^{i \omega (k-1)} \Omega(n,x_{k-1}) \bigl( u_0(t_n, x_{k-1}) + h u_1(t_n, x_{k-1}) \bigr) \Bigr)\\
% &=
% a(x_k) \Bigl( e^{i \omega k} \Omega(n,x_k) \bigl( u_0(t_n, x_k) + h u_1(t_n, x_k) \bigr)
% - e^{i \omega (k-1)} \Omega(n,x_{k-1}) \bigl( u_0(t_n, x_{k-1}) + h u_1(t_n, x_{k-1}) \bigr) \Bigr)\\
% &=
% a(x_k) \Bigl( e^{i \omega k} \Omega(n,x_k) \bigl( u_0(t_n, x_k) + h u_1(t_n, x_k) \bigr)\\
% &\quad - e^{i \omega (k-1)} \left( \Omega(n,x_k) - h \partial_x \Omega(n,x_k) + \frac {h^2}2 \partial^2_x \Omega(n,x_k) \right)
% \cdot \bigl( u_0(t_n, x_{k-1}) + h u_1(t_n, x_{k-1}) \bigr) \Bigr) + O(h^3)\\
% &=
% a(x_k) e^{i \omega k} \Bigl( \Omega(n,x_k) \bigl( u_0(t_n, x_k) + h u_1(t_n, x_k) \bigr) - e^{- i \omega} \left( \Omega(n,x_k) - h \partial_x \Omega(n,x_k) + \frac {h^2}2 \partial^2_x \Omega(n,x_k) \right)\\
% &\qquad \cdot \bigl( u_0(t_n, x_k) - h \partial_x u_0(t_n, x_k) + h^2 \frac{1}{2} \partial^2_x u_0(t_n, x_k) + h u_1(t_n, x_k) - h^2 \partial_x u_1(t_n, x_k) \bigr) \Bigr) + O(h^3)\\
&=
a(x_k) e^{i \omega k} \Bigl( \Omega(n,x_k) (1-e^{-i\omega}) u_0(t_n, x_k)\\
&\quad + h \Omega(n,x_k) (1 - e^{-i \omega}) u_1(t_n, x_k) + h e^{-i \omega} \bigl( \Omega(n,x_k) \partial_x u_0(t_n, x_k) + \partial_x \Omega(n,x_k)  u_0(t_n, x_k) \bigr)\\
&\quad + h^2 e^{-i\omega}  \bigl( \left( \Omega(n,x_k) \partial_x u_1(t_n, x_k) + \partial_x \Omega(n,x_k) u_1(t_n,x_k) \right) - \partial_x \Omega(n,x_k) \partial_x u_0(t_n, x_k) \bigr)\\
&\quad - h^2 \frac{1}{2} e^{-i\omega} \left( \Omega(n,x_k) \partial^2_x u_0(t_n, x_k) + \partial^2_x \Omega(n, x_k) u_0(t_n, x_k) \right)   \Bigr) + O(h^3).
% &\quad - e^{- i \omega} \left( \Omega(n,x_k) - h \partial_x \Omega(n,x_k) + \frac {h^2}2 \partial^2_x \Omega(n,x_k) \right)\\
% &\qquad \cdot \bigl( u_0(t_n, x_k) - h \partial_x u_0(t_n, x_k) + h^2 \frac{1}{2} \partial^2_x u_0(t_n, x_k)\\
% &\quad + h u_1(t_n, x_k) - h^2 \partial_x u_1(t_n, x_k) \bigr) \Bigr)\\
\end{split}
\end{align}

Betrachten wir beide Ergebnisse in der Summe \eqref{eq:varadv:verfahren}, so erhalten wir in $O(1)$ zunächst die von uns gewünschte Gleichung
\begin{align}
\Bigl( \Omega(n+1, x_k)  - \Omega(n,x_k) \bigl(1 - a(x_k) (1-e^{-i\omega}) \bigr) \Bigr) u_0(t_n, x_k) &= 0
\end{align}
aus welcher wir
\begin{align}
\Omega(n,x_k) &= \bigl(1 - a(x_k) (1-e^{-i\omega}) \bigr)^n, \qquad \text{und}\\
\begin{split}\label{eq:omegadiff}
\partial_x \Omega(n,x_k) &= n \bigl(1 - a(x_k) (1-e^{-i\omega}) \bigr)^{n-1} a'(x_k) (1-e^{-i\omega})\\
&= n a'(x_k) (1-e^{-i\omega}) \Omega(n-1, x_k)
\end{split}
\end{align}
ableiten.
Das führt in der Ordnung $O(h)$ zu dem Gleichungssystem
\begin{align}\label{eq:wkb:ohglgpre}
\partial_t u_0(t_n, x_k) + e^{-i \omega} \frac{\Omega(n,x_k)}{\Omega(n+1, x_k)} \partial_x u_0(t_n, x_k) &= - e^{-i \omega} \frac{\partial_x \Omega(n,x_k)}{\Omega(n+1, x_k)} u_0(t_n, x_k).
\end{align}
Auf der rechten Seite dieser Gleichung taucht nun der Term $- e^{-i \omega} \partial_x \Omega(n,x_k) u_0(t_n, x_k)$ auf.
In Gleichung \eqref{eq:omegadiff} haben wir $\partial_x \Omega(n,x_k) = n a'(x_k) (1-e^{-i\omega}) \Omega(n-1, x_k)$ gezeigt.
Darum wird aus Gleichung \eqref{eq:wkb:ohglgpre} 
\begin{align}\label{eq:wkb:ohpost}
\begin{split}
&\partial_t u_0(t_n, x_k) + \frac{e^{-i \omega}}{1 - a(x_k) (1-e^{-i\omega})} \partial_x u_0(t_n, x_k) =\\
&\qquad \qquad - \frac{e^{-i \omega} (1 - e^{-i\omega})}{(1 - a(x_k) (1-e^{-i\omega}))^2} \cdot n a'(x_k) \cdot u_0(t_n, x_k).
\end{split}
\end{align}

Weil aus $h \to 0$ auch $n \to \infty$ folgt, impliziert das Konvergenzprobleme für die linke Seite dieser Gleichung.
Bisher haben wir an dieser Stelle das Lemma \ref{lemma:transport:diskret:konvergenz_gitter} angewandt um von diskreten Gleichungen auf kontinuierliche Differentialgleichungen zu schlie"sen.
Lemma \ref{lemma:transport:diskret:konvergenz_gitter} macht in Gleichung \eqref{eq:lemma:diag:aussage} hier keine Aussage "uber die Konvergenz.
Ferner kann man dies hier durch eine Art Sublinear-Growth Bedingung ``korrigieren''.
Betrachte man ein $a_n(x) = 1 + O\left(\frac{1}{n} \right)$, zum Beispiel $a_n(x) = 1 + \frac{1}{n} b(x)$ f"ur ein beschr"anktes $b \in C^\infty(\R)$, so folgt $a'(x) = \frac{1}{n} b'(x)$ und somit
\begin{align}\label{eq:wkb:ohpost2}
\begin{split}
&\partial_t u_0(t_n, x_k) + \frac{e^{-i \omega}}{1 - a(x_k) (1-e^{-i\omega})} \partial_x u_0(t_n, x_k) =\\
&\qquad \qquad - \frac{e^{-i \omega} (1 - e^{-i\omega})}{(1 - a(x_k) (1-e^{-i\omega}))^2} \cdot b(x_k) \cdot u_0(t_n, x_k).
\end{split}
\end{align}

Das motiviert uns nun direkt in den Fall $a(x) = 1 + hb(x)$ überzugehen.

\section{Exponentialansatz f"ur kleine St"orungen}

Wir verwenden hier das Ergebnis aus Gleichung \eqref{eq:wkb:difflsg} aus dem Vorkapitel wieder und setzen $a(x) = 1 + hb(x)$ ein.
Es gilt
\begin{align}
% \begin{split}
\nonumber
&v^{n+1}_k - v^n_k + \bigl(1 + hb(x_k)\bigr) \bigl( v^n_k - v^n_{k-1} \bigr)\\
% &= e^{i \omega k} \Bigl(
% \bigl(\Omega(n+1) - \Omega(n) \bigr) u_0(t_n, x_k) + \bigl(1 + hb(x_k)\bigr) \Omega(n) \left(1-e^{-i \omega}\right) u_0(t_n, x_k)\\
% &\quad + h \bigl( \bigl(\Omega(n+1) - \Omega(n) \bigr) u_1(t_n, x_k) + \bigl(1 + hb(x_k)\bigr) \Omega(n) \left(1-e^{-i \omega}\right) u_1(t_n, x_k) \bigr)\\
% &\quad + h \bigl( \Omega(n+1) \partial_t u_0(t_n, x_k) + \bigl(1 + hb(x_k)\bigr) \Omega(n) e^{-i \omega} \partial_x u_0(t_n, x_k) \bigr)\\
% &\quad + h^2 \bigl( \Omega(n+1) \partial_t u_1(t_n, x_k) + \bigl(1 + hb(x_k)\bigr) \Omega(n) e^{-i \omega} \partial_x u_1(t_n, x_k)\\
% &\qquad + \frac{1}{2} \left( \Omega(n+1) \partial^2_t u_0(t_n, x_k) - \bigl(1 + hb(x_k)\bigr) \Omega(n) e^{-i \omega} \partial^2_x u_0(t_n, x_k) \right) \bigr) 
% \Bigr)\\&\quad + O(h^3)
% \end{split}\\
\begin{split}\label{eq:wkb:letzteglg}
&= e^{i \omega k} \Bigl(
\bigl(\Omega(n+1) - \Omega(n) \bigr) u_0(t_n, x_k) + \Omega(n) \left(1-e^{-i \omega}\right) u_0(t_n, x_k)\\
&\quad + h \bigl( \bigl(\Omega(n+1) - \Omega(n) \bigr) u_1(t_n, x_k) + \Omega(n) \left(1-e^{-i \omega}\right) u_1(t_n, x_k) \bigr)\\
&\quad + h \bigl( \Omega(n+1) \partial_t u_0(t_n, x_k) + \Omega(n) e^{-i \omega} \partial_x u_0(t_n, x_k) \bigr)\\
&\quad + h b(x_k) \Omega(n) \left(1-e^{-i \omega}\right) u_0(t_n, x_k)\\
&\quad + h^2 \bigl( \Omega(n+1) \partial_t u_1(t_n, x_k) + \Omega(n) e^{-i \omega} \partial_x u_1(t_n, x_k)\\
&\qquad + \frac{1}{2} \left( \Omega(n+1) \partial^2_t u_0(t_n, x_k) - \Omega(n) e^{-i \omega} \partial^2_x u_0(t_n, x_k) \right) \bigr)\\
&\qquad + b(x_k) \Omega(n) e^{-i \omega} \partial_x u_0(t_n, x_k) + b(x_k) \Omega(n) \left(1-e^{-i \omega}\right) u_1(t_n, x_k)
\Bigr)\\&\quad + O(h^3).
\end{split}
\end{align}
In erster Ordnung gilt also
\begin{align}
\left( \Omega(n+1) - e^{-i \omega} \Omega(n)  \right) u_0(t_n, x_k) = 0
\end{align}
und daraus folgern wir $\Omega(n) = e^{- i \omega n}$ für alle $n \in \N$.
Das setzen wir in die Gleichung \eqref{eq:wkb:letzteglg} ein und klammern den
Faktor $e^{- \omega n}$ aus:
\begin{align}
\begin{split}
0 &= v^{n+1}_k - v^n_k + \bigl(1 + hb(x_k)\bigr) \bigl( v^n_k - v^n_{k-1} \bigr)\\
&= e^{i \omega (k - n)} \Bigl(
h \bigl( e^{-i \omega} \partial_t u_0(t_n, x_k) + e^{-i \omega} \partial_x u_0(t_n, x_k) + b(x_k) \left(1-e^{-i \omega}\right) u_0(t_n, x_k) \bigr)\\
&\quad + h^2 \bigl( e^{-i \omega} \partial_t u_1(t_n, x_k) + e^{-i \omega} \partial_x u_1(t_n, x_k) + b(x_k) e^{-i \omega} \partial_x u_0(t_n, x_k)\\
&\qquad + \frac{1}{2} \left( e^{-i \omega} \partial^2_t u_0(t_n, x_k) - e^{-i \omega} \partial^2_x u_0(t_n, x_k) \right)  + b(x_k) \left(1-e^{-i \omega}\right) u_1(t_n, x_k) \bigr) 
\Bigr) + O(h^3)
\end{split}
\end{align}
Das liefert uns nun mit $C := e^{i \omega} \left(1-e^{-i \omega}\right)$ das Gleichungssystem
\begin{align}\label{eq:wkb:oh11}
\partial_t u_0(t, x) + \partial_x u_0(t, x) &= - C b(x) u_0(t, x)\\
\begin{split}\label{eq:wkb:oh22}
\partial_t u_1(t, x) + \partial_x u_1(t, x) &= - b(x) \partial_x u_0(t, x) - \frac{1}{2} \left( \partial^2_t u_0(t, x) - \partial^2_x u_0(t, x) \right)\\
&\quad - C b(x) u_1(t, x)
\end{split}\\
u_0(0,x) &= V(x)\\
u_1(0,x) &= 0.
\end{align}
Aus der Gleichung \eqref{eq:wkb:oh11} folgern wir
\begin{align*}
\partial^2_t u_0(t,x) &= \partial_t \Bigl( \partial_t u_0(t,x) \Bigr)\\
&= \partial_t \Bigl( -C b(x) u_0(t,x) - \partial_x u_0(t,x) \Bigr)\\
&= - C b(x) \partial_t u_0(t,x) - \partial_x \Bigl( \partial_t u_0(t,x)  \Bigr)\\
&= - C b(x) \Bigl( -C b(x) u_0(t,x) - \partial_x u_0(t,x) \Bigr) - \partial_x \Bigl( -C b(x) u_0(t,x) - \partial_x u_0(t,x) \Bigr)\\
&= C^2 b^2(x) u_0(t,x) + C b(x) \partial_x u_0(t,x) + C \partial_x \Bigl( b(x) u_0(t,x) \Bigr) + \partial^2_x u_0(t,x)\\
&= C^2 b^2(x) u_0(t,x) + C \partial_x b(x) u_0(t,x) + \partial^2_x u_0(t,x).
\end{align*}
Dies setzen wir in \eqref{eq:wkb:oh22} ein und erhalten die Gleichung
\begin{align}
\begin{split}\label{eq:wkb:oh222}
\partial_t u_1(t, x) + \partial_x u_1(t, x) &= - C b(x) u_1(t, x) \\
&\quad - b(x) \partial_x u_0(t, x) - \left( \frac{C^2}{2} b^2(x) + \frac{C}{2} \partial_x b(x) \right) u_0(t,x).
\end{split}
\end{align}
Die L"osungen zu diesen Differentialgleichungen lassen sich "ahnlich wie im Anhang bestimmen und sie lauten
\begin{align}
u_0(t,x) &= U(x-t) e^{- \Ree C \int^t_0 b(s) ds } \qquad \text{und}\\
u_1(t,x) &= - t \Bigl( \underbrace{b(x) U'(x-t) + \frac{1}{2} \left( \Ree C^2 b^2(x) + \Ree C \partial_x b(x) \right) U(x-t)}_{=: \Delta_b U(x-t)} \Bigr) e^{ \Ree C \int^t_0 b(s) ds } 
\end{align}
Zusammenfassend ergibt das f"ur diesen Ansatz
\begin{align}
v^n_i = e^{i \omega (k - n)}e^{ \Ree C \int^t_0 b(s) ds } \bigl(  U(x-t) - h t \Delta_b U(x-t) \bigr) + O(h^2).
\end{align}
Dabei ist schon erstaunlich wie sehr sich diese L"osung von der L"osung \eqref{eq:3:reg:loesung} unterscheidet.

Ferner w"are es auch m"oglich wieder einen Multiskalenansatz $u(t,\tau,x)$ mit einer langsamen Zeitvariablen $\tau = h t$ zu betrachten.
Wie schon gewohnt erhalten wir das Gleichungssystem mit einer weiteren Gleichung f"ur die langsame Zeit $\tau$.
\begin{align}\label{eq:wkb:oh11}
\partial_t u_0(t, \tau, x) + \partial_x u_0(t, \tau, x)         &= - C b(x) u_0(t, \tau, x)\\
\partial_\tau u_0(t, \tau, x) + b(x) \partial_x u_0(t, \tau, x) &= - \left( \frac{C^2}{2} b^2(x) + \frac{C}{2} \partial_x b(x) \right) u_0(t, \tau, x)\\
\partial_t u_1(t, \tau, x) + \partial_x u_1(t, \tau, x)         &= - C b(x) u_1(t, \tau, x)\\
u_0(0,0,x) &= V(x)\\
u_1(0,0,x) &= 0.
\end{align}
Qualitativ sieht man, dass man die urspr"ungliche partielle Differentialgleichung \eqref{eq:varadv:pde} nun auch in der langsamen Zeitvariablen $\tau$ wieder findet -- wie es auch zu erwarten ist, da wir die Transportgeschwindigkeit auch nur asymptotisch klein in $h$ "andern.