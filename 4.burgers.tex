%!TEX root=thesis.tex

Eine mögliche Fortsetzung dieser Arbeit ist die Untersuchung von diskreten Verfahren nichtlinearer  Differentialgleichungen wie die Burgersgleichung.
Diese lautet
\begin{align}
\begin{split}
\partial_t u(t,x) + \partial_x \left( \, u^2(t,x) \,\right) &= 0 \qquad \forall (t,x) \in \Rp \times \R\\
u(0,x) &= U(x).
\end{split}
\end{align}
Für diese Gleichung betrachte man das Godunov Verfahren
\begin{align}
v^{n+1}_i &= v^n_i - \lambda \bigl( g(v^n_i, v^n_{i+1}) - g(v^n_{i-1}, v^n_i) \bigr)
\end{align}
mit
\begin{align}
g(u,v) = \frac{\left(u^*\right)^2}2 \text{ und } u^* =
\begin{cases}
u & \text{falls $u > v$ und $\frac {u+v}2 > 0$}\\
v & \text{falls $u > v$ und $\frac {u+v}2 \leq 0$}\\
u & \text{falls $u \leq v$ und $u > 0$}\\
v & \text{falls $u \leq v$ und $v < 0$}\\
0 & \text{sonst.}
\end{cases}
\end{align}
Ein wichtiges Prinzip dieser Arbeit manifestiert sich für den linearen Fall in der Superposition von oszillierenden Ansätzen.
Die Sätze \ref{satz:alt:beidenull} und \ref{satz:glgregulaer} ermöglichen uns Multiskalenansätze der Form
\[ u(k,n,x,t) = \sum_{l=1}^n e^{i \omega_l k} \Omega(n,x,t) u_l(x,t) \]
besser zu verstehen, wenn wir die Lösungen der Ansätze für die einzelnen Summanden
\[ u(k,n,x,t) = e^{i \omega_l k} \Omega(n,x,t) u_i(x,t) \qquad l = 1, \ldots, n\]
kennen.
Im nichtlinearen Fall ist dies so nicht mehr direkt übertragbar und die Ansätze überlagern sich.
Hier muss folglich eine andere Herangehensweise gefunden werden.

Eine weitere weniger herausfordernde Erweiterung dieser Arbeit wäre die Formulierung dieser Techniken für höhere Raum-Dimensionen.
Diese Verallgemeinerung ist jedoch unkompliziert durchführbar.