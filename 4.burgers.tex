Eine natürliche Fortsetzung dieser Arbeit ist die Untersuchung von diskreten Verfahren nichtlinearer  Differentialgleichungen wie die Burgersgleichung.
Diese lautet
\begin{align}
\begin{split}
\partial_t u(t,x) + \partial_x \left( \, u^2(t,x) \,\right) &= 0\\
u(0,x) &= U(x).
\end{split}
\end{align}
Für diese Gleichung betrachte man das Godunov Verfahren
\begin{align}
v^{n+1}_i &= v^n_i - \lambda \left( g(v^n_i, v^n_{i+1}) - g(v^n_{i-1}, v^n_i) \right)
\end{align}
mit
\begin{align}
g(u,v) = \frac{u^*}2 \text{ und } u^* =
\begin{cases}
u & \text{falls $u > v$ und $\frac {u+v}2 > 0$}\\
v & \text{falls $u > v$ und $\frac {u+v}2 \leq 0$}\\
u & \text{falls $u \leq v$ und $u > 0$}\\
v & \text{falls $u \leq v$ und $V < 0$}\\
0 & \text{sonst.}
\end{cases}
\end{align}
Die Ansätze in dieser Arbeit funktionieren auf Grund der Linearität der Gleichungen.
Aus dieser folgt nämlich, dass die Frequenz von oszillierenden Abbildungen invariant unter dem diskreten Differentialoperator bleiben. 
Im nichtlinearen Fall muss ein anderer Ansatz gefunden werden, da dies hier mehr zutrifft.