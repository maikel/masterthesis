%!TEX root=thesis.tex

Eine naheliegende Fortsetzung dieser Arbeit ist die Untersuchung diskreter Verfahren nichtlinearer Differentialgleichungen wie die Burgersgleichung.
Diese lautet im eindimensionalen Fall
\begin{align}
\begin{split}
\partial_t u(t,x) + \partial_x \left( \, u^2(t,x) \,\right) &= 0 \qquad \forall (t,x) \in \Rp \times \R\\
u(0,x) &= U(x).
\end{split}
\end{align}
Für diese Gleichung betrachte man das Godunov Verfahren (siehe auch \cite{leveque2002finite})
\begin{align}
v^{n+1}_i &= v^n_i - \lambda \bigl( g(v^n_i, v^n_{i+1}) - g(v^n_{i-1}, v^n_i) \bigr)
\end{align}
mit
\begin{align}
g(u,v) = \frac{\left(u^*\right)^2}2 \text{ und } u^* =
\begin{cases}
u & \text{falls $u > v$ und $\frac {u+v}2 > 0$}\\
v & \text{falls $u > v$ und $\frac {u+v}2 \leq 0$}\\
u & \text{falls $u \leq v$ und $u > 0$}\\
v & \text{falls $u \leq v$ und $v < 0$}\\
0 & \text{sonst.}
\end{cases}
\end{align}
Die hier pr"asentierte Methodik wird jedoch nicht eins-zu-eins "ubertragbar sein.
Ein wichtiges Prinzip dieser Arbeit manifestiert sich n"amlich in der Superposition von oszillierenden Ansätzen.
Die Sätze \ref{satz:alt:beidenull} und \ref{satz:glgregulaer} ermöglichen es Multiskalenansätze der Form
\[ u(k,n,x,t) = \sum_{l=1}^n e^{i \omega_l k} \Omega(n,x,t) u_l(x,t) \]
besser zu verstehen, wenn wir die Lösungen der Ansätze für die einzelnen Summanden
\begin{align}\label{burgers:1}
u(k,n,x,t) = e^{i \omega_l k} \Omega(n,x,t) u_l(x,t) \qquad l = 1, \ldots, n
\end{align}
kennen.
Im nichtlinearen Fall ist dies nicht mehr so.
Startet man mit Ansatzfunktionen verschiedener Frequenzen, so können sich diese im nichtlinearen Fall überlagern und folglich beeinflussen.
Es muss also eine andere Herangehensweise gefunden werden.

Insgesamt sollte der mathematische Rahmen dieser Arbeit noch konkretisiert und funktionalanalytisch richtig eingeordnet werden.
So sind zum Beispiel die Voraussetzungen des Lemmas \ref{lemma:transport:diskret:konvergenz_gitter} zu stark.
Ich vermute, dass ein analoger Beweis mit Bedingungen an die Totalvariation der auftretenden Funktionen existiert.

Eine weitere und weniger herausfordernde Erweiterung dieser Arbeit wäre die Formulierung für höhere Raum-Dimensionen.