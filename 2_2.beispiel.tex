\begin{figure}
\centering
\begin{tikzpicture}
\begin{semilogyaxis}[
    title={$\err^n = \max_{i \in \Z} \abs{\sin\bigl(\pi (x_i -t_n)\bigr) - v^n_i}$},
    xlabel={Iterationen $n$},
    ylabel={maximaler Fehler},
    legend entries={$\err^n \, \eta=0.1;\, h=0.01$,$\err^n \, \eta=0.1;\, h=0.001$,$\err^n \, \eta=0.05;\, h=0.001$, $\bar \err^n = t_n \eta \frac{\pi^2}{2}$},
    legend style={at={(1.8,1)}}
]
\addplot[red, line width={0.8}, mark=*] table [x index={0}, y index={1}] {data/max_errors_eta_0.100_h_0.010.dat};
\addplot[dg, line width={0.8}, mark=*] table {data/max_errors_eta_0.100_h_0.001.dat};
\addplot[myblue, line width={0.8}, mark=*] table {data/max_errors_eta_0.050_h_0.001.dat};
\addplot[mygray, line width={0.8}, dashdotted] table [y index={4}] {data/max_errors_eta_0.050_h_0.001.dat};
\addplot[mygray, line width={0.8}, dashdotted] table [y index={4}] {data/max_errors_eta_0.100_h_0.001.dat};
\addplot[mygray, line width={0.8}, dashdotted] table [y index={4}] {data/max_errors_eta_0.100_h_0.010.dat};
\end{semilogyaxis}
\end{tikzpicture}
\caption{Hier vergleichen wir die maximalen Fehler der numerischen Lösungen für die Startwerte $U(x) = sin(\pi x)$ und $h = 10^{-3}$ für $\eta = 0.1, 0.05$ und $0.001$ zur echten Lösung $u(t,x) = sin(\pi(x - t))$ mit dem geschätztem Fehler $u(t_n, x_i) - v^n_i$ in \eqref{eq:transport:regulaer:fehler}.}
\label{figure:regular:fehler}
\end{figure}

Wir schauen uns die Ergebnisse des Verfahrens für die Anfangswerte $U(x) = \sin(\pi x)$ und $\lambda = 1 + \eta$ für verschiedene positive $\eta$ und Gitterweiten $h$ an.
Wir haben die Beispiele durch die Skripte im Anhang \ref{appendix:transport:beispiel} mit dem Programm \emph{GNU Octave} umgesetzt.
Das gibt uns erste Hinweise darauf, was wir in unseren Untersuchungen zu erwarten haben.

\begin{figure}
\centering
\begin{tikzpicture}[scale=0.58]
\begin{axis}[
    height = 0.6\textwidth,
    width = 0.6\textwidth,
    title = {$\eta = 0.1, h = 0.01, n = 198$},
    xtick = {-2,-1,...,2},
    grid  = both,
]
\addplot[myblue] table {data/V_sinus_eta_0.100_h_0.001.dat};
\end{axis}
\end{tikzpicture}
\begin{tikzpicture}[scale=0.58]
\begin{axis}[
    height = 0.6\textwidth,
    width = 0.6\textwidth,
    title = {$\eta = 0.1, h = 0.001, n = 198$},
    xtick = {-2,-1,...,2},
    grid  = both,
]
\addplot[myblue] table {data/V_sinus_eta_0.100_h_0.001.dat};
\end{axis}
\end{tikzpicture}
\begin{tikzpicture}[scale=0.58]
\begin{axis}[
    height = 0.6\textwidth,
    width = 0.6\textwidth,
    title = {$\eta = 0.05, h = 0.001, n = 378$},
    xtick = {-2,-1,...,2},
    grid  = both,
]
\addplot[myblue] table {data/V_sinus_eta_0.050_h_0.001.dat};
\end{axis}
\end{tikzpicture}
\caption{Vergleich des instabilen Verhaltens für verschiedene $\eta$ und $h$}
\label{fig:transport:beispiel}
\end{figure}

\begin{figure}

\end{figure}

Weil $\lambda > 1$ gilt, können wir \eqref{eq:transport:regulaer:tau:ergebnis} nicht verwenden.
In Abbildung \ref{figure:regular:fehler} vergleichen wir den maximalen Fehler der numerischen Lösungen mit dem Fehler der asymptotischen Entwicklung in \eqref{eq:transport:regulaer:ansatz:ergebnis}
\begin{align}\label{eq:transport:regulaer:fehler}
\begin{split}
\bar {\err}^n &= \max_{i \in \Z} \abs{ \sin\bigl( \pi (x_n - t_i) \bigr) - v^n_i }\\
&= \max_{i \in \Z} \abs{ t_n \frac{1-\lambda}{2} \partial^2_x \sin\bigl( \pi (x_n - t_i) \bigr) }\\
&=  t_n \frac{\pi^2 \eta}{2} \max_{i \in \Z} \abs{ \sin\bigl( \pi (x_n - t_i) \bigr) }\\
&= t_n \frac{\pi^2 \eta}{2}.
\end{split}
\end{align}
Man sieht in Abbildung \ref{figure:regular:fehler}, dass es zu einem exponentiellem Anstieg des Fehlers kommt, den die Entwicklung nicht vorhersagt.
Betrachtet man nun die Plots in Abbildung~\ref{fig:transport:beispiel}, so sieht man, dass sich hochfrequente Oszillationen aufschaukeln und die richtige Lösung überdecken.
Die Existenz solcher Oszillationen in den Anfangsdaten wurde im regulären Ansatz bisher ignoriert.
Vergleicht man die Plots untereinander, so erkennt man, dass die maximale Amplitude der Oszillation scheinbar unabh"angig von $h$ mit der Anzahl der Iterationen $n$ w"achst.
Verringert man jedoch den Wert für $\eta$, so ändert sich auch die Rate, um die die Amplitude wächst.
Wir werden dieses Verhalten in unseren Approximationen wiederfinden und geben ferner Abschätzungen für das Wachstum der Amplitude an.
% Die konkreten Iterationen $n$ für die Plots in den Abbildungen wurden mit Hilfe der Abschätzung $(1 + 2 \eta)^n$ aus dem Unterkapitel \ref{sec:transport:osz} ausgewählt.