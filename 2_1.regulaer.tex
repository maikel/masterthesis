%!TEX root=thesis.tex

Zunächst präsentieren wir die Ergebnisse, welche die Autoren Junk und Yang aus \cite{Junk2004} für das Upwind-Schema entwickeln.
Angenommen es gäbe Abbildungen $u_0, u_1 \in C^\infty\left( \Rp \times \R \right)$, so dass für alle $h > 0$, $n \in \N$ und $i \in \Z$
\begin{align}\label{eq:transport:regulaer:ansatz}
v^n_i = u_0(t_n,x_i) + h u_1(t_n,x_i) + O(h^2)
\end{align}
gilt, $v^n_i$ die Gleichung \eqref{eq:adv:scheme} erfüllt und die Anfangsbedingungen
\begin{align}\label{eq:transport:regulaer:anfangsbedingungen}
u_0(0,x) = U(x) \quad \text{und} \quad u_1(0, x) = 0
\end{align}
gelten.
Im Anhang \ref{appendix:regulaer:rechnungen} setzen wir den Ansatz \eqref{eq:transport:regulaer:ansatz} in die beiden Differenzen $v^{n+1}_i - v^n_i$ und $\lambda ( v^n_i - v^n_{i-1} )$ ein, verwenden, dass $u_0$ und $u_1$ differenzierbar sind und erhalten nach einem Grenz"ubergang von $h \to 0$ für alle Zeit-Raum Punkte $(t,x) \in \Rp \times \R$ das Gleichungssystem 
\begin{align}
\label{eq:transport:regulaer:oh}
\partial_t u_0(t,x) + \partial_x u_0(t,x) &= 0\\
\label{eq:transport:regulaer:oh2}
\partial_t u_1(t,x) + \partial_x u_1(t,x) &= \frac{1-\lambda}{2} \partial^2_x u_0(t,x).
\end{align}
Gleichung \eqref{eq:transport:regulaer:oh} zeigt, dass $u_0$ selbst schon eine Lösung von \eqref{eq:adv:pde} ist.
Somit bestimmt $u_1$ direkt den Fehler des Verfahrens zur echten Lösung, zumindest solange die asymptotische Entwicklung \eqref{eq:transport:regulaer:ansatz} gültig ist.
Im Anhang \ref{appendix:loesungen} haben wir bereits die Lösung dieser Gleichungen bestimmt.
$u_0$ ist wegen $u_0(0,x) = U(x)$ durch
\begin{align*}
u_0(t,x) = U(x-t)
\end{align*}
gegeben.
Die Gleichung \eqref{eq:transport:regulaer:oh2} fällt hingegen in den Fall einer inhomogenen Transportgleichung und weil $u_0$ selbst Lösung der homogenen Gleichung ist, greift die Lösung \eqref{eq:transport:analytisch:inhomogen:loesung:spezial}:
\begin{align*}
u_1(t,x) = t \frac{1-\lambda}{2} \partial^2_x U(x-t).
\end{align*}
Zwar bleibt der Ansatz \eqref{eq:transport:regulaer:ansatz} für festes $t \in \Rp$ asymptotisch in $h$ geordnet, d.\,h. es gilt stets
\begin{align*}
\lim_{h \to 0} h \frac{u_1(t,x)}{u_0(t,x)} = 0,
\end{align*}
allerdings konvergiert die asymptotische Entwicklung nicht gleichmäßig in $t$, denn es gilt für jedes $h > 0$
\begin{align*}
\lim_{t \to \infty} h \frac{u_1(t,x)}{u_0(t,x)} = \lim_{t \to \infty} h t \frac{(1-\lambda) \partial^2_x U(x-t) }{2 U(x-t)} = \infty.
\end{align*}
Das ist nach \cite{Junk2004} ein Zeichen dafür, dass der resultierende Ansatz
\begin{align}\label{eq:transport:regulaer:ansatz:ergebnis}
v^n_i = U(x-t) + h t \frac{1-\lambda}{2} \partial^2_x U(x-t) + O(h^2)
\end{align}
das Langzeitverhalten des Verfahrens nicht gut widergibt und dieses Problem greifen die Autoren mit einer zusätzlichen \emph{langsamen} Zeitskala $\tau = h t$ an.
Aus \eqref{eq:transport:regulaer:ansatz} wird
\begin{align}\label{eq:transport:regulaer:tau:ansatz}
v^n_i &= u_0(t, \tau, x) + h u_1(t, \tau, x) + O(h^2)
\end{align}
und die Gleichungen \eqref{eq:transport:regulaer:oh} und \eqref{eq:transport:regulaer:oh2} ändern sich zu
\begin{align}
\label{eq:transport:regulaer:tau:gleichungen}
\partial_t u_0(t,\tau,x) + \partial_x u_0(t,\tau,x) &= 0\\
\label{eq:transport:regulaer:tau:2}
\partial_t u_1(t,\tau,x) + \partial_x u_1(t,\tau,x) &= \frac{1-\lambda}{2} \partial^2_x u_0(t,\tau,x) - \partial_\tau u_0(t,\tau,x)\\
u_0(0, 0, x) &= U(x)\\
u_1(0, 0, x) &= 0.
\end{align}
Aus Gleichung \eqref{eq:transport:regulaer:tau:gleichungen} entnehmen wir die L"osung
\begin{align}
u_0(t, \tau, x) &= A(\tau, x-t)\\
A(0, x) &= U(x)
\end{align}
f"ur ein noch weiter zu bestimmendes $A$.
Durch die zusätzliche Zeitvariable $\tau$ haben wir einen weiteren Freiheitsgrad zur Verfügung und können in Gleichung \eqref{eq:transport:regulaer:tau:2} zus"atzlich
\begin{align}\label{eq:transport:regulaer:waermeleit}
\partial_\tau u_0(t,\tau,x) &= \frac{1-\lambda}{2} \partial^2_x u_0(t,\tau,x)\\
\Leftrightarrow \partial_\tau A(\tau, x-t) &= \frac{1-\lambda}{2} \partial^2_x A(\tau,x-t)
\end{align}
fordern. Dies ist die Wärmeleitungsgleichung, welche nur für $\lambda \leq 1$ lösbar ist. Die Fundamentall"osung lautet
\begin{align}
A(\tau, x-t) = (U * G_\tau)(x-t)
\end{align}
wobei $G_\tau$ der Glättungskern
\begin{align}
G_\tau(y) = \frac{1}{\sqrt{2 \pi (1-\lambda)\tau}} \exp\left( - \frac{y^2}{2 (1 - \lambda) \tau} \right).
\end{align} ist.
Für $\lambda \leq 1$ lautet die Lösung des Anfangswertproblems \eqref{eq:transport:regulaer:tau:gleichungen} mit der Bedingung \eqref{eq:transport:regulaer:waermeleit}
\begin{align}
\begin{split}
u_0(t,\tau,x) &= (U * G_\tau)(x - t) \quad \text{und}\\
u_1(t,\tau,x) &= 0
\end{split}
\end{align}
und die asymptotische Entwicklung von \eqref{eq:transport:regulaer:tau:ansatz} lautet hier
\begin{align}\label{eq:transport:regulaer:tau:ergebnis}
v^n_i = (U * G_\tau)(x_i - t_n) + O(h^2).
\end{align}