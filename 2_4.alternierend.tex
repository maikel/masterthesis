%!TEX root=thesis.tex

Hier präzisieren wir unseren Ansatz \eqref{eq:transport:diskret:ansatz} aus dem letztem Unterkapitel. % \ref{sec:transport:diskret}.
Sei $\eps_M > 0$ die Maschinengenauigkeit.
Diese Größe ist von System zu System unterschiedlich und ist in der Regel in der Größenordnung von etwa $10^{-16}$ groß.
Die Anfangsbedingung für $u_0$ in \eqref{eq:transport:diskret:anfangsbedinungen} lautet $u_0(0, x) = U(x)$, doch in Wahrheit rechnet der Computer mit gerundeten Zahlen, also
\begin{align*}
u_0(0, i, 0, x_i) &= \rd{U(x_i)} = U(x_i) - \underbrace{\left( U(x_i) - \rd{U(x_i)} \right)}_{=: \Delta_{\eps_M} U(x_i)}\\
&= U(x_i) - \Delta_{\eps_M} U(x_i),
\end{align*}
Das Vorzeichen der Rundungsfehler $\Delta_{\eps_M} U(x_i)$ kann von Gitterzelle zur Gitterzelle verschieden sein!
Deshalb motiviert das hier den Ansatz, dass Oszillationen im Raum und auf Gitterniveau vorhanden sind und wir untersuchen, wie sich die Amplitude in der Zeit ausbreitet.
Weil wir in \eqref{eq:adv:pde} eine lineare Differentialgleichung betrachten, vermuten wir, dass man $u_k$ für $k = 0,1,2$ und alle Iterationen $n \in \N$ als Summe einer glatten und einer unstetigen, hochfrequenten Funktion schreiben kann.
Daher wählen wir für $u_0, u_1$ und $u_2$ aus \eqref{eq:transport:diskret:ansatz} nun konkreter 
\begin{align}\label{eq:transport:osz:ansatz1}
u_k(n, i, t, x) &= w_k(t, x) + (-1)^i \Omega(n) z_k(t, x), \qquad \text{für $k = 0,1,2$}
\end{align}
für glatte Funktionen $w_k, z_k$, $\Omega(0) = 1$ und
\begin{align}\label{eq:transport:osz:anfangsbedinungen}
\begin{split}
w_0(0, x_i) &= U(x_i),\\
w_1(0, x_i) &= 0,\\
w_2(0, x_i) &= 0,
\end{split}&
\begin{split}
z_0(0, x_i) &= V(x_i),\\
z_1(0, x_i) &= 0,\\
z_2(0, x_i) &= 0.
\end{split}
\end{align}
Wobei $U$ und $V$ derart sind, dass alle nötigen Regularitätsbedingungen für die kommenden Betrachtungen erfüllt seien sollen.
Wir wollen nun $\Omega(n)$ näher bestimmen.
Setzt man dies in die vorherigen Rechnungen ein, so erhält man wegen der Gleichung \eqref{eq:transport:diskret:o1}

\noindent \emph{in der Ordnung $O(1)$:}
{\small \begin{align}\label{eq:transport:osz:o1} 
u_0(n+1, i, t_n, x_i) - u_0(n, i, t_n, x_i) + \lambda \bigl(u_0(n, i, t_n, x_i) - u_0(n, i-1, t_n, x_i) \bigr) &= 0. \nonumber\\
\begin{split}
w_k(t, x) + (-1)^i \Omega(n+1) z_k(t, x) - \bigl( w_k(t, x) + (-1)^i \Omega(n) z_k(t, x) \bigr) \quad &\nonumber\\
+ \lambda \bigl[ w_k(t, x) + (-1)^i \Omega(n) z_k(t, x) - \bigl( w_k(t, x) + (-1)^{i-1} \Omega(n) z_k(t, x) \bigr) \bigr]&
\end{split} = \nonumber\\
(-1)^i \Omega(n+1) z_k(t, x) - (-1)^i \Omega(n) z_k(t, x) + \lambda \bigl[ (-1)^i \Omega(n) z_k(t, x) - (-1)^{i-1} \Omega(n) z_k(t, x) \bigr] & =\nonumber\\
(-1)^i \Omega(n+1) z_k(t, x) - (-1)^i \Omega(n) z_k(t, x) + \lambda \bigl[ (-1)^i \Omega(n) z_k(t, x) + (-1)^{i} \Omega(n) z_k(t, x) \bigr] & =\nonumber\\
(-1)^i \Omega(n+1) z_k(t, x) - (-1)^i \Omega(n) z_k(t, x) + 2 \lambda (-1)^i \Omega(n) z_k(t, x) & =\nonumber\\
(-1)^i \Omega(n+1) z_k(t, x) + (-1 + 2\lambda) (-1)^i \Omega(n) z_k(t, x) & =\nonumber\\
- (-1)^i z_0(t_n, x_i) \Bigl[ \Omega(n+1) - ( 1 - 2 \lambda ) \Omega(n) \Bigr] & =
\end{align} }

Da diese Gleichung für alle Gitterkoordinaten $(n,i)$ gilt und die Funktion $F(t,x) = 0$ glatt ist, können wir unter der Annahme, dass $z_0 \neq 0$ gilt, Lemma \ref{lemma:transport:diskret:konvergenz_gitter} benutzen.
\\

\begin{satz}\label{satz:omega_n}
Es gelte für alle $(n,i) \in \N \times \Z$ die Gleichung \eqref{eq:transport:osz:o1}. Dann gilt für fast alle $n \in \N$
\[ \Omega(n) = (1 - 2 \lambda)^n. \]
\end{satz}
\begin{proof}
Weil $z_0 \neq 0$ gilt, existiert ein Punkt $(t,x) \in \Rp \times \R$ mit $z_0(t,x) \neq 0$.
Nach Lemma \ref{lemma:transport:diskret:konvergenz_gitter} existiert für jedes $n \in \N$ ein $i \in \Z$ mit
\[  \abs{(-1)^i z_0(t, x) \bigl( \Omega(n+1) - ( 1 - 2 \lambda ) \Omega(n) \bigr)} \leq \abs{(-1)^i \frac {t}{\lambda n} \partial_x z_0(t, x) \bigl( \Omega(n+1) - ( 1 - 2 \lambda ) \Omega(n) \bigr)}. \]
Angenommen es gelte $\Omega(n+1) - ( 1 - 2 \lambda ) \Omega(n) \neq 0$ für unendlich viele $n \in \N$.
Hieraus folgt wiederum für unendlich viele $n \in \N$
\[ \abs{z_0(t, x)} \leq \frac {t}{\lambda n} \abs{\partial_x z_0(t,x)}, \]
was zu $z_0(t,x) = 0$ führt, ein Widerspruch zur Voraussetzung $z_0(t,x) \neq 0$.
Das impliziert $\Omega(n+1) - ( 1 - 2 \lambda ) \Omega(n) = 0$ für fast alle $n \in \N$ und mit $\Omega(0) = 1$ folgt die Behauptung.
\end{proof}

Von nun an betrachten wir $\lambda = 1 + \eta$ für ein $\eta > 0$.
Mit dem Satz \ref{satz:omega_n} und weil wir im Ansatz \eqref{eq:transport:osz:ansatz1} für jedes $k = 0,1,2$ die selbe Abhängigkeit von $(n,i)$ annehmen, folgt hiermit, dass für alle $k = 0,1,2$ und für alle $n \in \N$ und $i \in \Z$
\begin{align}\label{eq:apx:alt:prop1}
u_k(n+1, i, t_n, x_i) - u_k(n, i, t_n, x_i) + (1 + \eta)\bigl(u_k(n, i, t_n, x_i) - u_k(n, i-1, t_n, x_i) \bigr) = 0
\end{align}
erfüllt ist.
Setzt man dies nun in den Ansatz ein, so fallen die Anteile von $u_2$ in der Gleichung \eqref{eq:transport:diskret:oh2} für diskrete Ansätze weg.
Mit $\lambda = 1 + \eta$ gilt $\Omega(n) = (-1)^n (1 + 2 \eta)^n$ und wir schreiben diesen Ansatz hier nun als
\begin{align}\label{eq:transport:osz:ansatz}
v^n_i = w_0(t_n, x_i) + h  w_1(t_n, x_i) + (-1)^{i+n} (1 + 2 \eta)^n \bigl( z_0(t_n, x_i) + h z_1(t_n, x_i) \bigr) + o(h)
\end{align}
auf und setzen ihn weiter in die Bedingungen aus \eqref{eq:transport:diskret:oh} bis \eqref{eq:transport:regulaer:oh2} ein.
In der Ordnung $O(h)$ gilt nach den Rechnungen zum diskreten Ansatz die Gleichung \eqref{eq:transport:diskret:oh}, die mit $\lambda = 1 + \eta$ hier lautet:
{\small \begin{align*}
(1 + \eta) \bigl( \partial_t u_0(n+1, i, t_n, x_i) + \partial_x u_0(n, i-1, t_n, x_i) \bigr) =
\begin{split}
&- \bigl( u_1(n+1, i, t_n, x_i) - u_1(n, i, t_n, x_i) \bigr)\\
&- (1 + \eta) \bigl(u_1(n, i, t_n, x_i) - u_1(n, i-1, t_n, x_i) \bigr).
\end{split}
\end{align*} }
Die Rechte Seite dieser Gleichung verschwindet wegen der Eigenschaft \eqref{eq:apx:alt:prop1} und es folgt demnach für alle $n \in \N$, $i \in \Z$ und $h > 0$
\begin{align*}
\partial_t u_0(n+1, i, t_n, x_i) + \partial_x u_0(n, i-1, t_n, x_i) = 0.
\end{align*}
Weil sowohl $w_0$ als auch $z_0$ stetig differenzierbar sind, folgt für $n \in \N$ und $i \in \Z$
\begin{align}
\begin{split}
\partial_t u_0(n+1, i, t_n, x_i) &= \partial_t \bigl( w_0(t_n, x_i) + (-1)^{i+n+1} (1 + 2\eta)^{n+1} z_0(t_n, x_i) \bigr)\\
&= \partial_t w_0(t_n, x_i) + (-1)^{i+n} (1 + 2\eta)^n \bigl( - (1 + 2 \eta) \partial_t z_0(t_n, x_i) \bigr)
\end{split}
\end{align}
und
\begin{align}
\begin{split}
\partial_x u_0(n, i-1, t_n, x_i) &= \partial_x \bigl( w_0(t_n, x_i) + (-1)^{i+n-1} (1 + 2\eta)^n z_0(t_n, x_i) \bigr)\\
&= \partial_x w_0(t_n, x_i) + (-1)^{i+n} (1 + 2\eta)^n \bigl( - \partial_x z_0(t_n, x_i) \bigr).
\end{split}
\end{align}
In der Summe ergibt das nun
\begin{align}\label{alt:summe1}
\begin{split}
0 &= \partial_t u_0(n+1, i, t_n, x_i) + \partial_x u_0(n, i-1, t_n, x_i)\\
&= \partial_t w_0(t_n, x_i) + \partial_x w_0(t_n, x_i)\\
&\quad - (-1)^{i+n} (1 + 2\eta)^n \bigl( (1 + 2 \eta) \partial_t z_0(t_n, x_i) + \partial_x z_0(t_n, x_i) \bigr),
\end{split}
\end{align}
also
\begin{align}\label{eq:apx:beidenull1}
\partial_t w_0(t_n, x_i) + \partial_x w_0(t_n, x_i) = (-1)^{i+n} (1 + 2\eta)^n \bigl( (1 + 2 \eta) \partial_t z_0(t_n, x_i) + \partial_x z_0(t_n, x_i) \bigr).
\end{align}
Wir verwenden nun Lemma \ref{lemma:transport:diskret:konvergenz_gitter}, um zu zeigen, dass beide Seiten unabhängig voneinander Null sein müssen.
\\

\begin{satz}\label{satz:alt:beidenull}
Sei $\omega \in (0, 2 \pi)$ und $\Omega\colon \N \to \R$ unbeschränkt. 
Es seien $F$ und $g$ Funktionen aus $\Cinf$. Für alle $h > 0$ und alle $n \in \N$ und $k \in \Z$ gelte außerdem
\begin{align}\label{eq:apx:beidenull}
\begin{split}
F(t_n,x_k) &= e^{i \omega k} \Omega(n) g(t_n,x_k) \quad \text{bzw.}\\
F(t_n,x_k) &= \Ree \left[ e^{i \omega k} \Omega(n) g(t_n,x_k) \right]\\
           &= \cos(\omega k) \Omega(n) g(t_n, x_k)
\end{split}
\end{align}
Dann folgt für alle $(t,x) \in \Rp \times \R$
\begin{align*}
F(t,x) &= 0 \qquad \text{und}\\ 
g(t,x) &= 0. 
\end{align*} 
\end{satz}
\begin{proof}
In das Lemma \ref{lemma:transport:diskret:konvergenz_gitter} setzen wir für $f$ die rechte Seite der Gleichung \eqref{eq:apx:beidenull} ein, also $f(n, i, t, x) = \cos(\omega k) \Omega(n) g(t_n, x_k)$.
Dann ist $f$ für alle $n \in \N$ und $k \in \Z$ differenzierbar und die Voraussetzungen von Lemma \ref{lemma:transport:diskret:konvergenz_gitter} sind erfüllt.

Es existiert also für alle $(t,x) \in \Rp \times \R$ und $n \in \N$ ein $k(n) \in \Z$, so dass die Ungleichung
\begin{align}\label{eq:satz:lemmaglg}
\begin{split}
\abs{F(t, x) - f(n, k(n), t, x)} &\leq \frac {t}{\lambda n} \Bigl(\bigl\lvert \partial_x f(n, k(n), t, x) \bigr\rvert + \bigl\lvert \partial_x F(t, x) \bigr\rvert \Bigr)\\
\Leftrightarrow \quad \abs{F(t, x) - \cos(\omega k) \Omega(n) g(t_n, x_k)} &\leq \frac {t}{\lambda n} \Bigl(\bigl\lvert \cos(\omega k) \Omega(n) \partial_x g(t_n, x_k) \bigr\rvert + \bigl\lvert \partial_x F(t, x) \bigr\rvert \Bigr)\\
\end{split}
\end{align}
erfüllt ist. Sollte es unendlich viele $k(n) \in \Z$ derart geben, dass $\cos(\omega k) = 0$ gilt, so folgt für unendlich viele $n \in \N$ die Ungleichung $\abs{F(t, x)} \leq \frac {t}{\lambda n} \abs{\partial_x F(t,x)}$, was die Behauptung impliziert.
Angenommen das wäre nicht der Fall.
Dann impliziert die Ungleichung \eqref{eq:satz:lemmaglg} wegen $\abs{f(n, k(n), t, x)} - \abs{F(t,x)} \leq \abs{F(t, x) - f(n, k(n), t, x)}$ die Gleichung
\[ \abs{f(n, k(n), t, x)} \leq \frac {t}{\lambda n} \abs{\partial_x f(n, k(n), t, x)} +  \frac {t}{\lambda n} \bigl\lvert \partial_x F(t, x) \bigr\rvert + \abs{F(t,x)}. \]
Also folgt, weil für fast alle $n \in \N$ die Ungleichung $\cos(\omega k(n)) \neq 0$ gilt und $\Omega(n)$ unbeschränkt ist, für fast alle $n \in \N$ mit $\Omega(n) \neq 0$
\begin{align*}
\abs{\Omega(n)} \abs{g(t,x)} &\leq \frac{t}{\lambda n} \cdot \abs{\Omega(n)} \abs{\partial_x g(t,x)} + \frac {t}{\lambda n} \bigl\lvert \partial_x F(t, x) \bigr\rvert + \abs{F(t,x)}\\
\Leftrightarrow \qquad \abs{g(t,x)} &\leq \frac{t}{\lambda n} \abs{\partial_x g(t,x)} + \abs{\Omega(n)}^{-1} \left( \frac {t}{\lambda n} \bigl\lvert \partial_x F(t, x) \bigr\rvert + \abs{F(t,x)} \right),
\end{align*}
woraus, weil $\Omega(n)$ unbeschränkt ist, $g(t,x) = 0$ und somit auch $F(t,x) = 0$ für alle $(t,x) \in \Rp \times \R$ folgt.
\end{proof}
Wir wenden nun Satz \ref{satz:alt:beidenull} auf unser $\Omega(n) = (1 + 2\eta)^n$ an und benutzen, dass $(1 + 2 \eta)^n \to \infty$ für $n \to \infty$ gilt.
Später beweisen wir mit Satz \ref{satz:glgregulaer} eine analoge Aussage um solche Bedingungen in anderen Ansätzen zu bekommen, bei denen man keine solche Wachstumsbedingung für $\Omega$ hat.
Wenden wir nun Satz \ref{satz:alt:beidenull} auf Gleichung \eqref{eq:apx:beidenull1} an, so erhalten wir
\\

\noindent \emph{in der Ordnung $O(h)$:}
\begin{align}
\partial_t w_0(t, x) + \partial_x w_0(t, x) = 0 \label{eq:transport:osz:oh_A}\\
\partial_t z_0(t, x) + \frac {1} {1 + 2\eta} \partial_x z_0(t, x) = 0 \label{eq:transport:osz:oh_B}.
\end{align}

Aus den Gleichungen \eqref{eq:transport:osz:oh_A} und \eqref{eq:transport:osz:oh_B} folgt insbesondere nun auch
\begin{align}\label{alt:abl2}
\partial^2_t w_0 = \partial^2_x w_0 \quad \text{und} \quad \partial^2_t z_0 = \frac {1}{(1 + 2 \eta)^2} \partial^2_x z_0.
\end{align}
Mit den Anfangsbedingungen \eqref{eq:transport:osz:anfangsbedinungen}, insbesondere $w_0(0, x) = U(x)$ und $z(0, x) = V(x)$, kann man diese Anfangswertprobleme wie in Anhang \ref{eq:transport:analytisch:homogen} lösen und $w_0$ und $z_0$ durch
\[ w_0(t,x) = U(x-t) \quad \text{und} \quad z_0(t,x) = V\left( x - \frac{t}{1 + 2 \eta} \right) \]
bestimmen.

Ferner setzen wir den Ansatz \eqref{eq:transport:osz:ansatz} nun in die Gleichung der Ordnung $O(h^2)$ für diskrete Ansätze \eqref{eq:transport:diskret:oh2} ein.
Diese lautet mit $\lambda = 1 + \eta$

\noindent \emph{in der Ordnung $O(h^2)$}
\begin{align*}
\begin{split}
(1 + \eta) \bigl( \partial_t u_1(n+1, i, t_n, x_i)\quad\\
+ \partial_x u_1(n, i-1, t_n, x_i) \bigr)
\end{split} =& \begin{split}
&\frac {1 + \eta}{2} \partial^2_x u_0(n, i-1, t_n, x_i)\\
&- \frac{(1 + \eta)^2}{2} \partial^2_t u_0(n+1, i, t_n, x_i)\\
& 0 = \begin{cases}
- \bigl( u_2(n+1, i, t_n, x_i) - u_2(n, i, t_n, x_i) \bigr)\\
- (1 + \eta) \bigl(u_2(n, i, t_n, x_i) - u_2(n, i-1, t_n, x_i) \bigr).
\end{cases}
\end{split}
\end{align*}
Es folgt also
{\small \begin{align}\label{eq:alt:oh21}
\partial_t u_1(n+1, i, t_n, x_i) + \partial_x u_1(n, i-1, t_n, x_i) &= \frac {1}{2} \partial^2_x u_0(n, i-1, t_n, x_i) - \frac{(1 + \eta)}{2} \partial^2_t u_0(n+1, i, t_n, x_i).
\end{align}}
Wegen \eqref{alt:abl2} gilt für die rechte Seite dieser Gleichung
{\small \begin{align}\nonumber\begin{split}
&\frac {1}{2} \partial^2_x u_0(n, i-1, t_n, x_i) - \frac{(1 + \eta)}{2} \partial^2_t u_0(n+1, i, t_n, x_i)\\
&= \frac 12  \partial^2_x \bigl[ w_k(t_n, x_i) + (-1)^{i+n-1} (1 + 2\eta)^n z_k(t_n, x_i) \bigr]
 - \frac{(1 + \eta)}{2}  \partial^2_t  \bigl[ w_k(t_n, x_i) + (-1)^{i+n+1} (1 + 2\eta)^{n+1} z_k(t_n, x_i) \bigr]\\
 &= \frac 12  \bigl[ \partial^2_x w_k(t_n, x_i) - (-1)^{i+n} (1 + 2\eta)^n \partial^2_x z_k(t_n, x_i) \bigr]
 - \frac{(1 + \eta)}{2}  \bigl[ \partial^2_t w_k(t_n, x_i) - (-1)^{i+n} (1 + 2\eta)^{n+1} \partial^2_t z_k(t_n, x_i) \bigr]\\
 &= \frac 12  \bigl[ \partial^2_x w_k(t_n, x_i) - (-1)^{i+n} (1 + 2\eta)^n \partial^2_x z_k(t_n, x_i) \bigr]
 - \frac{(1 + \eta)}{2}  \bigl[ \partial^2_x w_k(t_n, x_i) - (-1)^{i+n} \frac{(1 + 2\eta)^{n}}{(1 + 2\eta)} \partial^2_x z_k(t_n, x_i) \bigr]\\
 &= - \frac {\eta} 2  \partial^2_x w_k(t_n, x_i) - \frac {1} {2} (-1)^{i+n} (1 + 2\eta)^n \partial^2_x z_k(t_n, x_i)
 + \frac{(1 + \eta)}{2} (-1)^{i+n} \frac{(1 + 2\eta)^n}{(1 + 2\eta)} \partial^2_x z_k(t_n, x_i)\\
&= - \frac {\eta} 2  \partial^2_x w_k(t_n, x_i) - \frac {1} {2} (-1)^{i+n} (1 + 2\eta)^n \partial^2_x z_k(t_n, x_i)
 + \frac{(1 + 2\eta - \eta)}{2} (-1)^{i+n} \frac{(1 + 2\eta)^n}{(1 + 2\eta)} \partial^2_x z_k(t_n, x_i)
 \end{split}\\
 &= - \frac {\eta} 2  \partial^2_x w_k(t_n, x_i) - (-1)^{i+n} (1 + 2\eta)^n \left( \frac {\eta} {2 (1 + 2\eta)} \partial^2_x z_k(t_n, x_i) \right).
\end{align}}
Für die linke Seite der Gleichung \eqref{eq:alt:oh21} gilt ganz analog wie im Fall von $O(h)$ der Gleichung \eqref{alt:summe1}
\begin{align}
\begin{split}
&\partial_t u_1(n+1, i, t_n, x_i) + \partial_x u_1(n, i-1, t_n, x_i)\\
&= \partial_t w_0(t_n, x_i) + \partial_x w_0(t_n, x_i)\\
&\qquad - (-1)^{i+n} (1 + 2\eta)^n \bigl( (1 + 2 \eta) \partial_t z_0(t_n, x_i) + \partial_x z_0(t_n, x_i) \bigr)
\end{split}
\end{align*}
Hierauf kann man wieder Satz \ref{satz:alt:beidenull} anwenden und wir gelangen mit den Anfangswerten in \eqref{eq:transport:diskret:anfangsbedinungen} zu den Anfangswertproblemen
\begin{align}
\partial_t w_1(t, x) + \partial_x w_1(t, x) &= -\frac{\eta}{2}  \partial^2_x w_0(t, x), &w_1(0,x) &= 0 \qquad \text{und}\\
\partial_t z_1(t, x) + \frac{1}{1 + 2 \eta} \partial_x z_1(t, x) &= \frac{\eta}{2 (1 + 2 \eta)^2} \partial^2_x z_0(t, x), &z_1(0,x) &= 0 \label{eq:transport:osz:oh2_B}
\end{align}

Betrachtet man den Ansatz \eqref{eq:transport:osz:ansatz} nun genauer, so erkennt man, dass $(1 + 2 \eta)^n$ für $\eta > 0$ und $n \to \infty$ unbeschränkt ist.
Das heißt, dass der Ansatz für $h \to 0$ zu jeder noch so kleinen Zeit $t > 0$ beliebig groß wird, was das instabile Verhalten ja auch widergibt. 
% Genauer gibt es für jede Zeit $t > 0$ und zu jeder Schranke $X > 0$ ein $h > 0$, so dass $\abs{v^n_i} > X$ und $n \lambda h = t_n < t$ gilt.
Allerdings bedeutet das auch, dass wir im Gegensatz zum regulären Fall die Lösungen $w_k$
\begin{align}\label{eq:transport:osz:wk_loesungen}
w_0(t, x) &= U(x - t) \qquad \text{und}\\
w_1(t, x) &= - t \frac{\eta} 2 \partial^2_x U(x - t)
\end{align}
hinnehmen, da wir uns sowieso nicht wie z.\,B. im Fall $\lambda < 1$ für das Langzeitverhalten der Lösung interessieren.

Der Faktor $\frac{\eta}{2 (1 + 2 \eta)^2}$ in Gleichung \eqref{eq:transport:osz:oh2_B} ist für $\eta > 0$ positiv!
Das heißt, dass wir hier wie die Autoren in \cite{Junk2004} den im regulären Fall aus Kapitel \ref{sec:regulaer} gezeigten Ansatz mit langsamer Zeitskala $\tau_n = h t_n$ für $z_0$ anwenden können.
Dazu nehmen wir die Existenz einer Funktion $\tilde z(t,\tau,x)$ an, für die $\tilde z(t_n, h t_n, x) = z(t_n,x_i)$ gilt.
Es folgt
\begin{align*}
\partial_t z(t_n,x_i) = \partial_t \bigl( \tilde z(t_n, h t_n, x) \bigr) = \partial_t \tilde z(t_n, h t_n, x) + h \partial_\tau \tilde z(t_n, h t_n, x)
\end{align*}
und auch
\begin{align*}
\partial_t z_0(t, \tau, x) + \frac {1} {1 + 2 \eta} \partial_x z_0(t, \tau, x) &= 0\\
\partial_t z_1(t, \tau, x) + \frac {1} {1 + 2 \eta} \partial_x z_1(t, \tau, x) &= \frac{\eta}{2 (1 + 2 \eta)^2} \partial^2_x z_0(t, \tau, x) - \partial_\tau z_0(t, \tau, x)\\
\end{align*}
und wir fordern mit dem zusätzlichen Freiheitsgrad, dass
\[ 
\partial_\tau z_0(t, \tau, x) = \frac{\eta}{2 (1 + 2 \eta)^2} \partial^2_x z_0(t, \tau, x) 
\]
gilt. Mit $z_1(0,0,x) = 0$ folgt also $z_1 = 0$ und
\begin{align*}
z_0(t, \tau, x) &= (V * G_\tau)\left(x - \frac{t}{1 + 2 \eta} \right) \\
G_\tau(y) &= \sqrt {\frac{1}{\tau} \frac {(1 + 2 \eta)^2}{2 \pi \eta}} \exp\left( - \frac{y^2}{\tau} \frac {(1 + 2 \eta)^2}{2 \pi \eta} \right)
\end{align*}
Wenn wir alle Ergebnisse zusammensetzen, gelangen wir zu dem folgendem Ausdruck
\begin{align}\label{eq:transport:osz:loesung}
v^n_i = U(x_i - t_n) + (-1)^{i+n} (1+ 2 \eta)^n (V * G_{\tau_n})\left(x_i - \frac{t_n}{1 + 2 \eta} \right) - \frac{h\eta} 2 t_n \partial^2_x U(x_i - t_n) + o(h).
\end{align}
Wir nehmen für ein konkretes Beispiel die Startwerte $U(x) = \sin(\pi x)$ und $V(x) = \eps_M$.
Wenn wir das in \eqref{eq:transport:osz:loesung} einsetzen, so gilt zunächst für alle $x, \tau \in \R:$ $(V * G_\tau)(x) = \eps_M$ und es gilt $\partial^2_x U(x_i - t_n) = -\pi^2 \sin\bigl(\pi (x_i - t_n)\bigr)$ für alle $(t_n, x_i) \in G_h$.
Ersetzt man ferner $t_n = n (1+\eta) h$ so erhalten wir
\begin{align}
v^n_i \sim \sin\bigl(\pi (x_i - t_n) \bigr)\left( 1 + n \frac {\pi^2} 2 h^2 \eta (1 + \eta) \right)  + (-1)^{i+n} (1 + 2 \eta)^n \eps_M .
\end{align}
Damit ist die folgende Fehlerabschätzung verbunden 
\begin{align}\label{eq:transport:osz:sinus_fehler}
\err^n &= \max_{i \in \Z} \abs{\sin\bigl(\pi (x_i - t_n) \bigr) - v^n_i} \nonumber \\
&= \max_{i \in \Z} \abs{\sin\bigl(\pi (x_i - t_n) \bigr) n \frac {\pi^2} 2 h^2 \eta (1 + \eta)  + (-1)^{i+n} (1 + 2 \eta)^n \eps_M} \nonumber\\
&\leq \max_{i \in \Z} \left( \abs{\sin\bigl(\pi (x_i - t_n) \bigr) n \frac {\pi^2} 2 h^2 \eta (1 + \eta)}  + \abs{(-1)^{i+n} (1 + 2 \eta)^n \eps_M} \right) \nonumber\\
&= (1 + 2 \eta)^n \eps_M + n \frac {\pi^2} 2 h^2 \eta (1 + \eta).
\end{align}
\begin{figure}
\centering
\begin{tikzpicture}
\begin{semilogyaxis}[
    title={$\err^n = \max_{i \in \Z} \abs{\sin\bigl(\pi (x_i -t_n)\bigr) - v^n_i}$},
    xlabel={Iterationen $n$},
    ylabel={maximaler Fehler},
    legend entries={$\err^n \, \eta=0.1$,$\err^n \, \eta=0.05$,$\err^n \, \eta=0.01$,$(1+2 \eta)^n\eps_M + n \frac{\pi^2}{2} h^2 (1 + \eta) \eta$, $(1+2 \eta)^n\eps_M$},
    legend style={at={(1.8,1)}}
]
\addplot[red, line width={0.8}, mark=*] table {data/max_errors_eta_0.100_h_0.001.dat};
\addplot[dg, line width={0.8}, mark=*] table {data/max_errors_eta_0.050_h_0.001.dat};
\addplot[myblue, line width={0.8}, mark=*] table {data/max_errors_eta_0.010_h_0.001.dat};
\addplot[black, line width={0.3}, mark=x] table [y index={2}] {data/max_errors_eta_0.050_h_0.001.dat};
\addplot[mygray, line width={0.5}, dotted] table [y index={3}] {data/max_errors_eta_0.010_h_0.001.dat};
\addplot[mygray, line width={0.5}, dotted] table [y index={3}] {data/max_errors_eta_0.050_h_0.001.dat};
\addplot[mygray, line width={0.5}, dotted] table [y index={3}] {data/max_errors_eta_0.100_h_0.001.dat};
\addplot[black, line width={0.3}, mark=x] table [y index={2}] {data/max_errors_eta_0.010_h_0.001.dat};
\addplot[black, line width={0.3}, mark=x] table [y index={2}] {data/max_errors_eta_0.100_h_0.001.dat};
\end{semilogyaxis}
\end{tikzpicture}
\caption{Hier vergleichen wir die maximalen Fehler der numerischen Lösungen für die Startwerte $U(x) = sin(\pi x)$, $V(x) = \eps_M$ und $h = 10^{-3}$ für $\eta = 0.1, 0.05$ und $0.001$ zur echten Lösung $u(t,x) = sin(\pi(x - t))$ mit dem geschätzten Fehler $u(t_n, x_i) - v^n_i$ in \eqref{eq:transport:osz:sinus_fehler}.}
\label{fig:transport:osz:max_error}
\end{figure}

In Abbildung \ref{fig:transport:osz:max_error} vergleichen wir den maximalen Fehler der Beispiele im Unterkapitel \ref{sec:transport:beispiel} mit dem Fehler des Ansatzes $v^n_i$ aus \eqref{eq:transport:osz:sinus_fehler}.
Dabei erkennt man, dass dieses Modell schon eine ganz gute Approximation für das Wachstum der Rundungsfehler für verschiedene $\eta > 0$ liefert.
% Es bleibt trotzdem unbefriedigend, dass man über keine Konvergenz für $h \to 0$ im klassischen Sinne sprechen kann.
Für $h \to 0$ gilt stets $h \ll \eta$.
Das bedeutet jedoch, dass dieser Ansatz hier Aussagen für beliebig große $\eta$ trifft.
% Darin liegen auch die Grenzen von \eqref{eq:transport:osz:loesung} begründet.
Wir m"ochten verstehen, wie sich das Verhalten f"ur ``kleine St"orungen'' der Stabilit"at verh"alt und dieses Ph"anomen greifen wir mit dem nächstem Unterkapitel an.