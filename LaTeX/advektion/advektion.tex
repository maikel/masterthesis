\section{Advektion}

Wir betrachten die Differentialgleichung der eindimensionalen Advektion
\begin{align}\label{eq:adv:pde}
\partial_t u(t, x) + \partial_x u(t, x) = 0, \quad u(0, x) = u_0(x).
\end{align}
Auf diese wohlbekannte partielle Differentialgleichung wenden wir das Upwindverfahren mit dem Gitter $G_h(n,i) = (t_n, x_i) = h \cdot (\lambda n, i)$ an.
Dieses Gitter ist rechteckig, falls $\lambda \neq 1$ gilt und das Verfahren ist durch die Gleichung
\begin{align}\label{eq:adv:scheme_rechnung}
\frac {v^{n+1}_i - v^n_i} {\lambda h} + \ldots
\end{align}
bzw. in der Praxis durch
\begin{align}\label{eq:adv:scheme}
v^{n+1}_i = v^n_i - \lambda (v^n_i - v^n_{i-1})
\end{align}
bestimmt.
Es ist bereits bekannt, dass das Verfahren f"ur $\lambda \leq 1$ stabil und f"ur $\lambda = 1$ sogar exakt ist.
Wir versuchen hier den instabilen Fall $\lambda > 1$ zu verstehen und nehmen in diesem Kapitel daher $\lambda = 1 + \eps$ f"ur ein $\eps > 0$ an.

\subsection{Beispiele}

Wir schauen uns die Ergebnisse des Verfahrens für die Anfangswerte $u_0(x) = \sin(\pi x)$ und $\lambda = 1 + \eps$ für verschiedene $\eps$ an.
Wir haben die Beispiele durch folgendes Skript in \emph{GNU Octave} umgesetzt
\lstinputlisting[linerange={1-33}]{advektion/beispiel.m}
und untersuchen, was für kleine $\eps$ und kleine $h$ passiert.
\lstinputlisting[firstnumber=35, firstline=35]{advektion/beispiel.m}
\hspace{-0.4cm}
\begin{tikzpicture}
\begin{axis}[
    title = {$\eps = 0.1$, $h = 0.01$, $n = 120$},
    xtick = {-2,-1,...,2},
    grid  = both,
]
\addplot[myblue] table [y index= 121] {advektion/V_sinus_eps_0.1_h_0.01.dat};
\end{axis}
\end{tikzpicture}
\begin{tikzpicture}
\begin{axis}[
    title = {$\eps = 0.1$, $h = 0.01$, $n = 185$},
    xtick = {-2,-1,...,2},
    grid  = both,
]
\addplot[myblue] table [y index= 186] {advektion/V_sinus_eps_0.1_h_0.01.dat};
\end{axis}
\end{tikzpicture}\\
\begin{tikzpicture}
\begin{axis}[
    title = {$\eps = 0.1$, $h = 0.01$, $n = 200$},
    xtick = {-2,-1,...,2},
    grid  = both,
]
\addplot[myblue] table [y index= 201] {advektion/V_sinus_eps_0.1_h_0.01.dat};
\end{axis}
\end{tikzpicture}
\begin{tikzpicture}
\begin{axis}[
    title = {$\eps = 0.1$, $h = 0.01$, $n = 220$},
    xtick = {-2,-1,...,2},
    grid  = both,
]
\addplot[myblue] table [y index= 220] {advektion/V_sinus_eps_0.1_h_0.01.dat};
\end{axis}
\end{tikzpicture}