\section{Advektion}

Wir betrachten die Differentialgleichung der eindimensionalen Advektion
\begin{align}\label{eq:adv:pde}
\partial_t u(t, x) + \partial_x u(t, x) = 0, \quad u(0, x) = U(x).
\end{align}
Auf diese wohlbekannte, partielle Differentialgleichung wenden wir das Upwindverfahren mit dem Gitter $G_h(n,i) = (t_n, x_i) = h \cdot (\lambda n, i)$ an.
Dieses Gitter ist rechteckig, falls $\lambda \neq 1$ gilt und das Verfahren ist durch die Gleichung
\begin{align}\label{eq:adv:scheme_rechnung}
\frac {v^{n+1}_i - v^n_i} {\lambda h} + \frac {v^n_i - v^n_{i-1}} h = 0
\end{align}
bzw. in der Praxis durch
\begin{align}\label{eq:adv:scheme}
v^{n+1}_i = v^n_i - \lambda (v^n_i - v^n_{i-1})
\end{align}
bestimmt.
Es ist bereits bekannt, dass das Verfahren f"ur $\lambda \leq 1$ stabil und f"ur $\lambda = 1$ sogar exakt ist.
Wir versuchen hier den instabilen Fall $\lambda > 1$ zu verstehen und nehmen in diesem Kapitel daher $\lambda = 1 + \eps$ f"ur ein $\eps > 0$ an.

\subsection {Analytische L"osung}

F"ur die kommende Asymptotik ist es sinnvoll, die echte Lösung von \eqref{eq:adv:pde} zu kennen.
Die Lösung lässt sich analytisch mithilfe der Methode der Charakteristiken bestimmen.
Sei $u$ eine Lösung von \eqref{eq:adv:pde}.
Ferner sei $\phi\colon \tau \mapsto (\tau, x(\tau))$ eine Charakteristik mit $\phi(0) = (0, x_0)$.
$\phi$ ist eine Raumzeit-Kurve auf welcher $u$ konstant ist.
Demnach gilt nach der Kettenregel
\[ 0 = \frac {\dd}{\dd \tau} u(\phi(\tau)) = \partial_t u(\phi(\tau))  + \frac {\dd x(\tau)} {\dd \tau} \partial_x u(\phi(\tau)) \]
und wegen \eqref{eq:adv:pde} muss $\dot x(\tau) = 1$ gelten.
Das impliziert mit $x(0) = x_0$ die Lösung $x(t) = x_0 + t$.
Da $u(0, x_0) = u_0(x_0)$ und $x_0 = x(t) - t$ für alle $t > 0$ gilt, folgt insgesamt
\begin{align}\label{eq:adv:solution}
u(t, x) = u(0, x - t) = u_0(x - t).
\end{align}
Die wahre Lösung transportiert also nur die Anfangsdaten in die positive Raumrichtung.

Wenn wir eine Lösung $v$ von \eqref{eq:adv:pde} haben, so können wir Anfangswertproblem
\begin{align}\label{eq:adv:pdei}
\partial u(t,x) + \partial_x u(t,x) = v(t,x), \quad u(0,x) = 0
\end{align}
lösen, indem wir einfach $u(t,x) = t v(t,x)$ setzen. Denn dann gilt
{\small
\[ \partial_t u(t,x) = v(t,x) + t \partial_t v(t,x) = v(t,x) - t \partial_x v(t,x) = v(t,x) - \partial_x \bigl( t v(t,x) \bigr) = v(t,x) - \partial_x u(t,x), \]
}
womit zeigt ist, dass $u$ die Gleichung \eqref{eq:adv:pdei} erfüllt.


\subsection{Beispiele}

Wir schauen uns die Ergebnisse des Verfahrens für die Anfangswerte $u_0(x) = \sin(\pi x)$ und $\lambda = 1 + \eps$ für verschiedene $\eps$ an.
Wir haben die Beispiele durch folgendes Skript in \emph{GNU Octave} umgesetzt
\lstinputlisting[linerange={1-33}]{advektion/beispiel.m}
und untersuchen, was für kleine $\eps$ und kleine $h$ passiert.\\

\lstinputlisting[firstnumber=35, firstline=35]{advektion/beispiel.m}
\begin{figure}
\centering
% \begin{tikzpicture}[scale=0.58]
% \begin{axis}[
%     height = 0.6\textwidth,
%     width = 0.6\textwidth,
%     title = {$n = 120$},
%     xtick = {-2,-1,...,2},
%     grid  = both,
% ]
% \addplot[myblue] table [y index= 1] {data/V_sinus_eps_0.1_h_0.01.dat};
% \end{axis}
% \end{tikzpicture}
\begin{tikzpicture}[scale=0.58]
\begin{axis}[
    height = 0.6\textwidth,
    width = 0.6\textwidth,
    title = {$n = 185$},
    xtick = {-2,-1,...,2},
    grid  = both,
]
\addplot[myblue] table [y index= 2] {data/V_sinus_eps_0.1_h_0.01.dat};
\end{axis}
\end{tikzpicture}
\begin{tikzpicture}[scale=0.58]
\begin{axis}[
    height = 0.6\textwidth,
    width = 0.6\textwidth,
    title = {$n = 200$},
    xtick = {-2,-1,...,2},
    grid  = both,
]
\addplot[myblue] table [y index= 3] {data/V_sinus_eps_0.1_h_0.01.dat};
\end{axis}
\end{tikzpicture}
\begin{tikzpicture}[scale=0.58]
\begin{axis}[
    height = 0.6\textwidth,
    width = 0.6\textwidth,
    title = {$n = 220$},
    xtick = {-2,-1,...,2},
    grid  = both,
]
\addplot[myblue] table [y index= 4] {data/V_sinus_eps_0.1_h_0.01.dat};
\end{axis}
\end{tikzpicture}
\caption{Beispiel für $\eps = 10^{-1}$ und $h=10^{-2}$}
\label{fig:adv:e-1h-2}
\end{figure}

\begin{figure}
\centering
% \begin{tikzpicture}[scale=0.58]
% \begin{axis}[
%     height = 0.6\textwidth,
%     width = 0.6\textwidth,
%     title = {$n = 120$},
%     xtick = {-2,-1,...,2},
%     grid  = both,
% ]
% \addplot[myblue] table [y index= 1] {data/V_sinus_eps_0.1_h_0.001.dat};
% \end{axis}
% \end{tikzpicture}
\begin{tikzpicture}[scale=0.58]
\begin{axis}[
    height = 0.6\textwidth,
    width = 0.6\textwidth,
    title = {$n = 185$},
    xtick = {-2,-1,...,2},
    grid  = both,
]
\addplot[myblue] table [y index= 2] {data/V_sinus_eps_0.1_h_0.001.dat};
\end{axis}
\end{tikzpicture}
\begin{tikzpicture}[scale=0.58]
\begin{axis}[
    height = 0.6\textwidth,
    width = 0.6\textwidth,
    title = {$n = 200$},
    xtick = {-2,-1,...,2},
    grid  = both,
]
\addplot[myblue] table [y index= 3] {data/V_sinus_eps_0.1_h_0.001.dat};
\end{axis}
\end{tikzpicture}
\begin{tikzpicture}[scale=0.58]
\begin{axis}[
    height = 0.6\textwidth,
    width = 0.6\textwidth,
    title = {$n = 220$},
    xtick = {-2,-1,...,2},
    grid  = both,
]
\addplot[myblue] table [y index= 4] {data/V_sinus_eps_0.1_h_0.001.dat};
\end{axis}
\end{tikzpicture}
\caption{Beispiel für $\eps = 10^{-1}$ und $h=10^{-3}$}
\label{fig:adv:e-1h-3}
\end{figure}

\begin{figure}
\centering
% \begin{tikzpicture}[scale=0.58]
% \begin{axis}[
%     height = 0.6\textwidth,
%     width = 0.6\textwidth,
%     title = {$n = 240$},
%     xtick = {-2,-1,...,2},
%     grid  = both,
% ]
% \addplot[myblue] table [y index= 1] {data/V_sinus_eps_0.05_h_0.01.dat};
% \end{axis}
% \end{tikzpicture}
\begin{tikzpicture}[scale=0.58]
\begin{axis}[
    height = 0.6\textwidth,
    width = 0.6\textwidth,
    title = {$n = 370$},
    xtick = {-2,-1,...,2},
    grid  = both,
]
\addplot[myblue] table [y index= 2] {data/V_sinus_eps_0.05_h_0.01.dat};
\end{axis}
\end{tikzpicture}
\begin{tikzpicture}[scale=0.58]
\begin{axis}[
    height = 0.6\textwidth,
    width = 0.6\textwidth,
    title = {$n = 400$},
    xtick = {-2,-1,...,2},
    grid  = both,
]
\addplot[myblue] table [y index= 3] {data/V_sinus_eps_0.05_h_0.01.dat};
\end{axis}
\end{tikzpicture}
\begin{tikzpicture}[scale=0.58]
\begin{axis}[
    height = 0.6\textwidth,
    width = 0.6\textwidth,
    title = {$n = 440$},
    xtick = {-2,-1,...,2},
    grid  = both,
]
\addplot[myblue] table [y index= 4] {data/V_sinus_eps_0.05_h_0.01.dat};
\end{axis}
\end{tikzpicture}
\caption{Beispiel für $\eps = 5\cdot 10^{-2}$ und $h=10^{-2}$}
\label{fig:adv:e5-2h-2}
\end{figure}

\begin{figure}
\centering
\begin{tikzpicture}[scale=0.58]
\begin{axis}[
    height = 0.6\textwidth,
    width = 0.6\textwidth,
    title = {$n = 370$},
    xtick = {-2,-1,...,2},
    grid  = both,
]
\addplot[myblue] table [y index= 2] {data/V_sinus_eps_0.05_h_0.001.dat};
\end{axis}
\end{tikzpicture}
\begin{tikzpicture}[scale=0.58]
\begin{axis}[
    height = 0.6\textwidth,
    width = 0.6\textwidth,
    title = {$n = 400$},
    xtick = {-2,-1,...,2},
    grid  = both,
]
\addplot[myblue] table [y index= 3] {data/V_sinus_eps_0.05_h_0.001.dat};
\end{axis}
\end{tikzpicture}
\begin{tikzpicture}[scale=0.58]
\begin{axis}[
    height = 0.6\textwidth,
    width = 0.6\textwidth,
    title = {$n = 440$},
    xtick = {-2,-1,...,2},
    grid  = both,
]
\addplot[myblue] table [y index= 4] {data/V_sinus_eps_0.05_h_0.001.dat};
\end{axis}
\end{tikzpicture}
\caption{Beispiel für $\eps = 5\cdot 10^{-2}$ und $h=10^{-3}$}
\label{fig:adv:e5-2h-3}
\end{figure}

\caption{Beispiel für $\eps = 5\cdot 10^{-2}$ und $h=10^{-2}$}
\label{fig:adv:e5-2h-2}
\end{figure}

Vergleicht man nun die Plots auf Abbildung~\ref{fig:adv:e-1h-2} und mit den Plots auf Abbildung~\ref{fig:adv:e-1h-3}, so erkennt man, dass die Amplitude der Oszillation scheinbar unabh"angig von $h$ mit der Anzahl der Iterationen $n$ w"achst.
Halbiert man jedoch den Wert für $\eps$, so halbiert sich auch die Rate, um die die Amplitude wächst.
Wir hoffen dieses Verhalten in unseren Approximationen wiederzufinden.

% \subsection{Regul"arer Ansatz}

% Die Idee ist, dass die Gitterfunktion $v\colon G_h \to \R$ durch differenzierbare Abbildungen $u_i\colon \R^2 \to \R$ dargestellt werden kann, welche den Gitterpunkten $(t_n, x_i) \in G_h$ ausgewertet werden.
% Wir rechnen hier zun"achst den ``regulären Ansatz'' vor, welcher vielen weiteren Rechnungen zu Grunde liegt. Dieser lautet
% \begin{align}\label{eq:adv:regular:ansatz}
% v^n_i = u_0(t_n, x_i) + h u_1(t_n, x_i) + o(h).
% \end{align}
% Außerdem erinnern wir uns, dass $t_{n+1} = t_n + \lambda h$ und $x_{i-1} = x_i - h$ gilt.
% Wir rechnen nun die einzelnen Summanden aus \eqref{eq:adv:scheme_rechnung} aus.
% Es gilt
% \begin{align}\label{eq:adv:regular:sum1}
% \frac {v^{n+1}_i - v^n_i}{\lambda h}
% &= \partial_t u_0(t_n, x_i) + \frac {\lambda h}{2} \partial^2_t u_0(t_n, x_i) + h \partial_t u_1(t_n, x_i) + o(h). 
% \end{align}
% und
% \begin{align}\label{eq:adv:regular:sum2}
% \frac {v^n_i - v^n_{i-1}}{h} = \partial_x u_0(t_n, x_i) - \frac{h}{2} \partial^2_x u_0(t_n, x_i) + h \partial_x u_1(t_n, x_i) + o(h).
% \end{align}
% Setzt man dies in \eqref{eq:adv:scheme_rechnung} ein, so erhält man ein Gleichungssystem, welches für alle Zeit-Raum-Punkte $(t_n, x_i)$ und beliebiges $h > 0$ gilt.
% Sei $(h_k)_{k \in \N}$ eine Nullfolge.
% Dann gibt es für jedes Tupel $(t, x)$ eine Folge $(t_k, x_k)$ mit $(t_k, x_k) \in G_{h_k}$ und  $(t_k, x_k) \to (t,x)$ für $k \to \infty$.
% In $O(1)$ bekommt man dann das Anfangswertproblem
% \[ \partial_t u_0(t,x) + \partial_x u_0(t, x) = 0, \quad u(0, x) = U(x). \]
% Foglich ist $u_0$ direkt die Lösung von \eqref{eq:adv:pde} und durch $u_0(t,x) = U(x-t)$ gegeben.
% Wir können mithilfe von $u_1$ den Fehler des Verfahrens angeben.
% In $O(h)$ erhalten wir eine bestimmende Gleichung f"ur $u_1$
% \[ \partial_t u_1(t,x) + \partial_x u_1(t, x) = -\frac{\eps}{2} \partial^2_x u_0(t, x), \quad u(0, x) = 0. \]
% Hierbei handelt es sich um die inhomogene Gleichung wie bei \eqref{eq:adv:pdei}.
% $u_1$ ist deshalb durch $u_1(t,x) = - t \frac{\eps}{2} u_0(x - t)$  bestimmt.
% $u_1$ verletzt somit die Sublinear-Growth Bedingung und d.\,h. unser Ansatz kann nur für kleine Zeiten $t$ stimmen.
% Zusammenfassend folgt mit diesem Ansatz
% \[ v^n_i = U(x_i - t_n) - \frac{h \eps}{2} t_n U(x_i - t_n) + o(h). \]

\subsection {Oszillatorischer Ansatz}

Wir führen hier Zeit-Raum-Koordinaten auf einer kurzen Skala ein.
Genau genommen machen wir unsere Ansatzfunktionen zusätzlich von den diskreten Gitterkoordinaten $(n,i)$ abhängig.
Es gilt zwar $n = \frac {t_n} {\lambda h}$ und $i = \frac {x_i}{h}$, jedoch soll unsere Annahme sein, dass der Ansatz unstetig in $n$ und $i$ ist.
Möchte man jedoch einen Punkt $(t,x)$ in der Raumzeit approximieren, so gilt für $h \to 0$ jedoch immer $n,i \to \infty$ und dies liefert uns zusätzliche Bedingungen.
Da unsere Gleichung linear ist, nehmen wir an, dass wir die Gitterfunktion als Summe einer Lösung und kleinen Oszillationen schreiben können.
Folglich lautet unser Ansatz dieses mal
\begin{align}
v^n_i = u_0(t_n, x_i) + h u_1(n, i, t_n, x_i) + h^2 u_2(n, i, t_n, x_i) + o(h^2)
\end{align}
und wir möchten hier betonen, dass wir, wie immer, davon ausgehen, dass die Entwicklung asymptotisch geordnet ist.
Das heißt z.\,B., dass $h u_1 \in o(u_0)$ für $h \to 0$ gilt.
Dies lässt sich wiederum zu
\begin{align}
\lim_{h \to 0} \: h \frac{u_1(n, i, t_n, x_i)}{u_0(t_n, x_i)} = 0
\end{align}
umschreiben.
Zunächst fällt auf, dass wir hier Terme bis $h^2$ entwickeln.
Durch die kurze Zeit- und Ortskala wirken Terme in einer Ordnung niedriger.
Daher werden Terme von $u_2$ Gleichungen in $O(h)$ beeinflussen und müssen betrachtet werden.
Wir rechnen hier nun die einzelnen Summanden von \eqref{eq:adv:scheme_rechnung} aus:
{\small
\begin{align} \label{eq:adv:osz:diff1}
\begin{split}
\frac {v^{n+1}_i - v^n_i} {\lambda h}
&= \partial_t u_0(t_n, x_i) + \frac {\lambda h}{2} \partial^2_t u_0(t_n, x_i)\\
&\qquad + \frac 1 \lambda \bigl( u_1(n+1, i, t_n, x_i) - u_1(n, i, t_n, x_i) \bigr)\\
&\qquad + h \partial_t u_1(n+1, i, t_n, x_i) + \frac {\lambda h^2}{2} \partial^2_t u_1(n+1, i, t_n, x_i)\\
&\qquad + \frac h \lambda \bigl( u_2(n+1, i, t_n, x_i) - u_2(n, i, t_n, x_i) \bigr)\\
&\qquad + h^2 \partial_t u_2(n+1, i, t_n, x_i) + o(h^2).
\end{split}
\end{align}
} und
{\small 
\begin{align} \label{eq:adv:osz:diff2}
\begin{split}
\frac {v^n_i - v^n_{i-1}} h 
&= \partial_x u_0(t_n, x_i) - \frac {h}{2} \partial^2_x u_0(t_n, x_i)\\
&\qquad + u_1(n, i, t_n, x_i) - u_1(n, i-1, t_n, x_i)\\
&\qquad + h \partial_x u_1(n, i-1, t_n, x_i) - \frac {h^2}{2} \partial^2_x u_1(n, i-1, t_n, x_i)\\
&\qquad + h \bigl( u_2(n, i, t_n, x_i) - u_2(n, i-1, t_n, x_i) \bigr)\\
&\qquad + h^2 \partial_t u_2(n, i-1, t_n, x_i) + o(h^2).
\end{split}
\end{align}
}
Setzt man \eqref{eq:adv:osz:diff1} und \eqref{eq:adv:osz:diff2} in \eqref{eq:adv:scheme_rechnung} ein, liefert uns dies in $O(1)$ bei festem $h$ die Gleichung
\begin{align}\label{eq:adv:osz:o1}
\begin{split}
\partial_t u_0(t_n, x_i) + \partial_x u_0(t_n, x_i) &= - \frac 1 \lambda \bigl( u_1(n+1, i, t_n, x_i) - u_1(n, i, t_n, x_i) \bigr)\\
&\qquad - \bigl(u_1(n, i, t_n, x_i) - u_1(n, i-1, t_n, x_i) \bigr)
\end{split}
\end{align}
Hieraus schließen wir nun, dass die rechte Seite dieser Gleichung \eqref{eq:adv:osz:o1} verschwinden muss,
indem wir folgendes Lemma beweisen.

\begin{lemma}
Sei $f\colon \N \times \R^+_0 \to \R$ eine Abbildung, so dass $f(n, \,\cdot\,)$ für festes $n \in \N$ differenzierbar und beschränkt ist.
Sei weiter $G_h \subset \R^+_0$ ein äquidistantes Gitter mit $G_h(n) = t^h_n = n h$, für $n \in \N$.
Wenn für alle $h > 0$ und somit alle Gitter $G_h$ ein $F \in C^1\left(\R^+_0 \right)$ mit
\[ F(t^h_n) = f(n+1, t^h_n) - f(n, t^h_n) \]
existiert, so folgt $F = 0$.
\end{lemma}
\begin{proof}
Es gilt $F(t^h_{n+1}) = F(t^h_n) + h F'(t^h_n)$ und somit
\begin{align}
h F'(t^h_n) &= f(n+2, t^h_{n+1}) - f(n+1, t^h_{n+1}) - f(n+1, t^h_n) + f(n, t^h_n)\\
&= f(n+2, t^h_n) - f(n+1, t^h_n)
\end{align}
\end{proof}