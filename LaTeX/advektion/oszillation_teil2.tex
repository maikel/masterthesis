\subsection {Oszillatorischer Ansatz Teil 2}\label{sec:osz2}

Hier präzisieren wir unseren Ansatz \eqref{eq:adv:osz:ansatz} aus dem letztem Unterkapitel. % \ref{sec:osz1}.
Wir gehen nun davon aus, dass Oszillationen auf Gitterebene vorhanden sind und untersuchen, wie sich die Amplitude in der Zeit ausbreitet.
Daher wählen wir für $\phi_1$ und $\phi_2$ aus \eqref{eq:adv:osz:ansatz} nun konkreter 
\[ 
\phi_1(n, i, t_n, x_i) + h \phi_2(n, i, t_n, x_i) 
= \Omega(n) (-1)^i \Bigl[ \Psi_1(t_n, x_i) + h \Psi_2(t_n, x_i) \Bigr]
\]
Setzt man dies in die vorigen Rechnungen ein, so erhält man anstelle von Gleichung \eqref{eq:adv:osz:o1}

\vspace{0.4cm}
\noindent \textbf{In der Ordnung $O(1)$:}
\begin{align}\label{eq:osz2:o1} % \nonumber
% \begin{split}
\partial_t u_0(t_n, x_i) + \partial_x u_0(t_n, x_i) % &=
% - \frac 1 \lambda \bigl( \Omega(n+1) (-1)^i \Psi_1(t_n, x_i) - \Omega(n) (-1)^i \Psi_1(t_n, x_i) \bigr)\\
% &\qquad - \bigl( \Omega(n) (-1)^i \Psi_1(t_n, x_i) - \Omega(n) (-1)^{i-1} \Psi_1(t_n, x_i) \bigr)\\
% &= - \frac 1 \lambda (-1)^i \Psi_1(t_n, x_i) \Bigl[ \Omega(n+1) - \Omega(n) + \lambda \bigl( \Omega(n) + \Omega(n) \bigr) \Bigr]
% \end{split}\\
&= - \frac 1 \lambda (-1)^i \Psi_1(t_n, x_i) \Bigl[ \Omega(n+1) - ( 1 - 2 \lambda ) \Omega(n) \Bigr].
\end{align}
Gleichung \eqref{eq:osz2:o1} und die Sublinear-Growth Bedingung reichen schon aus, um $\Omega(n)$ zu bestimmen.

\begin{lemma}\label{lem:osz2:omega_n}
Es gelte $\lambda > \frac 12$, für alle $h > 0$, $n \in \N$ und $i \in \Z$ gelte die Gleichung \eqref{eq:osz2:o1} und es gelte die Sublinear-Grwoth Bedingung \eqref{eq:osz:sublineargrowth}.
Wenn außerdem $\Omega(0) = 1$ gilt und ein $h > 0$ mit $\Psi_1(\lambda h, h) \neq 0$ existiert, dann folgt
\begin{align*}
\Omega(n+1) - ( 1 - 2 \lambda ) \Omega(n) = 0
\end{align*}
und
\begin{align*}
\Omega(n) = (1 - 2 \lambda)^n.
\end{align*}
\end{lemma}
\begin{proof}
Sei $h > 0$ derart, dass $\Psi(\lambda h, h) \neq 0$ gilt. Dann gilt $t_1 = \lambda h$ und $x_1 = h$, also
\begin{align}\label{eq:osz2:n1}
\partial_t u_0(\lambda h, h) + \partial_x u_0(\lambda h, h) &= \frac 1 \lambda \Psi_1(\lambda h, h) \Bigl[ \underbrace{\Omega(2) - ( 1 - 2 \lambda ) \Omega(1)}_{=: A} \Bigr]
\end{align}
Für jedes $n \in \N$ betrachte $h_n = \frac h n$. Dann gelten $t^{h_n}_n = \lambda h$ und $x^{h_n}_n = h$ und somit folgt f"ur jedes $n \in \N$
\begin{align}\label{eq:osz2:n}
\partial_t u_0(\lambda h, h) + \partial_x u_0(\lambda h, h) &= - \frac 1 \lambda (-1)^n \Psi_1(\lambda h, h) \Bigl[ \Omega(n+1) - ( 1 - 2 \lambda ) \Omega(n) \Bigr]
\end{align}
Da \eqref{eq:osz2:n1} und \eqref{eq:osz2:n} beide gleich sind, folgt unter der Annahme, dass {\color{red}$\Psi_1(\lambda h, h) \neq 0$} gilt
\begin{align}\label{eq:osz:omegan}
-A (-1)^n = \Omega(n+1) - ( 1 - 2 \lambda ) \Omega(n)
\end{align}
Und das impliziert mit {\color{red} $\Omega(0) = 1$} wiederum
\begin{align}\nonumber
\Omega(n+1) &= -A \sum_{k=1}^n (-1)^{n-k} (1 - 2 \lambda)^k + (1 - 2 \lambda)^n\\
&= -A (- 2 \lambda)^n - (-A)^n + (1 - 2 \lambda)^n.
\end{align}
Hieraus folgt für $2 \lambda > 1$, also $2\lambda = 1 + K$ für ein $K > 0$
\begin{align*}
\abs{\Omega(n+1)} &= \abs{-A (- 2 \lambda)^n -(-A)^n + (1 - 2 \lambda)^n}\\
&\geq \bigl\lvert \abs{-A (- 2 \lambda)^n} - \abs{-(-A)^n + (1 - 2 \lambda)^n} \bigr\rvert\\
&\geq (2 \lambda)^n \abs{A}\\
&= (1 + K)^n \abs{A}\\
&\geq n K  \abs{A}\\
\end{align*}
und setzt man das in die Sublinear-Growth Bedingung \eqref{eq:osz:sublineargrowth} ein, so erhalten wir
\begin{align*}
0 = \lim_{h \to 0} \: h \frac{\abs{\phi_1(n, i, t_n, x_i)}}{\abs{u_0(t_n, x_i)}}
&= \lim_{h \to 0} \: h \frac{\abs{\Omega(n) (-1)^i \Psi_1(t_n, x_i)}}{\abs{u_0(t_n, x_i)}}\\
&= \lim_{h \to 0} \: h \abs{\Omega(n)} \frac{\abs{\Psi_1(t_n, x_i)}}{\abs{u_0(t_n, x_i)}}\\
&= \frac{\abs{\Psi_1(t, x)}}{\abs{u_0(t, x)}} \cdot \lim_{h \to 0} \: h \abs{\Omega(n)}\\
&\geq \frac{\abs{\Psi_1(t, x)}}{\abs{u_0(t, x)}} \cdot \lim_{h \to 0} \: h \cdot (n-1) K \abs{A}\\
&= \frac{\abs{\Psi_1(t, x)}}{\abs{u_0(t, x)}} \cdot \lim_{h \to 0} \: h \cdot \frac{t_{n-1}}{\lambda h} K \abs{A}\\
&= \frac{\abs{\Psi_1(t, x)}}{\abs{u_0(t, x)}} \cdot \frac{K}{\lambda} t \abs{A}\\
\end{align*}
Das impliziert $A = 0$ und mit \eqref{eq:osz:omegan} gilt für alle $n \in \N$
\[ 0 = \Omega(n+1) - (1 - 2\lambda) \Omega(n) \]
und mit $\Omega(0) = 1$
\[ \Omega(n) = (1 - 2\lambda)^n. \qedhere \]
\end{proof}
\noindent Mit dieser neuen Erkenntnis fassen wir den Ansatz nun zu
{\footnotesize
\begin{align*}
v^n_i &= u_0(t_n, x_i) + h \bigl(u_1(t_n, x_i) + (1 - 2\lambda)^n (-1)^i \Psi_1(t_n, x_i)\bigr) + h^2 \bigl( u_2(t_n, x_i) + (1 - 2\lambda)^n (-1)^i \Psi_2(t_n, x_i) \bigr) + o(h^2)
\end{align*}
}
zusammen. Anstelle der Gleichungen \eqref{eq:adv:osz:o1}, \eqref{eq:adv:osz:oh} und \eqref{eq:adv:osz:oh2} erhalten wir
