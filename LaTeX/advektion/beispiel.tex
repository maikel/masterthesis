\subsection{Beispiel}

Wir schauen uns die Ergebnisse des Verfahrens für die Anfangswerte $u_0(x) = \sin(\pi x)$ und $\lambda = 1 + \eps$ für verschiedene $\eps$ an.
Wir haben die Beispiele durch folgendes Skript in \emph{GNU Octave} umgesetzt und untersuchen, was für kleine $\eps$ und kleine $h$ passiert.
\lstinputlisting[linerange={1-33}]{advektion/beispiel.m}
% \lstinputlisting[firstnumber=35, firstline=35]{advektion/beispiel.m}
\begin{figure}
\centering
% \begin{tikzpicture}[scale=0.58]
% \begin{axis}[
%     height = 0.6\textwidth,
%     width = 0.6\textwidth,
%     title = {$n = 120$},
%     xtick = {-2,-1,...,2},
%     grid  = both,
% ]
% \addplot[myblue] table [y index= 1] {data/V_sinus_eps_0.1_h_0.01.dat};
% \end{axis}
% \end{tikzpicture}
\begin{tikzpicture}[scale=0.58]
\begin{axis}[
    height = 0.6\textwidth,
    width = 0.6\textwidth,
    title = {$n = 185$},
    xtick = {-2,-1,...,2},
    grid  = both,
]
\addplot[myblue] table [y index= 2] {data/V_sinus_eps_0.1_h_0.01.dat};
\end{axis}
\end{tikzpicture}
\begin{tikzpicture}[scale=0.58]
\begin{axis}[
    height = 0.6\textwidth,
    width = 0.6\textwidth,
    title = {$n = 200$},
    xtick = {-2,-1,...,2},
    grid  = both,
]
\addplot[myblue] table [y index= 3] {data/V_sinus_eps_0.1_h_0.01.dat};
\end{axis}
\end{tikzpicture}
\begin{tikzpicture}[scale=0.58]
\begin{axis}[
    height = 0.6\textwidth,
    width = 0.6\textwidth,
    title = {$n = 220$},
    xtick = {-2,-1,...,2},
    grid  = both,
]
\addplot[myblue] table [y index= 4] {data/V_sinus_eps_0.1_h_0.01.dat};
\end{axis}
\end{tikzpicture}
\caption{Beispiel für $\eps = 10^{-1}$ und $h=10^{-2}$}
\label{fig:adv:e-1h-2}
\end{figure}

\begin{figure}
\centering
% \begin{tikzpicture}[scale=0.58]
% \begin{axis}[
%     height = 0.6\textwidth,
%     width = 0.6\textwidth,
%     title = {$n = 120$},
%     xtick = {-2,-1,...,2},
%     grid  = both,
% ]
% \addplot[myblue] table [y index= 1] {data/V_sinus_eps_0.1_h_0.001.dat};
% \end{axis}
% \end{tikzpicture}
\begin{tikzpicture}[scale=0.58]
\begin{axis}[
    height = 0.6\textwidth,
    width = 0.6\textwidth,
    title = {$n = 185$},
    xtick = {-2,-1,...,2},
    grid  = both,
]
\addplot[myblue] table [y index= 2] {data/V_sinus_eps_0.1_h_0.001.dat};
\end{axis}
\end{tikzpicture}
\begin{tikzpicture}[scale=0.58]
\begin{axis}[
    height = 0.6\textwidth,
    width = 0.6\textwidth,
    title = {$n = 200$},
    xtick = {-2,-1,...,2},
    grid  = both,
]
\addplot[myblue] table [y index= 3] {data/V_sinus_eps_0.1_h_0.001.dat};
\end{axis}
\end{tikzpicture}
\begin{tikzpicture}[scale=0.58]
\begin{axis}[
    height = 0.6\textwidth,
    width = 0.6\textwidth,
    title = {$n = 220$},
    xtick = {-2,-1,...,2},
    grid  = both,
]
\addplot[myblue] table [y index= 4] {data/V_sinus_eps_0.1_h_0.001.dat};
\end{axis}
\end{tikzpicture}
\caption{Beispiel für $\eps = 10^{-1}$ und $h=10^{-3}$}
\label{fig:adv:e-1h-3}
\end{figure}

\begin{figure}
\centering
% \begin{tikzpicture}[scale=0.58]
% \begin{axis}[
%     height = 0.6\textwidth,
%     width = 0.6\textwidth,
%     title = {$n = 240$},
%     xtick = {-2,-1,...,2},
%     grid  = both,
% ]
% \addplot[myblue] table [y index= 1] {data/V_sinus_eps_0.05_h_0.01.dat};
% \end{axis}
% \end{tikzpicture}
\begin{tikzpicture}[scale=0.58]
\begin{axis}[
    height = 0.6\textwidth,
    width = 0.6\textwidth,
    title = {$n = 370$},
    xtick = {-2,-1,...,2},
    grid  = both,
]
\addplot[myblue] table [y index= 2] {data/V_sinus_eps_0.05_h_0.01.dat};
\end{axis}
\end{tikzpicture}
\begin{tikzpicture}[scale=0.58]
\begin{axis}[
    height = 0.6\textwidth,
    width = 0.6\textwidth,
    title = {$n = 400$},
    xtick = {-2,-1,...,2},
    grid  = both,
]
\addplot[myblue] table [y index= 3] {data/V_sinus_eps_0.05_h_0.01.dat};
\end{axis}
\end{tikzpicture}
\begin{tikzpicture}[scale=0.58]
\begin{axis}[
    height = 0.6\textwidth,
    width = 0.6\textwidth,
    title = {$n = 440$},
    xtick = {-2,-1,...,2},
    grid  = both,
]
\addplot[myblue] table [y index= 4] {data/V_sinus_eps_0.05_h_0.01.dat};
\end{axis}
\end{tikzpicture}
\caption{Beispiel für $\eps = 5\cdot 10^{-2}$ und $h=10^{-2}$}
\label{fig:adv:e5-2h-2}
\end{figure}

\begin{figure}
\centering
\begin{tikzpicture}[scale=0.58]
\begin{axis}[
    height = 0.6\textwidth,
    width = 0.6\textwidth,
    title = {$n = 370$},
    xtick = {-2,-1,...,2},
    grid  = both,
]
\addplot[myblue] table [y index= 2] {data/V_sinus_eps_0.05_h_0.001.dat};
\end{axis}
\end{tikzpicture}
\begin{tikzpicture}[scale=0.58]
\begin{axis}[
    height = 0.6\textwidth,
    width = 0.6\textwidth,
    title = {$n = 400$},
    xtick = {-2,-1,...,2},
    grid  = both,
]
\addplot[myblue] table [y index= 3] {data/V_sinus_eps_0.05_h_0.001.dat};
\end{axis}
\end{tikzpicture}
\begin{tikzpicture}[scale=0.58]
\begin{axis}[
    height = 0.6\textwidth,
    width = 0.6\textwidth,
    title = {$n = 440$},
    xtick = {-2,-1,...,2},
    grid  = both,
]
\addplot[myblue] table [y index= 4] {data/V_sinus_eps_0.05_h_0.001.dat};
\end{axis}
\end{tikzpicture}
\caption{Beispiel für $\eps = 5\cdot 10^{-2}$ und $h=10^{-3}$}
\label{fig:adv:e5-2h-3}
\end{figure}

Vergleicht man nun die Plots in Abbildung~\ref{fig:adv:e-1h-2} und mit den Plots auf Abbildung~\ref{fig:adv:e-1h-3}, so erkennt man, dass die Amplitude der Oszillation scheinbar unabh"angig von $h$ mit der Anzahl der Iterationen $n$ w"achst.
Halbiert man jedoch den Wert für $\eps$, so ändert sich auch die Rate, um die die Amplitude wächst.
Wir hoffen dieses Verhalten in unseren Approximationen wiederzufinden und Abschätzungen für die Amplitude zu finden.