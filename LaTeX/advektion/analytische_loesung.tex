\subsection {Analytische L"osung}

F"ur die kommende Asymptotik ist es sinnvoll, die echte Lösung von \eqref{eq:adv:pde} zu kennen.
Die Lösung lässt sich analytisch mithilfe der Methode der Charakteristiken bestimmen.
Sei $u$ eine Lösung von \eqref{eq:adv:pde}.
Ferner sei $\phi\colon \tau \mapsto (\tau, x(\tau))$ eine Charakteristik mit $\phi(0) = (0, x_0)$.
$\phi$ ist eine Raumzeit-Kurve auf welcher $u$ konstant ist.
Demnach gilt nach der Kettenregel
\[ 0 = \frac {\dd}{\dd \tau} u(\phi(\tau)) = \partial_t u(\phi(\tau))  + \frac {\dd x(\tau)} {\dd \tau} \partial_x u(\phi(\tau)) \]
und wegen \eqref{eq:adv:pde} muss $\dot x(\tau) = 1$ gelten.
Das impliziert mit $x(0) = x_0$ die Lösung $x(t) = x_0 + t$.
Da $u(0, x_0) = u_0(x_0)$ und $x_0 = x(t) - t$ für alle $t > 0$ gilt, folgt insgesamt
\begin{align}\label{eq:adv:solution}
u(t, x) = u(0, x - t) = u_0(x - t).
\end{align}
Die wahre Lösung transportiert also nur die Anfangsdaten in die positive Raumrichtung.

Wenn wir eine Lösung $v$ von \eqref{eq:adv:pde} haben, so können wir Anfangswertproblem
\begin{align}\label{eq:adv:pdei}
\partial u(t,x) + \partial_x u(t,x) = v(t,x), \quad u(0,x) = 0
\end{align}
lösen, indem wir einfach $u(t,x) = t v(t,x)$ setzen. Denn dann gilt
{\small
\[ \partial_t u(t,x) = v(t,x) + t \partial_t v(t,x) = v(t,x) - t \partial_x v(t,x) = v(t,x) - \partial_x \bigl( t v(t,x) \bigr) = v(t,x) - \partial_x u(t,x), \]
}
womit zeigt ist, dass $u$ die Gleichung \eqref{eq:adv:pdei} erfüllt.