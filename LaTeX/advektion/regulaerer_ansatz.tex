\subsection{Regul"arer Ansatz}

Die Idee ist, dass die Gitterfunktion $v\colon G_h \to \R$ durch differenzierbare Abbildungen $u_i\colon \R^2 \to \R$ dargestellt werden kann, welche den Gitterpunkten $(t_n, x_i) \in G_h$ ausgewertet werden.
Wir rechnen hier zun"achst den ``regulären Ansatz'' vor, welcher vielen weiteren Rechnungen zu Grunde liegt. Dieser lautet
\begin{align}\label{eq:adv:regular:ansatz}
v^n_i = u_0(t_n, x_i) + h u_1(t_n, x_i) + o(h).
\end{align}
Außerdem erinnern wir uns, dass $t_{n+1} = t_n + \lambda h$ und $x_{i-1} = x_i - h$ gilt.
Wir rechnen nun die einzelnen Summanden aus \eqref{eq:adv:scheme_rechnung} aus.
Es gilt
\begin{align}\label{eq:adv:regular:sum1}
\frac {v^{n+1}_i - v^n_i}{\lambda h}
&= \partial_t u_0(t_n, x_i) + \frac {\lambda h}{2} \partial^2_t u_0(t_n, x_i) + h \partial_t u_1(t_n, x_i) + o(h). 
\end{align}
und
\begin{align}\label{eq:adv:regular:sum2}
\frac {v^n_i - v^n_{i-1}}{h} = \partial_x u_0(t_n, x_i) - \frac{h}{2} \partial^2_x u_0(t_n, x_i) + h \partial_x u_1(t_n, x_i) + o(h).
\end{align}
Setzt man dies in \eqref{eq:adv:scheme_rechnung} ein, so erhält man ein Gleichungssystem, welches für alle Zeit-Raum-Punkte $(t_n, x_i)$ und beliebiges $h > 0$ gilt.
Sei $(h_k)_{k \in \N}$ eine Nullfolge.
Dann gibt es für jedes Tupel $(t, x)$ eine Folge $(t_k, x_k)$ mit $(t_k, x_k) \in G_{h_k}$ und  $(t_k, x_k) \to (t,x)$ für $k \to \infty$.
In $O(1)$ bekommt man dann das Anfangswertproblem
\[ \partial_t u_0(t,x) + \partial_x u_0(t, x) = 0, \quad u(0, x) = U(x). \]
Foglich ist $u_0$ direkt die Lösung von \eqref{eq:adv:pde} und durch $u_0(t,x) = U(x-t)$ gegeben.
Wir können mithilfe von $u_1$ den Fehler des Verfahrens angeben.
In $O(h)$ erhalten wir eine bestimmende Gleichung f"ur $u_1$
\[ \partial_t u_1(t,x) + \partial_x u_1(t, x) = -\frac{\eps}{2} \partial^2_x u_0(t, x), \quad u(0, x) = 0. \]
Hierbei handelt es sich um die inhomogene Gleichung wie bei \eqref{eq:adv:pdei}.
$u_1$ ist deshalb durch $u_1(t,x) = - t \frac{\eps}{2} u_0(x - t)$  bestimmt.
$u_1$ verletzt somit die Sublinear-Growth Bedingung und d.\,h. unser Ansatz kann nur für kleine Zeiten $t$ stimmen.
Zusammenfassend folgt mit diesem Ansatz
\[ v^n_i = U(x_i - t_n) - \frac{h \eps}{2} t_n U(x_i - t_n) + o(h). \]