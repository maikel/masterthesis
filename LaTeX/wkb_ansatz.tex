\subsection {WKB Ansatz}

Wir versuchen hier nun einen sogenannten WKB-Ansatz.
Dieser sieht wie folgt aus:
\begin{align}\label{eq:wkb:ansatz}
v^n_k = u_0(t_n, x_k) + h e^{i (\omega_1 n - \omega_2 k)} \Bigl( u_1(t_n, x_k) + h u_2(t_n, x_k) \Bigr) + o(h^2),
\end{align}
wobei $n = \frac {t_n}{\lambda h}$ und $k = \frac {x_k}{h}$ gilt.
Wir betrachten das Verfahren außerdem für den Spezialfall $a = 1$ und $\lambda > 1$.
Es folgen nun die n"otigten Rechnungen.
\begin{align}
v^{n+1}_k &= u_0(t_{n+1}, x_k) + h e^{i (\omega_1 (n+1) - \omega_2 k)} \Bigl( u_1(t_{n+1}, x_k) + h u_2(t_{n+1}, x_k) \Bigr) + o(h^2)\\
\begin{split}
&= u_0(t_n, x_k) + \lambda h \, \partial_t u_0(t_n, x_k) + \frac {(\lambda h)^2} 2 \partial^2_t u_0(t_n, x_k)\\
&\qquad + h e^{i (\omega_1 n - \omega_2 k)} e^{i \omega_1} \Bigl( u_1(t_n, x_k) + \lambda h \, \partial_t u_1(t_n, x_k) + h u_2(t_n, x_k) \Bigr) + o(h^2).
\end{split}
\end{align}
\begin{align}
v^n_{k-1} &= u_0(t_n, x_{k-1}) + h e^{i (\omega_1 n - \omega_2 (k-1))} \Bigl( u_1(t_n, x_{k-1}) + h u_2(t_n, x_{k-1}) \Bigr) + o(h^2)\\
\begin{split}
&= u_0(t_n, x_k) - h \, \partial_x u_0(t_n, x_k) + \frac {h^2} 2 \partial^2_x u_0(t_n, x_k)\\
&\qquad + h e^{i (\omega_1 n - \omega_2 k)} e^{i \omega_2} \Bigl( u_1(t_n, x_k) - h \, \partial_x u_1(t_n, x_k) + h u_2(t_n, x_k) \Bigr) + o(h^2).
\end{split}
\end{align}
Und in das Verfahren $\frac {v^{n+1}_k - v^n_k}{\lambda h} + \frac {v^n_k - v^n_{k-1}}{h} = 0$ eingesetzt ergibt das in $O(1)$ die Gleichung
\begin{align}
\partial_t u_0(t_n, x_k) + \partial_x u_0(t_n, x_k) = - e^{i (\omega_1 n - \omega_2 k)} \cdot \underbrace{\frac {e^{i \omega_1}  - 1 + \lambda - \lambda e^{i \omega_2}}{\lambda}}_{=: \Lambda(\omega_1, \omega_2)} \cdot u_1(t_n, x_k).
\end{align}
Für $O(h)$ erhalten wir hingegen
{\small
\begin{align}
e^{i (\omega_1 n - \omega_2 k)} \Bigl( e^{i \omega_1} \partial_t u_1(t_n, x_k) + e^{i \omega_2} \partial_x u_1(t_n, x_k) \Bigr) = \frac {1 - \lambda} 2 \partial^2_x u_0(t_n, x_k) - e^{i (\omega_1 n - \omega_2 k)} \Lambda(\omega_1, \omega_2) u_2(t_n, x_k)
 \end{align}
 }