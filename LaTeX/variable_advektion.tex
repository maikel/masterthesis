\section{Variable Advektion}
% \secttoc 

\label{sec:varadv}

Wir betrachen die Differentialgleichung der eindimensionalen, variablen Advektion
\begin{align}\label{eq:varadv:pde}
u_t(t,x) + a(x) u_x(t,x) = 0, \quad u(0,x) = u_0(x)
\end{align}
mit glattem und beschränktem $a\colon \R \to \R$.
Wir wenden für diese Differentialgleichung das Upwindverfahren mit dem Gitter $G_h(n,i) = (t_n, x_i) = h \cdot (\lambda n,i)$ an.
Dieses Gitter ist rechteckig, falls $\lambda \neq 1$ gilt und wir erhalten die Gleichung
\begin{align}\label{eq:varadv:verfahren}
v^{n+1}_i = v^n_i - a_i \lambda \left( v^n_i - v^n_{i-1} \right).
\end{align}

% ===========================================================================
%                 1.1      Analytische Lösung
% ===========================================================================
\subsection{Analytische Lösung}

F"ur die kommende Asymptotik könnte es sinnvoll sein, die echte Lösung von \eqref{eq:varadv:pde} zumindest qualitativ zu kennen.
Die Lösung lässt sich analytisch mithilfe der Methode der Charakteristiken bestimmen.
Aus \eqref{eq:varadv:pde} folgt nämlich, dass jede Lösung $u$ dieser Gleichung konstant auf der Kurve $(\tau, x(\tau))$ ist, für die
\begin{align} \label{eq:varadv:ode}
\dot x(\tau) = a(x(\tau))
\end{align} gilt.
Sei nun $\phi\colon \R^2 \to \R$ der Fluss der Differentialgleichung \eqref{eq:varadv:ode}.
Dann gilt für jedes $x_0 \in \R$ per defintionem, dass $\phi(\, \cdot \,,x_0)$ das Anfangswertproblem \begin{align}\label{eq:varadv:ivp} \dot x(\tau) = a(x(\tau)), \quad x(0) = x_0 \end{align} löst.
Insbesondere gilt $\phi_{t+s} = \phi_t \circ \phi_s$ und $\phi_0 = \id$.
F"ur jedes $x \in \R$ gibt es ein $x_0 \in \R$, so dass $u(t,x) = u_0(x_0)$ gilt.
Wegen $x_0 = \phi(0,x_0)$ und $x = \phi(t, x_0)$ folgt damit \[ x_0 = \phi(0, x_0) = \phi(t - t, x_0) = \phi(-t, \phi(t, x_0)) = \phi(-t, x) \] und somit auch \begin{align}\label{eq:varadv:analytic_solution} u(t,x) = u_0(x_0) = u_0(\phi(-t, x)).\end{align}

\begin{example}
Für $a(x) = x$ folgt wegen des Anfangswertproblems $\dot x(\tau) = x, x(0) = x_0$, dass $u(t,x)$ konstant auf den Kurven der Form $x(\tau) = x_0 e^{\tau}$ ist.
Der Fluss $\phi$ ist uns dann durch $\phi(t,x) = x e^t$ gegeben. Wegen \eqref{eq:varadv:analytic_solution} folgt dann \[ u(x,t) = u_0(\phi(-t,x)) = u_0(x e^{-t}) \] als Lösung für \eqref{eq:varadv:pde}.
\end{example}
\begin{example} \label{ex:varadv:hat}
Setze
\[
a_\eps(x) = \begin{cases} 
    1 + \eps (1 - \abs{x}) & \abs{x} \leq 1\\
    1 & \text{ sonst.}
\end{cases}
\]
Dann ist $a$ absolut stetig und es gilt $a' \in L^\infty(\R)$.
Folglich ist $a$ Lipschitz-stetig und $\eqref{eq:varadv:ode}$ besitzt einen eindeutigen Fluss, welcher sich stückweise zusammensetzt. (TODO)
\end{example}

% \begin{remark}Finde glatte Beispiele, für die man einen Fluss hinschreiben und Numerik machen kann!\end{remark}

% \begin{remark}Die Flüsse für $a(x) = e^{-x^2}$ oder einer Glättung $a(x) = (\varphi * f)(x)$ für einen Glättungskern $\varphi$ und einer beliebigen Funktion $f$ scheinen mir nicht trivial.\end{remark}

% \begin{frage}Geht das überhaupt?\end{frage}

% \begin{frage}Ist $\phi$ genau so regulär wie $a$? ODE basics, technisch, aber vielleicht könnte man dann das Beispiel \ref{ex:varadv:hat} überall durchrechnen und konkrete Modelle testen.\end{frage}

Wenn wir eine Lösung $v$ von \eqref{eq:varadv:pde} haben, so können wir die folgende partielle Differentialgleichung
\begin{align}\label{eq:varadv:inhompde}
u_t(t,x) + a(x) u_x(t,x) = v(t,x), \quad u(0,x) = 0
\end{align}
lösen, indem wir einfach $u(t,x) = t v(t,x)$ setzen. Denn dann gilt
\[ u_t(t,x) = v(t,x) + t v_t(t,x) = v(t,x) - t a(x) v_x(t,x) \]
und wegen $t a(x) v_x(t,x) = a(x) (t v(t,x))_x = a(x) u_x(t,x)$ folgt damit \eqref{eq:varadv:inhompde}.

Weiterhin gilt, dass für jede Lösung $v$ von \eqref{eq:varadv:pde} die Funktion $u(t,x) = v_t(t,x)$ ebenfalls eine Lösung von \eqref{eq:varadv:pde} ist.
Und damit auch $\partial^n_t v$ für jedes $n \in \N$.
Denn es gilt $u_t = \partial_t (v_t) = \partial_t(- a v_x) = - a \partial_t v_x = -a \partial_x v_t = -a u_x$.

\subsection{Fallbeispiel}

Ich habe das Verfahren \eqref{eq:varadv:verfahren} mit GNU Octave implementiert:
\vspace{0.4cm}
\lstinputlisting[firstnumber=15, firstline=15, lastline=22]{../Programme/upwind.m}
Im Folgenden werden wir Beispiel \ref{ex:varadv:hat} näher untersuchen.
Dieses Beispiel interessiert uns, weil wir für die Advektionsgeschwindigkeit $a(x) = 1 + \eps (1 - \abs{x})$ die Lösung zu \eqref{eq:varadv:pde} kennen und hinschreiben können.
$u_0(x)$ und $a(x)$ haben wir wie folgt implementiert
\lstinputlisting[firstnumber=5, linerange={5-12}]{../Programme/step_hat.m}
und der Test lief unter diesen Bedingungen:
\lstinputlisting[firstnumber=18, linerange={18-36}]{../Programme/step_hat.m}
Die Daten können durch folgende Plots visualisiert werden.
\vspace{0.4cm}

\hspace{-1cm}
\begin{tikzpicture}
% \pgfplotsset{every axis/.append style={line width=1pt}}
\begin{axis}[
    title = {$\lambda a_i$ und $v^{130}_i$},
]
\addplot[dashed] table {../Programme/output/a_step_hat.dat};
\addplot[myblue] table [y index= 131] {../Programme/output/u_step_hat.dat};

\legend{$\lambda a_i$,$v^{130}_i$}
\end{axis}
\end{tikzpicture}
\begin{tikzpicture}
% \pgfplotsset{every axis/.append style={line width=1pt}}
\begin{axis}[
    title = {$\lambda a_i$ und $v^{163}_i$},
]
\addplot[dashed] table {../Programme/output/a_step_hat.dat};
\addplot[myblue] table [y index= 164] {../Programme/output/u_step_hat.dat};

\legend{$\lambda a_i$,$v^{163}_i$}
\end{axis}
\end{tikzpicture}

\hspace{-1cm}
\begin{tikzpicture}
% \pgfplotsset{every axis/.append style={line width=1pt}}
\begin{axis}[
    title = {$\lambda a_i$ und $v^{260}_i$},
]
\addplot[dashed] table {../Programme/output/a_step_hat.dat};
\addplot[myblue] table [y index= 261] {../Programme/output/u_step_hat.dat};

\legend{$\lambda a_i$,$v^{260}_i$}
\end{axis}
\end{tikzpicture}
\begin{tikzpicture}
% \pgfplotsset{every axis/.append style={line width=1pt}}
\begin{axis}[
    title = {$\lambda a_i$ und $v^{280}_i$},
]
\addplot[dashed] table {../Programme/output/a_step_hat.dat};
\addplot[myblue] table [y index= 280] {../Programme/output/u_step_hat.dat};

\legend{$\lambda a_i$,$v^{280}_i$}
\end{axis}
\end{tikzpicture}

\subsection{Regul"arer Ansatz}

Wir verwenden hier Methoden, die in \cite{Junk2004} vorgestellt wurden.
Dafür approximieren wir das Verfahren \eqref{eq:varadv:verfahren} zunächst mit dem regulärem Ansatz
\[ v^n_i = \u0(t_n, x_i) + h \u1(t_n, x_i) + O(h^2). \]
Dabei erkennt man dann hoffentlich auch welche Terme in der späteren Analyse eine Rolle spielen werden.
Für die Differenzen $v^{n+1}_i - v^n_i$ und $\lambda( v^n_i - v^n_{i-1} )$ in \eqref{eq:varadv:verfahren} folgt also

\begin{align} \label{eq:varadv:regular:diff1}
v^{n+1}_i - v^n_i
&= \u0(t_{n+1}, x_i) + h \u1(t_{n+1}, x_i) + O(h^2) - \left( \u0(t_n, x_i) + h \u1(t_n, x_i) + O(h^2)\right)\notag\\
&= \u0(t_{n+1}, x_i) - \u0(t_n, x_i) + h \left( \u1(t_{n+1}, x_i) - \u1(t_n, x_i)\right)\notag\\
&= \u0(t_n + \lambda h, x_i) - \u0(t_n, x_i) + h \left( \u1(t_n + \lambda h, x_i) - \u1(t_n, x_i)\right)\notag\\
&= \lambda h \, \u0_t (t_n, x_i) + \frac 12 \lambda^2 h^2 \u0_{tt}(t_n, x_i) + \lambda h^2 \u1_t(t_n, x_i) + O(h^3)\notag\\
&= \lambda h\, \Biggl( \u0_t (t_n, x_i) + h \left( \frac \lambda 2 \u0_{tt}(t_n, x_i) + \u1_t(t_n, x_i) \right) \Biggr) + O(h^3) 
\end{align}
und
% und für $\lambda (v^n_{i} - v^n_{i-1})$ folgt ganz analog
{\small\begin{align} \label{eq:varadv:regular:diff2}
\lambda (v^n_{i} - v^n_{i-1})
&= \lambda \Bigl( \u0(t_n, x_i) + h \u1(t_n, x_i) + O(h^2) - \left( \u0(t_n, x_{i-1}) + h \u1(t_n, x_{i-1}) + O(h^2) \right) \Bigr)\notag\\
&= \lambda \Bigl( \u0(t_n, x_i) - \u0(t_n, x_{i-1}) + h \left( \u1(t_n, x_i) - \u1(t_n, x_{i-1}) \right) \Bigr) \notag\\
&= \lambda \Bigl( \u0(t_n, x_i) - \u0(t_n, x_i-h) + h \left( \u1(t_n, x_i) - \u1(t_n, x_i-h) \right) \Bigr)\notag\\
&= \lambda \left( h \u0_x(t_n, x_i) - \frac 12 h^2 \u0_{xx}(t_n, x_i) + h^2 \u1_x(t_n, x_i) \right) + O(h^3)\notag\\
&= \lambda h \,\Biggl( \u0_x(t_n, x_i) + h \left( \u1_x(t_n, x_i) - \frac 12 \u0_{xx}(t_n, x_i) \right) \Biggr) + O(h^3).
\end{align}}

Setzt man nun \eqref{eq:varadv:regular:diff1} und \eqref{eq:varadv:regular:diff2} in das Verfahren \eqref{eq:varadv:verfahren} ein und teilt durch $\lambda h$, so erhält man die Gleichung
\begin{align*}
\begin{split}
O(h^2) ={}& \u0_t (t_n, x_i) + h \left( \frac \lambda 2 \u0_{tt}(t_n, x_i) + \u1_t(t_n, x_i) \right)\\
          & + a_i \u0_x(t_n, x_i) + h a_i \left( \u1_x(t_n, x_i) - \frac 12 \u0_{xx}(t_n, x_i) \right).
\end{split}\end{align*}
Dies gilt also f"ur alle Gitter $G_h$ und lassen wir nun $h \to 0$ zu, so erhalten wir, sortiert nach den Ordnungen von $h$, folgendes System von Differentialgleichungen für $\u i$:
\begin{equation}\label{eq:varadv:regular:pde_u0}
\u0_t(t, x) + a(x) \u0_x(t,x) = 0, \qquad \u0(0, x) = u_0(x)
\end{equation}
und
\begin{align}\label{eq:varadv:regular:pde_u1}
\u1_t(t, x) + a(x) \u1_x(t, x) &= a(x) \frac 12 \u0_{xx}(t, x) - \frac \lambda 2 \u0_{tt}(t, x) \qquad \quad \u1(0, x) = 0\notag\\
&= a(x) \cdot \frac{\u0_{xx}(t,x) - \lambda \left( a(x) \u0_x(t,x) \right)_x}{2},\\
\label{eq:varadv:regular:pde_u1_richtig}
&=  \frac {a(x) - \lambda a^2(x)}{2} \u0_{xx}(t,x) - \lambda \frac{a(x) a'(x)}{2} \u0_x(t,x)
\end{align}
wobei wir \eqref{eq:varadv:regular:pde_u0} benutzt haben, um \eqref{eq:varadv:regular:pde_u1} zu erhalten, denn es gilt
\begin{align*}
\u0_{tt}(t,x) &= \partial_t \u0_t(t,x)\\
              &= \partial_t \bigl( - a(x) \u0_x(t,x) \bigr)\\
              &= - a(x) \u0_{xt}(t,x)\\
              &= -a(x) \u0_{tx}(t,x)\\
              &= - a(x) \partial_x \bigl( \u0_t(t,x) \bigr)\\
              &= - a(x) \partial_x \bigl( - a(x) \u0_x(t,x) \bigr) = a(x) \left( a(x) \u0_x(t,x) \right)_x.
\end{align*}
% \begin{bemerkung}
% Der Fehlerterm in \eqref{eq:varadv:regular:pde_u1} stimmt für den Fall $a(x) = 1$ mit dem in \cite{Junk2004} überein.
% Ein gutes Zeichen!
% \end{bemerkung}
Da $\u0$ nun \eqref{eq:varadv:regular:pde_u0} löst kann man es auch als
\[ \u0(t,x) = u_0(\phi(-t,x)) \]
schreiben, wobei $\phi$ der Fluss der Differentialgleichung \eqref{eq:varadv:ode} sei.

% \begin{frage}
% Ist \eqref{eq:varadv:regular:pde_u1} so überhaupt lösbar?
% In \cite{Junk2004} haben die Autoren die Oszillation mithilfe der Gittervariablen eingeführt, weil die Gleichung dort eben nicht lösbar war.
% Das war dort ein Indiz dafür, dass das Verfahren mit der Gleichung inkompatibel ist, was hier ja auch der Fall ist!
% Trotzdem könnte es hier etwas komplizierter sein: Ich vermute man kann \eqref{eq:varadv:regular:pde_u1} als Summe von ``gutartiger'' Diffusion und ``bösartiger'' Oszillation schreiben.
% Ein gute Frage könnte also sein, ob man das in einen lösbaren Teil und in einen oszillierenden Teil trennen kann.
% Welche Diffusionsterme würde man rechts von \eqref{eq:varadv:regular:pde_u1} erwarten? Naja, siehe einfach normale Advektion?
% Wegen \eqref{eq:varadv:inhompde} könnte es eine gute Idee sein, es als Summe einer Lösung von \eqref{eq:varadv:pde} und einem Restterm zu schreiben.
% \end{frage}

% \begin{bemerkung}
% \eqref{eq:varadv:regular:pde_u1_richtig} habe ich nach der Frage aufgeschrieben.
% Vermutlich ist \eqref{eq:varadv:regular:pde_u1_richtig} die richtige Art das aufzuschreiben. Aufgabe: Pr"ufe nach, ob das schon gut plus böse ist.
% \end{bemerkung}

% \begin{bemerkung}
% Wahnsinn!! Der Hinweis von Herrn Klein!! $a(x) \sim 1 + o(\eps)\abs{x-x_0}$ um ein $x_0$ dann ist $a'(x) \sim o(\eps)$!
% Damit wird auch Beispiel \ref{ex:varadv:hat} attraktiver.
% \end{bemerkung}


\subsection{Oszillatorischer Ansatz}

Das Fallbeispiel zeigt, dass es auf einer sehr kurzen Zeitskala zu Oszillationen kommt, welche im Fall $\lambda a - 1 = \eps > 0$ größer werden.
Für diesen Fall lässt sich das Verfahren \eqref{eq:varadv:verfahren} auch in
\begin{align}
v^{n+1}_i - v^n_{i-1} = - \underbrace{\eps}_{> 0} \bigl( v^n_i - v^n_{i-1} \bigr)
\end{align}
umschreiben.
Für den Fall $\eps > 0$ erwarten wir also auf Gitterebene starke Vorzeichenwechsel.
Ziel ist es nun, diese Vorzeichenwechsel in unserem Ansatz widerzugeben.
Dafür führen wir in den regulären Ansatz eine Funktion $\tu1(n, i, t_n, x_i)$ ein, welche zusätzlich von den Gitterkoordinaten $(n,i)$ abhängt.
Dies entspräche den zusätzlichen Skalen $n = \frac t {\lambda h}$ und $i = \frac t h$.
Wir gehen jedoch davon aus, dass $\tu1$ unstetig in $n$ und $i$ ist.
Wir wählen daher zusammenfassend den Ansatz
\begin{align}\label{eq:varadv:osz:ansatz}
v^n_i = \u0(t_n, x_i) + h \u1(t_n, x_i) + h \tu1(n, i, t_n, x_i) + O(h^2)
\end{align}
für unsere Gitterfunktion $v$ und bestimmten $\tu1$ anhand der daraus resultierenden Bedingungen noch näher.
Wir setzen den Ansatz \eqref{eq:varadv:osz:ansatz} in \eqref{eq:varadv:verfahren} ein.
Daher rechnen wir die Differenzen $v^{n+1}_i - v^n_i$ und $\lambda a (v^n_i - v^n_{i-1})$ aus und entwickeln bis zur zweiten Ordnung von $h$.
Betrachte also
{\small
\begin{align}
\begin{split} \nonumber
v^{n+1}_i - v^n_i &= \u0(t_{n+1}, x_i) - \u0(t_n, x_i)\\
&\qquad + h \bigl( \u1(t_{n+1}, x_i) - \u1(t_n, x_i) \bigr)\\
&\qquad + h \bigl( \tu1(n+1, i, t_{n+1}, x_i) - \tu1(n, i, t_n, x_i) \bigr)
\end{split}\\
\begin{split} \nonumber
&= \lambda h \u0_t(t_n, x_i) + \frac {(\lambda h)^2} 2 \u0_{tt}(t_n, x_i)\\
&\qquad + h \bigl( \lambda h \u1_t(t_n, x_i) \bigr)\\
&\qquad + h \bigl( \tu1(n+1, i, t_n, x_i) - \tu1(n, i, t_n, x_i) + \lambda h \tu1_t(n+1, i, t_n, x_i) \bigr)\\
&\qquad + O(h^3)
\end{split}\\
\begin{split} \label{eq:osz:diff1}
&= \lambda h \cdot \Bigl( \u0_t(t_n, x_i) + \frac {\lambda h} 2 \u0_{tt}(t_n, x_i)\\
&\qquad + h \u1_t(t_n, x_i) + \frac{1}{\lambda} \bigl( \tu1(n+1, i, t_n, x_i) - \tu1(n, i, t_n, x_i)  \bigr) + h \tu1_t(n+1, i, t_n, x_i)\\
&\qquad + O(h^2) \Bigr).
\end{split}
\end{align}
}
und
{\small
\begin{align}
\begin{split}\nonumber
\lambda a (v^n_{i} - v^n_{i-1}) &= \lambda a \cdot \Bigl( \u0(t_n, x_i) - \u0(t_n, x_{i-1})\\
&\qquad + h \bigl( \u1(t_n, x_i) - \u1(t_n, x_{i-1}) \bigr)\\
&\qquad +  h \bigl( \tu1(n, i, t_n, x_i) - \tu1(n, i-1, t_n, x_{i-1}) \bigr) \Bigr)
\end{split}\\
\begin{split} \label{eq:osz:diff2}
&= \lambda h \cdot a  \Bigl( \u0_x(t_n, x_i) - \frac {h} 2 \u0_{xx}(t_n, x_i)\\
&\qquad + h \u1_x(t_n, x_i) +  \tu1(n, i, t_n, x_i) - \tu1(n, i-1, t_n, x_i) + h \tu1_x(n, i-1, t_n, x_i)\\
&\qquad + O(h^2) \Bigr)
\end{split}
\end{align}
}
Wenn wir dies nun in das Verfahren \eqref{eq:varadv:verfahren} einsetzen, erhalten wir in $O(1)$ die Gleichung
\begin{align}\label{eq:osz:ord1}
\begin{split}
\u0_t(t_n, x_i) + a \u0_x(t_n, x_i)
&= - \frac 1 \lambda \Bigl(  \tu1(n+1, i, t_n, x_i) - \tu1(n, i, t_n, x_i) \Bigr)
\\&\qquad - a \Bigl( \tu1(n, i, t_n, x_i) - \tu1(n, i-1, t_n, x_i) \Bigr)
\end{split}
\end{align}
Da wir von unserem Verfahren erwarten, dass es für sehr kleine $h$ die richtige Lösung von der PDE \eqref{eq:varadv:pde} annähert, leiten wir aus \eqref{eq:osz:ord1} ab, dass
\begin{align}\label{eq:osz:u1bed}
\tu1(n+1, i, t_n, x_i) - \tu1(n, i, t_n, x_i)
&= - a \lambda \Bigl( \tu1(n, i, t_n, x_i) - \tu1(n, i-1, t_n, x_i) \Bigr)
\end{align}
gilt.
Diese Gleichung zeigt, dass, wenn $\tu1$ stark im Ort oszilliert, dann auch in der Zeit.
Die Gleichung \eqref{eq:osz:u1bed} motiviert für $\tu1$ Ansatz
\begin{align} \label{eq:osz:u1ansatz}
\tu1(n, i, t_n, x_i) = (1 - 2 a \lambda)^n (-1)^i \Psi(t_n, x_i),
\end{align}
für eine glatte und beschr"ankte Funktion $\Psi(t_n, x_i)$.
Dieser Ansatz erfüllt die Bedingung \eqref{eq:osz:u1bed}, denn es gilt
\begin{align*}
\tu1(n+1, i, t_n, x_i) - \tu1(n, i, t_n, x_i)
&= (1 - 2 a \lambda)^{n+1} (-1)^i \Psi(t_n, x_i) - (1 - 2 a \lambda)^n (-1)^i \Psi(t_n, x_i)\\
&= (1 - 2 a \lambda)^n (-1)^i \Bigl( (1 - 2 a \lambda) \Psi(t_n, x_i) - \Psi(t_n, x_i) \Bigr)\\
&= - 2 a \lambda (1 - 2 a \lambda)^n (-1)^i \Psi(t_n, x_i)
\end{align*}
und
\begin{align*}
- \lambda a \cdot \Bigl( \tu1(n, i, t_n, x_i) - \tu1(n, i-1, t_n, x_i) \Bigr)
&= (1 - 2 a \lambda)^n (-1)^i \Psi(t_n, x_i) - (1 - 2 a \lambda)^n (-1)^{i-1} \Psi(t_n, x_i)\\
&= (1 - 2 a \lambda)^n (-1)^i \Bigl( \Psi(t_n, x_i) - (-1) \cdot \Psi(t_n, x_i) \Bigr)\\
&= - \lambda a (1 - 2 a \lambda)^n (-1)^i \Bigl( 2 \Psi(t_n, x_i) \Bigr)\\
&= - 2 a \lambda (1 - 2 a \lambda)^n (-1)^i \Psi(t_n, x_i).
\end{align*}
Hier sei erw"ahnt, dass die Faktoren $(-1)^i$ und so $(1 - 2 \lambda a)^n$ gerade so bestimmt wurden, dass die Bedingung \eqref{eq:osz:u1bed} gilt.
Dabei ist zu beobachten, dass der Term $1 - 2 \lambda a$ genau dann gr"o"ser als 1 ist, wenn $a \lambda > 1$ gilt.
Folglich tr"agt der oszillatorische Anteil, $\tu1$, f"ur den Fall $\lambda a > 1$ exponentiell in $n$ zur Gitterfunktion bei und verschwindet ebenso schnell im Fall $\lambda a < 1$, was auch in den Fallbeispielen zu beobachten ist.
Mit einem so bestimmtem $\tu1$ folgt für $\u0$ mit der Anfangsbedingung $\u0(0,x) = u_0(x)$ die L"osung $\u0(t,x) = u_0(x - at)$.

F"ur $O(h)$ erhalten wir aus den Ergebnissen \eqref{eq:osz:diff1} und \eqref{eq:osz:diff2} die Gleichung
\begin{align}
\u1_t(t_n, x_i) + a \u1_x(t_n, x_i) &= \frac {a ( 1 - \lambda a )}2 \u0_{xx}(t_n, x_i)\\
&\quad - \Bigl( \tu1_{t}(n+1, i, t_n, x_i) + \tu1_{x}(n, i-1, t_n, x_i) \Bigr).
\end{align}
und fordern
\begin{align*}
\tu1_{t}(n+1, i, t_n, x_i) + \tu1_{x}(n, i-1, t_n, x_i) = \frac {a ( 1 - \lambda a )}2 \u0_{xx}(t_n, x_i).
\end{align*}
Wenn wir nun den Ansatz \eqref{eq:osz:u1ansatz} einsetzen, erhalten wir f"ur $\Psi(t_n, x_i)$
{\small
\begin{align}
\tu1_{t}(n+1, i, t_n, x_i) + \tu1_{x}(n, i-1, t_n, x_i)
% &= (1 - 2 a \lambda)^{n+1} (-1)^i \Psi_t(t_n, x_i) + (1 - 2 a \lambda)^n (-1)^{i-1} \Psi_x(t_n, x_i)\\
&= (1 - 2a \lambda)^n (-1)^i \Bigl( (1 - 2a \lambda) \Psi_t(t_n, x_i) - \Psi_x(t_n, x_i) \Bigr)
\end{align}
} und damit
\begin{align}
(1 - 2a \lambda) \Psi_t(t_n, x_i) - \Psi_x(t_n, x_i) = \frac {a (1 - \lambda a)}{2 (1 - 2a \lambda)^n (-1)^i} \u0_{xx}(t_n, x_i)
\end{align}
% Nachdem wir nach den Ordnungen von $h$ sortiert haben, führt uns das zu dem Gleichungssystem
% \begin{align*}
% \u0_t(t_n, x_i) + a(x_i) \u0_x(t_n, x_i) = 0, \qquad \u0(0, x_i) = u_0(x_i)
% \end{align*}
% und 
% \begin{align*}
% \u1_t(t_n, x_i) + a(x_i) \u1_x(t_n, x_i) &= \frac {a(x_i) - \lambda a^2(x_i)}{2} \u0_{xx}(t_n,x_i) - \frac{a(x_i) a'(x_i)}{2} \u0_x(t_n,x_i)\\
% & \quad + \tu2(t_n, i) - \tu2(t_n, i-1), \qquad \u1(0, x_i) = 0.
% \end{align*}

% Ich würde gerne diesen Oszillatorischen Ansatz zunächst lokal für den Fall machen, dass $a(x) = a = 1+\delta$ derart ist, dass $a\lambda > 1$, also $a (1 - a\lambda) = - 2\eps$ für ein $\eps > 0$ gilt:
% \begin{align*}
% \u1_t(t_n, x_i) + a(x_i) \u1_x(t_n, x_i)
% &= - \eps \u0_{xx}(t_n,x_i) + \tu2(t_n, i) - \tu2(t_n, i-1), \qquad \u1(0, x_i) = 0.
% \end{align*}
% wobei $u^{(0)}(t_i, x_i)$ durch $u_0(x_i - a t_n)$ gegeben ist.
% Foglich wollen wir $\tu2$ so wählen, dass \[ \eps \partial^2_x u_0(x_i-at_n) = \tu2(t_n, i) - \tu2(t_n, i-1) \] für alle Gitterpunkte $i$ und Zeiten $t_n$ gilt.
% Schauen wir uns mal ein Beispiel dazu an und hoffen, dass wir etwas an Stellen mit großer Krümmung sehen:
% Ähnlich wie beim Fallbeispiel haben mir mit 
% \lstinputlisting[firstnumber=5, linerange={5-11}]{../Programme/sinus_const.m}
% und
% \lstinputlisting[firstnumber=17, linerange={17-35}]{../Programme/sinus_const.m}
% ein Beispiel gerechnet und kamen zu den folgenden Ergebnissen:

% \hspace{-1cm}
% \begin{tikzpicture}
% % \pgfplotsset{every axis/.append style={line width=1pt}}
% \begin{axis}[
%     title = {$\lambda a_i$ und $v^{5}_i$},
% ]
% \addplot[dashed] table {../Programme/output/a_sin_const.dat};
% \addplot[myblue] table [y index= 6] {../Programme/output/u_sin_const.dat};

% \legend{$\lambda a_i$,$v^{5}_i$}
% \end{axis}
% \end{tikzpicture}
% \begin{tikzpicture}
% % \pgfplotsset{every axis/.append style={line width=1pt}}
% \begin{axis}[
%     title = {$\lambda a_i$ und $v^{105}_i$},
% ]
% \addplot[dashed] table {../Programme/output/a_sin_const.dat};
% \addplot[myblue] table [y index= 106] {../Programme/output/u_sin_const.dat};

% \legend{$\lambda a_i$,$v^{105}_i$}
% \end{axis}
% \end{tikzpicture}

% \hspace{-1cm}
% \begin{tikzpicture}
% % \pgfplotsset{every axis/.append style={line width=1pt}}
% \begin{axis}[
%     title = {$\lambda a_i$ und $v^{113}_i$},
% ]
% \addplot[dashed] table {../Programme/output/a_sin_const.dat};
% \addplot[myblue] table [y index= 114] {../Programme/output/u_sin_const.dat};

% \legend{$\lambda a_i$,$v^{113}_i$}
% \end{axis}
% \end{tikzpicture}
% \begin{tikzpicture}
% % \pgfplotsset{every axis/.append style={line width=1pt}}
% \begin{axis}[
%     title = {$\lambda a_i$ und $v^{124}_i$},
% ]
% \addplot[dashed] table {../Programme/output/a_sin_const.dat};
% \addplot[myblue] table [y index= 125] {../Programme/output/u_sin_const.dat};

% \legend{$\lambda a_i$,$v^{124}_i$}
% \end{axis}
% \end{tikzpicture}